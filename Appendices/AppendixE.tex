% Appendix A

\chapter{Thermal properties of SnSe} % Main appendix title

\label{thermal-properties} % For referencing this appendix elsewhere, use \ref{AppendixA}

In Fig. \ref{thermal-analysis} we plot the group velocity, linewidth, cumulative lattice thermal conductivity
($\kappa_{l}^{c}(\omega)=\int_{0}^{\omega}\kappa(\omega)d\omega$) and the phonon density of states (PDOS).
\begin{figure}[th]
\includegraphics[width=\linewidth]{Figures/thermal_analysis.eps}
\caption{a) Absolute value of the phonon group velocity and absolute values of the group velocity Cartesian 
components. b) Average linewidth as a function of frequency for perturbative (P) (red) and non-perturbative (NP) 
(black) approaches. c) Diagonal components of the cumulative thermal conductivity as a function of frequency at $800$ 
K for perturbative (P) (red) and non-perturbative (NP) (black) approaches. d) Phonon density of states and the 
projections in Sn and Se atoms. The calculations are done within LDA in the experimental structure using 
$\Omega^{(S)}_{\mu}$ frequencies and non-perturbative 3BFC at $800$ K.}
\label{thermal-analysis}
\end{figure}
As we see in Fig. \ref{thermal-analysis}(a), due to the layered structure of the system, the group velocity is much 
lower in the out-of-plane direction $x$, leading to a reduced thermal conductivity. The two in-plane directions show 
very similar group velocities, as we would expect in this high symmetry phase. Interestingly, we can see in 
Fig. \ref{thermal-analysis}(b) that the non-perturbative linewidth has a similar dispersion to the perturbative one, 
however, it is homogeneously bigger in the whole frequency range. In agreement with the spectral function, this 
makes clear the need for a non-perturbative treatment of SnSe and yields a much lower thermal conductivity as we can 
see in Fig. \ref{thermal-analysis}(c). From Fig. \ref{thermal-analysis}(c) we can also extract that almost the entire 
contribution to the thermal conductivity is coming from vibrational modes with frequency smaller than $100$ cm$^{-1}$. Furthermore, more than $50\%$ of the thermal conductivity is coming from the acoustic modes
($\omega<75$ cm$^{1}$) which, looking at the phonon density of states in Fig. \ref{thermal-analysis}(d), is mainly 
coming from the vibrations of Sn atoms. The rest of the in-plane thermal conductivity is coming from the Sn and Se 
vibrations at around $90$ cm$^{-1}$. The situation is different for the out-of-plane component due to its very low 
group velocity in the $50-150$ cm$^{-1}$ frequency range. The rest of its contribution is coming from high energy 
modes where group velocities are higher. The contribution of the optical modes is strongly suppressed by the large 
phonon linewidths, which are a consequence of the strong anharmonicity of this compound. The contribution of the 
acoustic modes is particularly low, which ensures a low $\kappa_l$, specially because they can strongly scatter 
among themselves and with the $\Gamma_1$ mode. Therefore, the strongly anharmonic modes ($\Gamma_{1}$, $\Gamma_{2}$) 
provide an important scattering channel for lowering the thermal conductivity and making SnSe a very good 
thermoelectric material. 
