% Appendix A

\chapter{Volume effects on the harmonic phonons of SnSe} % Main appendix title

\label{volume-effects} % For referencing this appendix elsewhere, use \ref{AppendixA}

In this section we will analyze the volume effects in the harmonic phonon spectrum of $Cmcm$ SnSe. In 
Fig. \ref{harmonic-volume} we can see the LDA and
PBE phonon spectra in the theoretical and experimental structures.
\begin{figure}[th]
\begin{center}
\includegraphics[width=0.8\linewidth]{Figures/phonon-exp-dft.eps}
\caption{a) LDA harmonic phonon spectra in the theoretical and experimental structures. b) The same as a) within PBE. 
The LO-TO splitting is included in the calculations.}
\label{harmonic-volume}
\end{center}
\end{figure}
As we can see, within LDA, the difference between phonons in the theoretical and experimental structures is clearly 
visible. Due to the bigger volume in the experimental cell, there is a red shift of almost all the vibrational modes 
and furthermore, more imaginary frequencies arise at the $\Gamma$ and $Y$ points. Within PBE, the phonon spectra in 
the theoretical and experimental structures are much more similar, because the lattice parameters in these structures 
are very similar as well. These results are in agreement with our $T_{c}$ calculations shown in 
chapter \ref{Chapter5}, where $T_{c}$ is basically the same in the two structures within PBE and completely different 
within LDA. We conclude that the harmonic phonons of SnSe are very sensitive to the volume of the unit cell. \\

In order to compare with the calculations in Ref. \cite{dewandre2016two} and see whether the change from $c/b>1$ to 
$c/b<1$ is responsible for creating the instability at the point $Y$, we have done the phonon calculation using the 
lattice parameters of Ref. \cite{dewandre2016two}. The result is shown in Fig. \ref{exchangebc}.
\begin{figure}[th]
\begin{center}
\includegraphics[width=0.8\linewidth]{Figures/exchangebc-harmonic.eps}
\caption{PBE harmonic phonon spectra with the following lattice parameters: $a=22.492$, $b=8.094$ and $c=8.090$ (in 
Bohr units) (red) in one case and exchanging $b$ and $c$ (black) in the other. We have done the calculation in a 
$2\times2\times2$ supercell and Fourier interpolated to plot the spectrum. The LO-TO splitting is included in the 
calculations.}
\label{exchangebc}
\end{center}
\end{figure}
It is clear that the change in the lattice parameters does not alter the instability, as expected for such a small 
change.
