% Appendix A

\chapter{Interpolation of SSCHA dynamical matrices} % Main appendix title

\label{appendixB} % For referencing this appendix elsewhere, use \ref{AppendixA}

In the SSCHA forces are calculated in supercells and, therefore, anharmonic dynamical matrices are obtained in a 
commensurate $\boldsymbol{q}$ points grid. Computational costs increase enormously with the supercell size making 
dense sampling calculations extremely time-demanding. Therefore, whenever a fine sampling of the 1BZ is required, we 
have used the following interpolation scheme. \\

Let us call the anharmonic dynamical matrices obtained in the output of the SSCHA calculation in a coarse 
$\boldsymbol{q}$ point grid (small supercell) $\boldsymbol{D}_{coarse}^{S}(\boldsymbol{q})$. We will name the 
harmonic dynamical matrices obtained in the same grid as $\boldsymbol{D}_{coarse}(\boldsymbol{q})$. Our goal is to 
obtain anharmonic dynamical matrices in a finer grid, written as $\boldsymbol{D}_{fine}^{(S)}(\boldsymbol{q})$, by 
taking advantage of having the fine grid harmonic dynamical matrices $\boldsymbol{D}_{fine}(\boldsymbol{q})$ 
already computed (which are faster to obtain than the anharmonic ones). Assuming that
\begin{equation}
\tilde{\boldsymbol{D}}_{coarse}(\boldsymbol{q})\equiv\boldsymbol{D}_{coarse}^{(S)}(\boldsymbol{q})-
\boldsymbol{D}_{coarse}(\boldsymbol{q})
\end{equation}
is slowly varying in the reciprocal space ($i.e.$ the differences between the SSCHA and the harmonic force constant 
matrices are more localized in real space than the harmonic force constant matrices) we can interpolate 
$\tilde{\boldsymbol{D}}_{coarse}(\boldsymbol{q})$ using Fourier interpolation to the fine grid of our choice to 
obtain $\tilde{\boldsymbol{D}}_{fine}(\boldsymbol{q})$. Now, it is straightforward to obtain
\begin{equation}
\boldsymbol{D}_{fine}^{(S)}(\boldsymbol{q})=\boldsymbol{D}_{fine}(\boldsymbol{q})+
\tilde{\boldsymbol{D}}_{fine}(\boldsymbol{q}).
\end{equation}
