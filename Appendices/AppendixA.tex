% Appendix A

\chapter{Mathematical formulas and derivations for the membrane model} % Main appendix title

\label{appendixA} % For referencing this appendix elsewhere, use \ref{AppendixA}

In this appendix we include all the mathematical derivations and calculation details of the membrane model. \\

The general rotationally invariant potential for a membrane can be written as follows
\begin{equation}
 \label{full-potential}
 U=\frac{1}{2}\int_{\Omega}{d^{2}x\left(\kappa(\partial^{2}h)^{2}+\sum_{n\geq 2}u_{i_{1}j_{1}}\dots u_{i_{n}j_{n}}C_{i_{1}j_{1}\dots i_{n}j_{n}}^{(2n)}\right)},
\end{equation}
where $\Omega$ is the area of the membrane in equilibrium, $\kappa$ is the bending rigidity, $h$ is the
out-of-plane component of the displacement field and the rotationally invariant strain tensor $u_{ij}$ is defined
using the in-plane displacement field $u_{i}$
\begin{equation}
 \label{strain-tensor}
 u_{ij}=\frac{1}{2}(\partial_{i}u_{j}+\partial_{j}u_{i}+\partial_{i}\boldsymbol{u}\cdot\partial_{j}\boldsymbol{u}+\partial_{i}h\partial_{j}h).
\end{equation}
$C^{(2n)}_{i_{1}j_{1}\dots i_{n}j_{n}}$ is the generic elastic tensor of rank $2n$. In the previous expression the
subscripts label the 2d coordinates $x,y$ and the sum over indices is assumed. The second-order expansion of
Eq. \ref{full-potential} with respect to the phonon fields is given by
\begin{equation}
 \label{rank2-potential-strain}
 U=\frac{1}{2}\int_{\Omega}{d^{2}x\left(\kappa(\partial^{2}h)^{2}+C^{(4)}_{ijkl}u_{ij}u_{il}\right)},
\end{equation}
with $C^{(4)}_{ijkl}=\lambda\delta_{ij}\delta_{kl}+\mu(\delta_{ik}\delta_{jl}+\delta_{il}\delta_{jk})$. By using
equation \ref{strain-tensor} and $C_{ijkl}^{(4)}=C^{ijkl}$, equation \ref{rank2-potential-strain} can be rewritten as
\begin{multline}
 \label{rank2-potential-disp}
 U=\frac{1}{2}\int_{\Omega}d^{2}x[\kappa(\partial^{2}h)^{2}+C^{ijkl}\partial_{i}u_{j}\partial_{k}u_{l}+C^{ijkl}\partial_{i}u_{j}\partial_{k}h\partial_{l}h+\frac{C^{ijkl}}{4}\partial_{i}h\partial_{j}h\partial_{
 k}h\partial_{l}h+\\+\frac{C^{ijkl}}{2}\partial_{i}\boldsymbol{u}\cdot\partial_{j}\boldsymbol{u}\partial_{k}h\partial_{l}h+C^{ijkl}\partial_{i}u_{j}\partial_{k}\boldsymbol{u}\cdot\partial_{l}\boldsymbol{u}+
\frac{C^{ijkl}}{4}\partial_{i}\boldsymbol{u}\cdot\partial_{j}\boldsymbol{u}\partial_{k}\boldsymbol{u}\cdot\partial_{l}\boldsymbol{u}].
\end{multline}
If we allow the lattice spacing $a$ to be a variable, we can vary it by simply shifting the derivatives of the 
in-plane displacements according to $\partial_{i}u_{j}\rightarrow\partial_{i}u_{j}+\delta^{ij}\delta a$, where
$\delta a=(a-a_{0})/a_{0}$. Then, by taking into account periodic boundary conditions 
$\int_{\Omega}{d^{2}x\partial_{i}u_{j}}$ we can rewrite the potential
\begin{multline}
 \label{deltaa-potential}
 U\rightarrow U+2\Omega(1+\delta a)(\lambda+\mu)\delta a^{2}+(1+\frac{\delta a}{2})\delta a(\lambda+\mu)\int_{\Omega}{d^{2}x\partial_{k}h\partial_{k}h}+\frac{\delta a}{2}\int_{\Omega}{d^{2}xC^{ijkl}\partial_{i}u_{j}\partial_{k}h\partial_{
 l}h}+\\(1+\frac{\delta a}{2})\delta a\int_{\Omega}{d^{2}xC^{ijkl}\partial_{i}u_{j}\partial_{k}u_{l}}+(1+\frac{\delta a}{2})\delta a(\lambda+\mu)\int_{\Omega}{d^{2}x\partial_{k}\boldsymbol{
 u}\cdot\partial_{k}\boldsymbol{u}}+\frac{\delta a^{4}\Omega}{2}(\lambda+\mu)+\\\frac{\delta a}{4}\int_{\Omega}{d^{2}xC^{ijkl}[\partial_{i}\boldsymbol{u}\cdot\partial_{j}\boldsymbol{u}\partial_{k}u_{l}+\partial_{
 i}u_{j}\partial_{k}\boldsymbol{u}\cdot\partial_{l}\boldsymbol{u}]}.
\end{multline}
If we Fourier transform the potential and we apply $\langle u_{\alpha}(\boldsymbol{q})u_{\alpha}(-\boldsymbol{q})\rangle_{\rho_{\mathcal{H}}}=g[\Omega_{\alpha}^{(S)}(\boldsymbol{q})]$ ($\alpha=$LA,TA), $\langle h(\boldsymbol{q})h(-\boldsymbol{q})\rangle_{\rho_{\mathcal{H}}}=g[\Omega_{ZA}^{(S)}(\boldsymbol{q})]$, the SCHA free energy can be written 
as (we use $\hbar=k_{B}=1$)
\begin{multline}
\label{scha-f}
 \mathcal{F}(\mathcal{U})=F_{\mathcal{U}}+2\Omega(1+\delta a+\frac{\delta a^{2}}{4})(\lambda+\mu)\delta a^{2}+\frac{1}{2}\sum_{\boldsymbol{q}}\{g[\Omega_{ZA}^{(S)}(\boldsymbol{q})]\kappa|\boldsymbol{q}|^{4}+\\\{(\lambda+2\mu)
 g[\Omega_{LA}^{(S)}(\boldsymbol{q})]+\mu g[\Omega_{TA}^{(S)}(\boldsymbol{q})]\}|\boldsymbol{q}|^{2}+\\\frac{\lambda+2\mu}{4\Omega}\sum_{\boldsymbol{k}}g[\Omega_{ZA}^{(S)}(\boldsymbol{q})]g[\Omega_{ZA}^{(S)}(\boldsymbol{
 k})][|\boldsymbol{q}|^{2}|\boldsymbol{k}|^{2}+2(\boldsymbol{q}\cdot\boldsymbol{k})^{2}]+\\\frac{1}{2\Omega}\sum_{\boldsymbol{k}}g[\Omega_{ZA}^{(S)}(\boldsymbol{k})]\{g[\Omega_{LA}^{(S)}(\boldsymbol{q})]+g[\Omega_{TA}^{(S)}(
 \boldsymbol{q})]\}[\lambda|\boldsymbol{q}|^{2}|\boldsymbol{k}|^{2}+2\mu(\boldsymbol{q}\cdot\boldsymbol{k})^{2}]+\\+2(1+\frac{\delta a}{2})\delta a(\lambda+\mu)g[\Omega_{ZA}^{(S)}(\boldsymbol{q})]|\boldsymbol{q}|^{2}+2(1+\frac{
 \delta a}{2})\delta a\{(\lambda+2\mu)g[\Omega_{LA}^{(S)}(\boldsymbol{q})]+\mu g[\Omega_{TA}^{(S)}(\boldsymbol{q})]\}|\boldsymbol{q}|^{2}+\\2(1+\frac{\delta a}{2})\delta a(\lambda+\mu)\{g[\Omega_{LA}^{(S)}(\boldsymbol{
 q})]+g[\Omega_{TA}^{(S)}(\boldsymbol{q})]\}|\boldsymbol{q}|^{2}-g[\Omega_{ZA}^{(S)}(\boldsymbol{q})]\Phi_{ZA}^{(S)}(\boldsymbol{q})-\\\sum_{\alpha}g[\Omega_{\alpha}^{(S)}(\boldsymbol{q})]\Phi_{\alpha}^{(S)}(\boldsymbol{q})+ \\
 \frac{1}{4\Omega}\sum_{\boldsymbol{q}\boldsymbol{k}}[4g[\Omega_{LA}^{(S)}(
 \boldsymbol{q})]g[\Omega_{TA}^{(S)}(\boldsymbol{k})][\lambda(\boldsymbol{q}\cdot\boldsymbol{k})^{2}+\mu|\boldsymbol{q}|^{2}|\boldsymbol{k}|^{2}+\mu(\boldsymbol{q}\cdot\boldsymbol{k})^{2}](\hat{\boldsymbol{q}_{\perp}}\cdot\hat{
 \boldsymbol{k}})+ \\ 2g[\Omega_{LA}^{(S)}(\boldsymbol{q})]g[\Omega_{TA}^{(S)}(\boldsymbol{k})][\lambda|\boldsymbol{q}|^{2}|\boldsymbol{k}|^{2}+2\mu(\boldsymbol{q}\cdot\boldsymbol{k})^{2}]+ \\ (g[\Omega_{LA}^{(S)}(
 \boldsymbol{q})]g[\Omega_{LA}^{(S)}(\boldsymbol{k})]+g[\Omega_{TA}^{(S)}(\boldsymbol{q})]g[\Omega_{TA}^{(S)}(\boldsymbol{k})])[\lambda|\boldsymbol{q}|^{2}|\boldsymbol{k}|^{2}+2\mu(\boldsymbol{q}\cdot\boldsymbol{k})^{2}]+
 \\2(g[\Omega_{LA}^{(S)}(\boldsymbol{q})]g[\Omega_{LA}^{(S)}(\boldsymbol{k})])[\lambda(\boldsymbol{q}\cdot\boldsymbol{k})^{2}+\mu|\boldsymbol{q}|^{2}|\boldsymbol{k}|^{2}+\mu(\boldsymbol{q}\cdot\boldsymbol{k})^{2}](\hat{
 \boldsymbol{q}}\cdot\hat{\boldsymbol{k}}) + \\ 2(g[\Omega_{TA}^{(S)}(\boldsymbol{q})]g[\Omega_{TA}^{(S)}(\boldsymbol{k})])[\lambda(\boldsymbol{q}\cdot\boldsymbol{k})^{2}+\mu|\boldsymbol{q}|^{2}|\boldsymbol{k}|^{2}+\mu(
 \boldsymbol{q}\cdot\boldsymbol{k})^{2}](\hat{\boldsymbol{q}_{\perp}}\cdot\hat{\boldsymbol{k}_{\perp}})] \},
\end{multline}
where $g(\omega)=coth((\omega/2T))/(2\rho\omega)$ and $\Omega_{\alpha}^{(S)}(\boldsymbol{q})=\sqrt{\Phi_{\alpha}(\boldsymbol{q})/\rho}$ ($\alpha=$ZA,LA and, TA) is the SCHA frequency. $\rho$ is the mass density. In 
Eq. \ref{scha-f} the in plane displacement vector $\boldsymbol{u}(\boldsymbol{q}$) is separated into longitudinal 
and transversal components $\boldsymbol{u}(\boldsymbol{q})=u_{LA}(\boldsymbol{q})\hat{\boldsymbol{q}}+u_{TA}(\boldsymbol{q})\hat{\boldsymbol{q}_{\perp}}$, $\hat{\boldsymbol{q}}_{\perp}$ being the unitary vector perpendicular to 
$\hat{\boldsymbol{q}}$. $F_{\mathcal{U}}$ is the harmonic free energy of the harmonic trial potential $\mathcal{U}$. 
Now, by taking the derivative of the SCHA free energy with respect to the lattice constant and SCHA 2BFC, we arrive 
to the SCHA equations
\begin{multline}
 \frac{\partial\mathcal{F}(\mathcal{U})}{\partial\delta a}=0=2\Omega(2\delta a+3\delta a^{2}+\delta a^{3})(\lambda+
\mu)+\frac{1}{2}\sum_{\boldsymbol{q}}g[\Omega_{ZA}^{(S)}(\boldsymbol{q})]2(1+\delta a)(\lambda+\mu)|\boldsymbol{
 q}|^{2}\\+\frac{1}{2}\sum_{\boldsymbol{q}}g[\Omega_{LA}^{(S)}(\boldsymbol{q})][2(1+\delta a)(\lambda+2\mu)|\boldsymbol{q}|^{2}+2(1+\delta a)(\lambda+\mu)|\boldsymbol{q}|^{2}]+\\\frac{1}{2}\sum_{\boldsymbol{q}}g[\Omega_{
 TA}^{(S)}(\boldsymbol{q})][2(1+\delta a)\mu|\boldsymbol{q}|^{2}+2(1+\delta a/2)(\lambda+\mu)|\boldsymbol{q}|^{2}],
\end{multline}
\begin{multline}
 \Phi_{ZA}(\boldsymbol{q})=\kappa|\boldsymbol{q}|^{4}+2(1+\delta a/2)\delta a(\lambda+\mu)|\boldsymbol{q}|^{2}+\frac{\lambda+2\mu}{2\Omega}\sum_{\boldsymbol{k}}g[\Omega_{ZA}^{(S)}(\boldsymbol{k})][|\boldsymbol{q}|^{2}|\boldsymbol{
 k}|^{2}+2(\boldsymbol{q}\cdot\boldsymbol{k})^{2}]+\\\frac{1}{2\Omega}\sum_{\boldsymbol{k}}\{g[\Omega_{LA}^{(S)}(\boldsymbol{k})]+g[\Omega_{TA}^{(S)}(\boldsymbol{k})]\}[\lambda|\boldsymbol{q}|^{2}|\boldsymbol{k}|^{2}+2\mu(
 \boldsymbol{q}\cdot\boldsymbol{k})^{2}],
\end{multline}
\begin{multline}
 \Phi_{LA}(\boldsymbol{q})=(\lambda+2\mu)|\boldsymbol{q}|^{2}+2(1+\delta a/2)\delta a(\lambda+2\mu)|\boldsymbol{q}|^{2}+2(1+\delta a/2)\delta a(\lambda+\mu)|\boldsymbol{q}|^{2}\\+\frac{1}{2\Omega}\sum_{\boldsymbol{k}}g[\Omega_{ZA}^{(S)}(\boldsymbol{k})][\lambda|\boldsymbol{q}|^{2}|\boldsymbol{k}|^{2}+2\mu(\boldsymbol{q}\cdot\boldsymbol{k})^{2}]+\frac{1}{4\Omega}\sum_{\boldsymbol{k}}\{4g[\Omega_{TA}^{(S)}(\boldsymbol{k})][\lambda(\boldsymbol{q}\cdot\boldsymbol{
 k})^{2}+\\\mu|\boldsymbol{q}|^{2}|\boldsymbol{k}|^{2}+\mu(\boldsymbol{q}\cdot\boldsymbol{k})^{2}](\hat{\boldsymbol{q}_{\perp}}\cdot\hat{\boldsymbol{k}})+ \\ 2g[\Omega_{TA}^{(S)}(\boldsymbol{k})][\lambda|\boldsymbol{q}|^{2}|\boldsymbol{
 k}|^{2}+2\mu(\boldsymbol{q}\cdot\boldsymbol{k})^{2}]+ \\ 2g[\Omega_{LA}^{(S)}(\boldsymbol{k})][\lambda|\boldsymbol{q}|^{2}|\boldsymbol{k}|^{2}+2\mu(\boldsymbol{q}\cdot\boldsymbol{k})^{2}]+ \\ 4g[\Omega_{LA}^{(S)}(\boldsymbol{
 k})][\lambda(\boldsymbol{q}\cdot\boldsymbol{k})^{2}+\mu|\boldsymbol{q}|^{2}|\boldsymbol{k}|^{2}+\mu(\boldsymbol{q}\cdot\boldsymbol{k})^{2}](\hat{\boldsymbol{q}}\cdot\hat{\boldsymbol{k}})\}
\end{multline}
and,
\begin{multline}
 \Phi_{TA}(\boldsymbol{q})=\mu|\boldsymbol{q}|^{2}+2(1+\delta a/2)\delta a\mu|\boldsymbol{q}|^{2}+2(1+\delta a/2)\delta a(\lambda+\mu)|\boldsymbol{q}|^{2}\\+\frac{1}{2\Omega}\sum_{\boldsymbol{k}}g[\Omega_{ZA}^{(S)}(\boldsymbol{k})][\lambda|\boldsymbol{q}|^{2}|\boldsymbol{k}|^{2}+2\mu(\boldsymbol{q}\cdot\boldsymbol{k})^{2}]+\frac{1}{4\Omega}\sum_{\boldsymbol{k}}\{
 4g[\Omega_{TA}^{(S)}(\boldsymbol{k})][\lambda(\boldsymbol{q}\cdot\boldsymbol{k})^{2}+\\\mu|\boldsymbol{q}|^{2}|\boldsymbol{k}|^{2}+\mu(\boldsymbol{q}\cdot\boldsymbol{k})^{2}](\hat{\boldsymbol{q}_{\perp}}\cdot\hat{
 \boldsymbol{k}_{\perp}})]+\\4g[\Omega_{LA}^{(S)}(\boldsymbol{k})][\lambda(\boldsymbol{q}\cdot\boldsymbol{k})^{2}+\mu|\boldsymbol{q}|^{2}|\boldsymbol{k}|^{2}+\mu(\boldsymbol{q}\cdot\boldsymbol{k})^{2}](\hat{\boldsymbol{
 q}_{\perp}}\cdot\hat{\boldsymbol{k}})+ \\ 2g[\Omega_{SCHA}^{(TA)}(\boldsymbol{k})][\lambda|\boldsymbol{q}|^{2}|\boldsymbol{k}|^{2}+2\mu(\boldsymbol{q}\cdot\boldsymbol{k})^{2}]  \}.
\end{multline}
We have solved this equations in a circular grid with $60\times60$ $\boldsymbol{q}$ points with a radius of $1$ 
$\AA^{-1}$ by applying the Newton-Raphson method. We have checked the convergency of the results with denser grids. 
The parameters which have been calculated by using the atomistic empirical potential are the following: 
$\lambda=4.32$ eV$\AA^{-2}$, $\mu=9.32$ eV$\AA^{-2}$, $\kappa=1.49$ eV and, $\rho/\hbar^{2}=1090.64$ eV$\AA^{-4}$. \\

Regarding the second derivative of the free energy, the physical phonons in the static approach, the most general
formula for the correction to the SCHA phonon frequencies ($\boldsymbol{D}^{corr}(-\boldsymbol{q},\boldsymbol{q})=\boldsymbol{D}^{(F)}(-\boldsymbol{q},\boldsymbol{q})-\boldsymbol{D}^{(S)}(-\boldsymbol{q},\boldsymbol{q})$) is
\begin{equation}
 \label{correction}
 D_{\alpha\beta}^{corr}(-\boldsymbol{q},\boldsymbol{q})=\sum_{\gamma\delta\epsilon\zeta}\sum_{\boldsymbol{p}\boldsymbol{k}}\overset{(3)}{D}{}_{\alpha\gamma\delta}(-\boldsymbol{q},\boldsymbol{p},
 \boldsymbol{q}-\boldsymbol{p})[1-\overset{(4)}{D}{}_{\gamma\delta\epsilon\zeta}(-\boldsymbol{p},\boldsymbol{p}-\boldsymbol{q},\boldsymbol{k},\boldsymbol{q}-\boldsymbol{k})]^{-1}\overset{(3)}{D}{}_{
 \epsilon\zeta\beta}(-\boldsymbol{k},\boldsymbol{k}-\boldsymbol{q},\boldsymbol{q}),
\end{equation}
where the subindexes run on the normal coordinates $\alpha,\beta,\gamma,\delta,\epsilon,\zeta=h,u_{LA},u_{TA}$ and 
the dynamical matrices in normal coordinates are defined as
\begin{equation}
 \label{third-order}
\overset{(3)}{D}{}^{(S)}_{\alpha\beta\gamma}(\boldsymbol{q},\boldsymbol{k},\boldsymbol{p})=\frac{1}{\rho^{3/2}}\left\langle\frac{\partial^{3}V}{\partial\alpha(\boldsymbol{q})\partial\beta(\boldsymbol{k})\partial\gamma(\boldsymbol{
 p})}\right\rangle_{\mathcal{H}}\sqrt{G_{\beta\gamma}(\boldsymbol{k},\boldsymbol{p})},
\end{equation}
\begin{equation}
 \label{fourth-order}
\overset{(4)}{D}{}^{(S)}_{\alpha\beta\gamma\epsilon}(\boldsymbol{q},\boldsymbol{q}',\boldsymbol{k},\boldsymbol{k}')=\frac{1}{\rho^{2}}\left\langle\frac{\partial^{4}V}{\partial\alpha(\boldsymbol{q})\partial\beta(\boldsymbol{
 q}')\partial\gamma(\boldsymbol{k})\partial\epsilon(\boldsymbol{k}')}\right\rangle_{\mathcal{H}}\sqrt{G_{\alpha\beta}(\boldsymbol{q},\boldsymbol{k})G_{\gamma\epsilon}(\boldsymbol{q}',\boldsymbol{k}')}.
\end{equation}
The matrix $G_{\alpha\beta}(\boldsymbol{q},\boldsymbol{k})$ is defined as
\begin{equation}
	G_{\alpha\beta}(\boldsymbol{q},\boldsymbol{k})=\frac{F(0,\Omega_{\alpha}^{(S)}(\boldsymbol{q}),\Omega_{\beta}^{(S)}(\boldsymbol{k}))}{\Omega_{\alpha}^{(S)}(\boldsymbol{q})\Omega_{\beta}^{(S)}(\boldsymbol{k})},
\end{equation}
$F(0,\Omega_{\alpha}^{(S)}(\boldsymbol{q}),\Omega_{\beta}^{(S)}(\boldsymbol{k}))$ being the function defined in
Eq. \ref{raffaello-function}. We are interested in the corrections to the out-of-plane modes, therefore, we are
interested in the terms of the type
\begin{equation}
D_{hh}^{corr}(-\boldsymbol{q},\boldsymbol{q})=\sum_{\gamma\delta\epsilon\zeta}\sum_{\boldsymbol{p}\boldsymbol{k}}\overset{(3)}{D}{}^{(S)}_{h\gamma\delta}(-\boldsymbol{q},\boldsymbol{p},
\boldsymbol{q}-\boldsymbol{p})[1-\overset{(4)}{D}{}^{(S)}_{\gamma\delta\epsilon\zeta}(-\boldsymbol{p},\boldsymbol{p}-\boldsymbol{q},\boldsymbol{k},\boldsymbol{q}-\boldsymbol{k})]^{-1}\overset{(3)}{D}{}^{(S)}_{
 \epsilon\zeta h}(-\boldsymbol{k},\boldsymbol{k}-\boldsymbol{q},\boldsymbol{q}),
\end{equation}
By looking at Eq. \ref{deltaa-potential} we can see that only the terms of the type $\int_{\Omega}{d^{2}xC^{ijkl}\partial_{i}u_{j}\partial_{k}h\partial_{l}h}$ will contribute to the statistical average in
Eq. \ref{third-order}. Therefore, Eq. \ref{correction} can be rewritten as
\begin{equation}
\label{correction-simple}
D_{hh}^{corr}(-\boldsymbol{q},\boldsymbol{q})=4\sum_{\alpha\beta}\sum_{\boldsymbol{p}\boldsymbol{k}}\overset{(3)}{D}{}^{(S)}_{hh\alpha}(-\boldsymbol{q},\boldsymbol{p},\boldsymbol{q}-\boldsymbol{p})[
1-\overset{(4)}{D}{}^{(S)}_{h\alpha h\beta}(-\boldsymbol{p},\boldsymbol{p}-\boldsymbol{q},\boldsymbol{k},\boldsymbol{q}-\boldsymbol{k})]^{-1}\overset{(3)}{D}{}^{(S)}_{hh\beta}(\boldsymbol{q},-\boldsymbol{k},\boldsymbol{k}-\boldsymbol{q}),
\end{equation}
where now the subindexes only run in $\alpha,\beta=u_{LA},u_{TA}$. Now, we can calculate the statistical averages
\begin{equation}
 \left\langle\frac{\partial^{3}V}{\partial h(\boldsymbol{k}_{1})\partial h(\boldsymbol{k}_{2})\partial u_{LA}(\boldsymbol{k}_{3})}\right\rangle_{\mathcal{H}}=\frac{1+\delta a}{\sqrt{\Omega}}\delta_{\boldsymbol{k}_{1}+\boldsymbol{
 k}_{2}+\boldsymbol{k}_{3},0}\left[\lambda|\boldsymbol{k}_{3}|\boldsymbol{k}_{1}\cdot\boldsymbol{k}_{2}+2\mu\frac{(\boldsymbol{k}_{3}\cdot\boldsymbol{k}_{1})(\boldsymbol{k}_{3}\cdot\boldsymbol{k}_{2})}{|\boldsymbol{k}_{3}|}\right],
\end{equation}
\begin{equation}
 \left\langle\frac{\partial^{3}V}{\partial h(\boldsymbol{k}_{1})\partial h(\boldsymbol{k}_{2})\partial u_{TA}(\boldsymbol{k}_{3})}\right\rangle_{\mathcal{H}}=\frac{\mu(1+\delta a)}{\sqrt{\Omega}}\delta_{\boldsymbol{k}_{
 1}+\boldsymbol{k}_{2}+\boldsymbol{k}_{3},0}\left[\frac{(\boldsymbol{k}_{3}\cdot\boldsymbol{k}_{1})(\boldsymbol{k}_{3\perp}\cdot\boldsymbol{k}_{2})+(\boldsymbol{k}_{3}\cdot\boldsymbol{k}_{2})(\boldsymbol{k}_{3\perp}\cdot
 \boldsymbol{k}_{1})}{|\boldsymbol{k}_{3}|}\right],
\end{equation}
\begin{multline}
\left\langle\frac{\partial^{4}V}{\partial h(\boldsymbol{k}_{1})\partial h(\boldsymbol{k}_{2})\partial u_{LA}(\boldsymbol{k}_{3})\partial u_{LA}(\boldsymbol{k}_{4})}\right\rangle_{\mathcal{H}}=\frac{1}{\Omega}\delta_{\boldsymbol{
 k}_{1}+\boldsymbol{k}_{2}+\boldsymbol{k}_{3}+\boldsymbol{k}_{4},0}\frac{\boldsymbol{k}_{3}\cdot\boldsymbol{k}_{4}}{|\boldsymbol{k}_{3}||\boldsymbol{k}_{4}|}[\lambda(\boldsymbol{k}_{3}\cdot\boldsymbol{k}_{4})(\boldsymbol{k}_{
 1}\cdot\boldsymbol{k}_{2})+\\\mu(\boldsymbol{k}_{3}\cdot\boldsymbol{k}_{1})(\boldsymbol{k}_{4}\cdot\boldsymbol{k}_{2})+\mu(\boldsymbol{k}_{3}\cdot\boldsymbol{k}_{2})(\boldsymbol{k}_{4}\cdot\boldsymbol{k}_{1})],
\end{multline}
\begin{multline}
 \left\langle\frac{\partial^{4}V}{\partial h(\boldsymbol{k}_{1})\partial h(\boldsymbol{k}_{2})\partial u_{TA}(\boldsymbol{k}_{3})\partial u_{TA}(\boldsymbol{k}_{4})}\right\rangle_{\mathcal{H}}=\frac{1}{\Omega}\delta_{\boldsymbol{
 k}_{1}+\boldsymbol{k}_{2}+\boldsymbol{k}_{3}+\boldsymbol{k}_{4},0}\frac{\boldsymbol{k}_{3\perp}\cdot\boldsymbol{k}_{4\perp}}{|\boldsymbol{k}_{3}||\boldsymbol{k}_{4}|}[\lambda(\boldsymbol{k}_{3}\cdot\boldsymbol{k}_{4})(\boldsymbol{k}_{
 1}\cdot\boldsymbol{k}_{2})+\\\mu(\boldsymbol{k}_{3}\cdot\boldsymbol{k}_{1})(\boldsymbol{k}_{4}\cdot\boldsymbol{k}_{2})+\mu(\boldsymbol{k}_{3}\cdot\boldsymbol{k}_{2})(\boldsymbol{k}_{4}\cdot\boldsymbol{k}_{1})],
\end{multline}
and
\begin{multline}
 \left\langle\frac{\partial^{4}V}{\partial h(\boldsymbol{k}_{1})\partial h(\boldsymbol{k}_{2})\partial u_{LA}(\boldsymbol{k}_{3})\partial u_{TA}(\boldsymbol{k}_{4})}\right\rangle_{\mathcal{H}}=\frac{1}{\Omega}\delta_{\boldsymbol{
 k}_{1}+\boldsymbol{k}_{2}+\boldsymbol{k}_{3}+\boldsymbol{k}_{4},0}\frac{\boldsymbol{k}_{3}\cdot\boldsymbol{k}_{4\perp}}{|\boldsymbol{k}_{3}||\boldsymbol{k}_{4}|}[\lambda(\boldsymbol{k}_{3}\cdot\boldsymbol{k}_{4})(\boldsymbol{k}_{
 1}\cdot\boldsymbol{k}_{2})+\\\mu(\boldsymbol{k}_{3}\cdot\boldsymbol{k}_{1})(\boldsymbol{k}_{4}\cdot\boldsymbol{k}_{2})+\mu(\boldsymbol{k}_{3}\cdot\boldsymbol{k}_{2})(\boldsymbol{k}_{4}\cdot\boldsymbol{k}_{1})].
\end{multline}
The equations cannot be further simplified but we have all the ingredients to calculate them numerically. We have
checked numerically that, as in the atomistic case, the contribution of $\overset{(4)}{\boldsymbol{D}}$ is
completely negligible. \\

By neglecting the fourth-order terms containing in-plane displacement fields in Eq. \ref{full-potential}, the SCHA 
can be applied analytically in this model. The SCHA equations simplify to
\begin{equation}
\label{scha-tommaso1}
\delta a=-\frac{1}{4\Omega}\sum_{\boldsymbol{q}}|\boldsymbol{q}|^{2}g[\Omega_{ZA}^{(S)}(\boldsymbol{q})],
\end{equation}
\begin{equation}
\label{scha-tommaso2}
\Phi_{ZA}(\boldsymbol{q})=\kappa|\boldsymbol{q}|^{4}+2\delta a(\lambda+\mu)|\boldsymbol{q}|^{2}+\frac{\lambda+2\mu}{2\Omega}\sum_{\boldsymbol{k}}g[\Omega_{ZA}^{(S)}(\boldsymbol{k})][|\boldsymbol{q}|^{2}|\boldsymbol{k}|^{2}+2(\boldsymbol{q}\cdot\boldsymbol{k})^{2}].
\end{equation}
By inserting Eq. \ref{scha-tommaso1} in Eq. \ref{scha-tommaso2} and considering the infinite volume limit 
($\Omega\rightarrow\infty$), we obtain
\begin{equation}
\label{tommaso1}
\Phi_{ZA}(\boldsymbol{q})=\kappa|\boldsymbol{q}|^{4}+\gamma|\boldsymbol{q}|^{2},
\end{equation}
where $\gamma$ is given by the solution of
\begin{equation}
\label{tommaso2}
\gamma=\gamma\frac{\lambda+3\mu}{16\pi\kappa\sqrt{\rho\kappa}}\int_{0}^{\Lambda\sqrt{\kappa/\gamma}}ds\frac{s^{2}coth[\gamma s\sqrt{1+s^{2}}/(2T\sqrt{\rho\kappa})]}{\sqrt{1+s^{2}}}.
\end{equation}
$\Lambda$ is an ultraviolet cutoff that avoids divergencies.
Eqs. \ref{tommaso1} and \ref{tommaso2} show that the dispersion of the SCHA ZA modes is linear. By calculating the 
correction for getting the physical phonons in the static approach in Eq. \ref{correction-simple} (in this case the 
fourth-order tensor is $0$) the result is
\begin{equation}
\Phi_{ZA}^{(F)}(\boldsymbol{q})=\kappa|\boldsymbol{q}|^{4}+(\gamma-\sigma)|\boldsymbol{q}|^{2}+O(|\boldsymbol{q}|^{4}),
\end{equation}
where at $T=0$ K
\begin{equation}
\sigma=\frac{\rho\sqrt{\gamma}}{8\pi\kappa^{3/2}}\sum_{\alpha=LA,TA}v_{\alpha}f(\Lambda\sqrt{\kappa/\gamma},v_{\alpha}\sqrt{\rho/\gamma}),
\end{equation}
with
\begin{equation}
f(x,y)=\int_{0}^{x}ds\frac{s^{2}}{\sqrt{1+s^{2}}[\sqrt{1+s^{2}}+y]}.
\end{equation}
By setting the ultraviolet cutoff to the value of the Debye momentum, 
$\Lambda=\sqrt{\frac{8\pi}{3^{1/2}a_{0}}}=1.55\AA$, we obtain $1-\sigma/\gamma=20\%$. This means that the linear 
component of the Physical frequencies turns out to be a factor of $40\%$ smaller than the one of the SCHA frequency. 
The non zero linear term in the physical frequencies appears because neglecting the fourth-order terms including 
in-plane displacements break the rotational invariance of the potential.
