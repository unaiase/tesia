% Chapter 1

\chapter*{Conclusions} % Main chapter title

\label{Chapter9} % For referencing the chapter elsewhere, use \ref{Chapter1} 
\addcontentsline{toc}{part}{Conclusions}

%----------------------------------------------------------------------------------------

The main goal of this thesis was to understand and predict the phase transitions that occur in high efficiency 
thermoelectric materials SnSe and SnS. Another purpose was to calculate the lattice thermal conductivity in these 
materials to understand the experimental results and find the origin of such a low thermal conductivity. \\

We have seen that both SnSe and SnS suffer second-order phase transitions from the low symmetry low temperature 
$Pnma$ phase to the high symmetry high temperature $Cmcm$ phase. This means that they suffer continuous displacive 
phase transitions. As it is expected in a second-order phase transition, the thermal conductivity is very similar 
for both phases close to the transition temperature. This together with the fact that the $Cmcm$ phase has a 
higher $PF$ due to a smaller band gap, makes these materials to have the maximum $ZT$ just after the phase 
transition. Therefore, the role of the phase transition is key for having a high thermoelectric efficiency. \\

The thermal conductivity of both materials is ultralow, around 0.3-1.0 W/mK at around $800$ K. We have seen that the 
main contribution to the lattice thermal conductivity comes from the acoustic modes. Their values are low due to the 
high anharmonicity of the material and low group velocities. Perturbation theory breaks down in the calculation of 
second-order and high-order force-constants and the inclusion of anharmonicity at a non-pertubative level is crucial 
for a proper description of the lattice thermal conductivity. We have seen that we only get results in agreement 
with experiments when we include anharmonic effects at a non-perturbative level. \\

From a qualitative point of view our transition temperature $T_{c}$ calculations are in agreement with experiments. 
We have seen that the quantitative agreement in $T_{c}$ is out of the scope of this thesis. This happens because the 
vibrational properties are very sensitive to the volume of the unit cell. Regarding the lattice thermal conductivity 
we get a very good agreement with experimental results and this has helped clarifying many experimental 
uncertainties. In the work by Zhao et al.\cite{zhao2014ultralow} they showed ultralow and isotropic values of the 
lattice thermal conductivity, lower than our calculations for the in-plane results. These results lead to 
$ZT_{max}=2.6$, the maximum $ZT$ so far, for an in-plane direction. Later 
experiments\cite{ibrahim2017reinvestigation,wei2019thermoelectric} have shown that the in-plane thermal conductivity 
is higher and are in good agreement with our results. It seems that the origin of the low in-plane lattice thermal 
conductivity of Zhao et al. is due to Sn vacancies. The later experimental works have shown that, due to the higher 
in-plane lattice thermal conductivity, the $ZT_{max}$ of intrinsic SnSe is around 1. Our calculations suggest that 
SnSe and SnS have very similar thermoelectric properties and, therefore, SnS may be an efficient thermoelectric 
material in the high temperature phase. This is also partially supported by experiments as both systems seem to have 
very similar $ZT$s in the low temperature phase close to the phase transition. Of course, this is not true if we 
consider the results by Zhao et al. \\

Another important conclusion of this thesis is related to the experimental measurement of the phonon spectral 
function. We have seen that the Lorentzian approximation breaks down for some vibrational modes in SnSe and SnS. 
This is in principle not an unexpected results as we have shown that they are strongly anharmonic materials. The 
break down of the Lorentzian approximation given by the strong anharmonicity creates shoulders and double peaks in 
the phonon spectral function, which makes it more difficult to understand. These results are key for the 
understanding of future inelastic scattering experiments in this kind of materials. \\

Another purpose of this thesis was to study the ferroelectric phase transition and thermal conductivity in monolayer 
SnSe. We have seen that monolayer SnSe suffers a second-order phase transition from the low symmetry low temperature 
$Pnm2_{1}$ to the high symmetry high temperature $Pnmm$ phase. The low symmetry $Pnm2_{1}$ phase has no inversion 
symmetry and, therefore, it is a ferroelectric phase. As it happens in the bulk case, the agreement with 
experiments is not quantitatively good as experimentally it is measured to be in the $Pnm2_{1}$ phase at 300 K. 
According to our calculations the phase transition occurs at around 110 K. Regarding the calculation of the lattice 
thermal conductivity, we have found obstacles partially related to the two dimensional nature of the system. This has 
motivated the last chapter of this thesis. \\

The acoustic out-of-plane modes (ZA) of planar materials have a quadratic dispersion within the harmonic 
approximation. This dispersion is fixed by the two dimensional (2D) symmetry. This dispersion creates problematic 
results, such as, diverging atomic displacements as a function of the sample size or finite line width of acoustic 
in-plane phonons (LA/TA) at vanishing momenta. Therefore, there seems to be a conflict between the symmetry forced 
quadratic dispersion, which is expected to be measured experimentally, and the magnitudes that can be calculated 
using it. We have shown that such a conflict does not exist, anharmonicity is the answer to the problem. We have 
applied the SSCHA to include anharmonic effects. In order to calculate physical properties, such as, atomic 
displacements or line widths, the SSCHA formalism uses the SSCHA auxiliary phonons frequencies. We have seen that 
the SSCHA ZA phonons have a linear dispersion, as a result, diverging atomic displacements and finite line widths 
disappear. However, when the physical phonons are calculated within the SSCHA, the ones that are supposed to be 
measured experimentally, the quadratic dispersion is recovered as expected by symmetry. \\

To sum up, we have shown that non-perturbative anharmonicity is key for understanding all the physics that appear 
in the systems under study in this thesis. However, there are still open questions and work to do in this field. For 
example, the quantitative agreement of the transition temperature is one of them. As we have seem, the phonon 
frequencies are extremely sensitive to the volume of the unit cell in monochalcogenide materials. With the current 
exchange correlation functionals and the available computational resources for the supercell calculations, the 
$T_{c}$ calculations is still a very challenging task. An interesting future research is the limitation of the 
Lorentzian approximation to calculate the lattice thermal conductivity in systems where the Lorentzian approximation
fails to describe the phonons. This is actually the case in monochalcogenide materials. It would be interesting, from 
a fundamental point of view, to understand what is the effect of not assuming the Lorentzian approximation when 
calculating the lattice thermal conductivity in SnSe and SnS. Another interesting future research project is the 
calculation of the lattice thermal conductivity in graphene. As we have seen, the inclusion of anharmonicity at 
a non-perturbative level, dramatically changes the line widths of acoustic modes. This will have a non-negligible 
effect in the lattice thermal conductivity of graphene and also in all the two dimensional materials.
