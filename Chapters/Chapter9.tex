% Chapter 1

\chapter*{Conclusions} % Main chapter title

\label{Chapter9} % For referencing the chapter elsewhere, use \ref{Chapter1} 
\addcontentsline{toc}{part}{Conclusions}

%----------------------------------------------------------------------------------------

The main goal of this thesis was to understand and predict the phase transitions that occur in high efficiency 
thermoelectric materials. Another purpose was to calculate the lattice thermal conductivity in these materials to 
understand the experimental results and find the origin of such a low thermal conductivity. \\

We have seen that both SnSe and SnS suffer second-order phase transitions from the low symmetry low temperature 
$Pnma$ phase to the high symmetry high temperature $Cmcm$ phase. This means that they suffer continuous displacive 
phase transitions. As it is expected in a second-order phase transition, the thermal conductivity is very similar 
for both phases close to the transition temperature. This together with the fact that the $Cmcm$ phase has a 
higher $PF$ due to a smaller band gap, makes these materials to have the maximum $ZT$ just after the phase 
transition. Therefore, the role of the phase transition is key for having a high thermoelectric efficiency. \\

The thermal conductivity of both materials is ultralow, around 0.3-1.0 W/mK at around $800$ K. We have seen that the 
main contribution to the lattice thermal conductivity comes from the acoustic modes. Their values are low due to the 
high anharmonicity of the material and low group velocities. Perturbation theory breaks down in the calculation of 
second-order and high-order force-constants and the inclusion of anharmonicity at a non-pertubative level is crucial 
for a proper description of the lattice thermal conductivity. We have seen that we only get results in agreement 
with experiments when we include anharmonic effects at a non-perturbative level. \\

From a qualitative point of view our transition temperature $T_{c}$ calculations are in agreement with experiments. 
We have seen that the quantitative agreement in $T_{c}$ is out of the scope of this thesis. This happens because the 
vibrational properties are very sensitive to the volume of the unit cell. Regarding the lattice thermal conductivity 
we get a very good agreement with experimental results and this has helped clarifying many experimental 
uncertainties. In the work by Zhao et al.\cite{zhao2014ultralow} they showed ultralow and isotropic values of the 
lattice thermal conductivity, lower than our calculations for the in-plane results. These results lead to 
$ZT_{max}=2.6$, the maximum $ZT$ so far, for an in-plane direction. Later 
experiments\cite{ibrahim2017reinvestigation,wei2019thermoelectric} have shown that the in-plane thermal conductivity 
is higher and are in good agreement with our results. It seems that the origin of the low in-plane lattice thermal 
conductivity of Zhao et al. is due to Sn vacancies. The later experimental works have shown that, due to the higher in-plane lattice thermal conductivity, the $ZT_{max}$ of intrinsic SnSe is around 1. Our calculations suggest that 
SnSe and SnS have very similar thermoelectric properties and, therefore, SnS may be a efficient thermoelectric 
material in the high temperature phase. This is also partially supported by experiments as both systems seem to have 
very similar $ZT$s in the low temperature phase close to the phase transition. Of course, this is not true for the 
results by Zhao et al. \\

Another important conclusion of this thesis is related to the experimental measurement of the phonon spectral 
function. We have seen that the Lorentzian approximation break down for some vibrational modes in SnSe and SnSe. 
This is in principle not an unexpected results as we have shown that they are strongly anharmonic materials. The 
break down of the Lorentzian approximations creates shoulders and double peak in the spectral function, which makes 
it more difficult to understand. These results are key for future inelastic scattering experiments in this kind of 
materials. \\
