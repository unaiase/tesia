% Chapter 1

\chapter{Monolayer SnSe} % Main chapter title

\label{Chapter7} % For referencing the chapter elsewhere, use \ref{Chapter1} 

%----------------------------------------------------------------------------------------

% Define some commands to keep the formatting separated from the content 
%\newcommand{\keyword}[1]{\textbf{#1}}
%\newcommand{\tabhead}[1]{\textbf{#1}}
%\newcommand{\code}[1]{\texttt{#1}}
%\newcommand{\file}[1]{\texttt{\bfseries#1}}
%\newcommand{\option}[1]{\texttt{\itshape#1}}

%----------------------------------------------------------------------------------------

\section{introduction}

Thin film ferroelectrics are the key for modern device applications\cite{lallart2011ferroelectrics}. However, it is 
known that there is a critical thickness for thin ferroelectrics, below which the depolarization field destroys the 
ferroelectric state\cite{batra1973new,zhong1994giant,dawber2005physics}. This effect decreases the transition 
temperature to the ferroelectric state as a function of the sample 
thickness\cite{fong2004ferroelectricity,fong2006stabilization} and sets a fundamental limit for technological  
applications. Layered Van der Waals materials have been proposed to be good candidates to overcome this 
limitation\cite{shirodkar2014emergence,fei2016ferroelectricity}. \\

Bulk monochalcogenides studied in chapters \ref{Chapter5} and \ref{Chapter6} are in this family of Van der Waals 
materials, which in principle, could be exfoliated to get monolayer materials. Actually, it has been experimetally 
shown that it is possible to synthesize monolayer SnSe\cite{li2013single,zhao2015controlled}. On top of this, it has 
been shown in monolayer SnTe, another monochalcogenide material, that the monolayer materials can show a robust 
ferroelectric state with a higher transition temperature than its bulk counterpart\cite{chang2016discovery}. These 
properties make very interesting the theoretical study of the ferroelectric transition in SnSe. \\

The structure and physical and thermoelectric properties of bulk SnSe have been explained in chapter \ref{Chapter5}. 
As can be seen in Fig. , monolayer SnSe has the same structure and space group but in the unit cell it only contains 
one bilayer with four atoms per unit cell (the bulk has two bilayers with 8 atoms). The electronic band gap is 
bigger in the monolayer\cite{wang2015thermoelectric,hu2017high}, due to quantum confinement effects, and the lattice 
thermal conductivity is of the same order of magnitude\cite{wang2015thermoelectric,hu2017high}.
