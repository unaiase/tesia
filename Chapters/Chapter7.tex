% Chapter 1

\chapter{Monolayer SnSe} % Main chapter title

\label{Chapter7} % For referencing the chapter elsewhere, use \ref{Chapter1} 

%----------------------------------------------------------------------------------------

% Define some commands to keep the formatting separated from the content 
%\newcommand{\keyword}[1]{\textbf{#1}}
%\newcommand{\tabhead}[1]{\textbf{#1}}
%\newcommand{\code}[1]{\texttt{#1}}
%\newcommand{\file}[1]{\texttt{\bfseries#1}}
%\newcommand{\option}[1]{\texttt{\itshape#1}}

%----------------------------------------------------------------------------------------

\section{introduction}

Thin film ferroelectrics are the key for modern device applications\cite{lallart2011ferroelectrics}. However, it is 
known that there is a critical thickness for thin ferroelectrics, below which the depolarization field destroys the 
ferroelectric state\cite{batra1973new,zhong1994giant,dawber2005physics}. This effect decreases the transition 
temperature to the ferroelectric state as a function of the sample 
thickness\cite{fong2004ferroelectricity,fong2006stabilization} and sets a fundamental limit for technological  
applications. Layered Van der Waals materials have been proposed to be good candidates to overcome this 
limitation\cite{shirodkar2014emergence,fei2016ferroelectricity}. \\

Bulk monochalcogenides studied in chapters \ref{Chapter5} and \ref{Chapter6} are in this family of Van der Waals 
materials, which in principle, could be exfoliated to get monolayer materials. Actually, it has been experimetally 
shown that it is possible to synthesize monolayer SnSe\cite{li2013single,zhao2015controlled}. On top of this, it has 
been shown in monolayer SnTe, another monochalcogenide material, that the monolayer materials can show a robust 
ferroelectric state with a higher transition temperature than its bulk counterpart\cite{chang2016discovery}. These 
properties make very interesting the theoretical study of the ferroelectric transition in SnSe. \\

The structure and physical and thermoelectric properties of bulk SnSe have been explained in chapter \ref{Chapter5}. 
As can be seen in Fig. \ref{pnma-cmcm-mono}, monolayer SnSe has the same structure of the bulk, but in the unit cell 
it only contains one bilayer with four atoms per unit cell (the bulk has two bilayers with 8 atoms, see 
Fig. \ref{pnma-cmcm}). 
\begin{figure}[h]
\begin{center}
\includegraphics[width=0.8\linewidth]{Figures/monolayer-structure.pdf}
\caption{Monolayer SnSe crystal structure in the a) $Pnmm$ (high symmetry) and b) $Pnm2_{1}$ (low symmetry) phases.}
\label{pnma-cmcm-mono}
\end{center}
\end{figure}
The electronic band gap is bigger in the monolayer\cite{wang2015thermoelectric,hu2017high}, due to quantum 
confinement effects, and the lattice thermal conductivity is of the same order of 
magnitude\cite{wang2015thermoelectric,hu2017high} compared to the bulk counterpart. According to several theoretical 
works, monolayer SnSe in the $Pnm2_{1}$ structure could have a higher thermoelectric figure of 
merit\cite{wang2015thermoelectric,hu2017high} than bulk $Pnma$. This makes very interesting the study of the 
thermoelectric properties of monolayer SnSe in the $Pnmm$ structure as in the bulk it is a better thermoelectric 
material at high temperatures than the $Pnma$. \\

Experimentally, there are no measurements for the ferroelectric phase transition temperature. However, monolayer 
SnSe seems to crystallize in the $Pnm2_{1}$ structure at room temperature\cite{li2013single}, therefore, the 
ferroelectric transition temperature must be higher. According to theoretical studies applying molecular dynamics 
simulations\cite{mehboudi2016structural,barraza2018tuning} the transition temperature is around $200$ K. Another 
work applying DFT and an effective Hamiltonian claims\cite{fei2016ferroelectricity} that the transition occurs at 
around $320$ K. Regarding the thermoelectric properties in monolayer SnSe in the $Cmcm$ structure, there are no 
theoretical calculations of the lattice thermal conductivity as it has an imaginary phonon frequency at the $\Gamma$ 
point within the harmonic approximation, which hinders the calculation of $\kappa_{l}$. \\

In this chapter we study the vibrational properties of $Pnmm$ monolayer SnSe and the associated transition to the 
ferroelectric state. According to our calculations monolayer SnSe suffers a second-order phase transition to the 
ferroelectric $Pnm2_{1}$ phase at around $200$ K. We also try to calculate the lattice thermal conductivity of 
$Pnm2_{1}$ monolayer SnSe, howerver, we will see that it is not possible due to some problems generated by the two 
dimensional character of the system. This last result will be an extra  motivation for the last chapter of this 
thesis.

\section{Crystal structure and high symmetry points}

The $Pnmm$ and $Pnm2_{1}$ phases are rectangular and their two dimensional structure is shown in 
Fig. \ref{pnma-cmcm-mono}. The primitive cell of both structures contains 4 atoms in the primitive cell. The 
primitive lattice vectors of both structures are: $\mathbf{a}_{1}=(a,0)$ and $\mathbf{a}_{2}=(0,b)$, where $a$ and 
$b$ are the lattice constants. \\

The reciprocal lattice vectors of the $Cmcm$ primitive 1BZ are $\mathbf{b}_{1}=2\pi(1/a,0)$ and 
$\mathbf{b}_{2}=2\pi(0,1/b)$. The high symmetry points and their coordinates used in phonon dispersion figures are 
listed in table \ref{qpoints-mono}.
\begin{table}
\begin{center}
\begin{tabular*}{0.45\textwidth}{l c}
 \hline
 \hline
 %\multicolumn{4}{|c|}{Country List} \\
             Symmetry point  & Reduced $\mathbf{q}$ vector  \\
 \hline
 $X$                  &  0.5, 0.0 \\
 $Y$                  &  0.0, 0.5 \\
 $R$                  &  0.5, 0.5  \\
 $\Gamma$             &  0.0, 0.0  \\
 \hline
 \hline
\end{tabular*}
\end{center}
\caption{Reduced $\mathbf{q}$ vectors of the high symmetry points in the Brillouin zone of the $Pnmm$ and $Pnm2_{1}$ 
phases.}
\label{qpoints-mono}
\end{table}

\section{Calculation details}

Our calculations are based on DFT using the QUANTUM-ESPRESSO\cite{giannozzi2009quantum} software package. Harmonic 
phonons were calculated within DFPT. Anharmonic phonons were calculated within the SSCHA. For the 
exchange-correlation interaction we use the Perdew-Burke-Ernzerhof (PBE) generalized gradient approximation and the 
local density approximation (LDA) with ultrasoft (US) and Projector Augmented Wave method (PAW) pseudopotentials 
respectively. We use a cutoff energy of 70 Ry and a grid of $16\times16$ $\boldsymbol{k}$ points to sample the first 
Brillouin zone. For the harmonic phonon calculations we use a $6\times6$ supercell for both harmonic and SSCHA 
calculations.

\section{Ferroelectric phase transition}

As in the bulk cases described in chapters \ref{Chapter5} and \ref{Chapter6}, 
symmetry\cite{orobengoa2009amplimodes,perez2010mode} dictates that it is possible to have a second-order phase 
transition between the $Pnmm$ and $Pnm2_{1}$ phases. In this case, The transition is dominated by the distortion 
pattern associated to a mode ($\Gamma_{1}$) at the zone center $\Gamma$ point. The distortion of this 
mode is shown in Fig. \ref{transition-mono} and it corresponds to the imaginary phonon in 
Fig. \ref{harmonic-mono}. \\
\begin{figure}[h]
\includegraphics[width=\linewidth]{Figures/transition-mono.pdf}
	\caption{Distortion pattern associated to the $\Gamma_{1}$ mode.}
\label{transition-mono}
\end{figure}

This mode is different in the bulk, where the distortion is associated to a mode at the 
zone border. In the monolayer the second bilayer does not exist, that is why the mode is at the zone center. This 
fact makes that the bulk $Pnma$ is centrosymmetryc but not the monolayer $Pnm2_{1}$, which has no inversion symmetry 
and it is a ferroelectric phase. \\

In a second-order displacive phase transition scenario, the transition temperature $T_{c}$ is defined as 
$\partial^{2}F/\partial Q^{2}(T=T_{c})=0$ where $Q$ is the order parameter that transforms the system continuously 
from the $Pnm2_{1}$ ($Q\ne0$, low symmetry) to the $Pnmm$ ($Q=0$, high symmetry) phase. As the distortion is 
dominated by the $\Gamma_{1}$ phonon, $\partial^{2}F/\partial Q^{2}(T)$ is proportional to 
$\Omega^{(F)2}_{\Gamma_{1}}(T)$, which can be calculated using Eq. \ref{free-energy-hessian}. \\

In the bulk case, in both LDA and PBE approximations the $Y_{1}$ is unstable within the harmonic approximation, 
which is the typical scenario for a second-order phase transition. In Fig. \ref{harmonic-mono} we show the 
harmonic phonon of monolayer SnSe in the $Pnmm$ structure within the LDA and PBe approximations. 
\begin{figure}[h]
\includegraphics[width=\linewidth]{Figures/harmonic-mono.eps}
\caption[Harmonic phonons of monolayer SnSe.]{Harmonic phonons of monolayer SnSe in the $Pnmm$ phase within the LDA 
and PBE approximations in the theoretical structure.}
\label{harmonic-mono}
\end{figure}
As we can see, the instability $\Gamma_{1}$ appears within PBE but not within LDA. This results again shows the huge 
pseudopotential and volume dependence of the vibrational properties of SnSe, which was also discussed in 
chapter \ref{Chapter5}. According to this result, the second-order phase transition scenario only appears within PBE 
in the case of the monolayer, from now on we will only show results within PBE. \\

Figure \ref{freq-transition-mono} shows $\Omega^{(F)2}_{Y_{1}}(T)$ within the PBE approximation.
\begin{figure}[h]
\includegraphics[width=\linewidth]{Figures/freq-mono.eps}
\caption[Phonon collapse in monolayer SnSe.]{$\Omega^{(F)2}_{\Gamma_{1}}$ as a function of temperature within PBE 
using the theoretical lattice parameters.}
\label{freq-transition-mono}
\end{figure}
As in the cases of bulk SnSe and SnS, the second derivative of the free energy is positive at high temperatures and 
decreases lowering the temperature. For both approximations, it becomes negative at the critical temperature $T_c$, 
which means that the $Pnmm$ phase is not any longer a minimum of the free energy and the structure distorts adopting 
the $Pnm2_{1}$ phase. Compare here with the other theoretical results.

\section{Problem for calculating lattice thermal conductivity}

As shown in chapters \ref{Chapter3}, \ref{Chapter5}, and \ref{Chapter6} the SSCHA phonons ($\Omega^{(S)}_{\mu}$) are 
a properly defined basis for the three phonon scattering phase space and provide an accurate lattice thermal 
conductivity. In Fig. \ref{sscha-freq-mono} we show the SSCHA phonons of monolayer SnSe in the $Pnmm$ structure at 
$300$ K, which is a temperature where the high symmetry phase is stable (see Fig. \ref{freq-transition-mono}).
\begin{figure}[h]
\includegraphics[width=\linewidth]{Figures/sscha-mono.eps}
\caption[Anharmonic phonons in monolayer SnSe.]{$\Omega^{(S)}_{\mu}(\boldsymbol{q})$ phonon spectrum at $300$ K 
within PBE using the theoretical lattice parameters.}
\label{sscha-freq-mono}
\end{figure}
As we can see
