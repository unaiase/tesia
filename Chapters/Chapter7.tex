% Chapter 1

\chapter{Monolayer SnSe} % Main chapter title

\label{Chapter7} % For referencing the chapter elsewhere, use \ref{Chapter1} 

%----------------------------------------------------------------------------------------

% Define some commands to keep the formatting separated from the content 
%\newcommand{\keyword}[1]{\textbf{#1}}
%\newcommand{\tabhead}[1]{\textbf{#1}}
%\newcommand{\code}[1]{\texttt{#1}}
%\newcommand{\file}[1]{\texttt{\bfseries#1}}
%\newcommand{\option}[1]{\texttt{\itshape#1}}

%----------------------------------------------------------------------------------------

\section{Introduction}

Thin film ferroelectrics are key for modern device applications\cite{lallart2011ferroelectrics}. However, it is 
known that there is a critical thickness for thin ferroelectrics, below which the depolarization field destroys the 
ferroelectric state\cite{batra1973new,zhong1994giant,dawber2005physics}. This effect decreases the transition 
temperature to the ferroelectric state as a function of the sample 
thickness\cite{fong2004ferroelectricity,fong2006stabilization} and sets a fundamental limit for technological  
applications. \\

Layered Van der Waals materials have been proposed to be good candidates to overcome this 
limitation\cite{shirodkar2014emergence,fei2016ferroelectricity}. 
Bulk monochalcogenides studied in chapters \ref{Chapter5} and \ref{Chapter6} are inside the family of Van der Waals 
materials, which in principle, could be exfoliated to get monolayer materials. Actually, it has been experimentally 
shown that it is possible to synthesize monolayer SnSe\cite{li2013single,zhao2015controlled}. On top of this, it has 
been shown in monolayer SnTe, another monochalcogenide material, that the monolayer materials can show a robust 
ferroelectric state with a higher transition temperature than its bulk counterpart\cite{chang2016discovery}. These 
properties make very interesting the theoretical study of the ferroelectric transition in SnSe. Actually, in a very 
recent experimental work\cite{chang2020controlled}, they show how the polarization of monolayer SnSe can be switched 
at room temperature. \\

The thermoelectric properties of bulk SnSe have been explained in chapter \ref{Chapter5}. As can be seen in 
Fig. \ref{pnma-cmcm-mono}, monolayer SnSe has the same structure of the bulk, but in the unit cell 
it only contains one bilayer with four atoms in the unit cell (the bulk has two bilayers with 8 atoms, see 
Fig. \ref{pnma-cmcm}). 
\begin{figure}[h]
\begin{center}
\includegraphics[width=0.8\linewidth]{Figures/monolayer-structure.pdf}
	\caption[Monolayer SnSe crystal structure]{Monolayer SnSe crystal structure in the (a) $Pnmm$ (high symmetry) 
	and (b) $Pnm2_{1}$ (low symmetry) phases.}
\label{pnma-cmcm-mono}
\end{center}
\end{figure}
The electronic band gap is bigger in the monolayer\cite{wang2015thermoelectric,hu2017high}, due to quantum 
confinement effects. According to several theoretical studies\cite{wang2015thermoelectric,hu2017high} in the 
monolayer low symmetry phase, the lattice thermal conductivity is of the same order of magnitude compared to the 
bulk counterpart. These theoretical works show that, monolayer SnSe in the $Pnm2_{1}$ (low symmetry) structure could 
have a higher thermoelectric figure of merit\cite{wang2015thermoelectric,hu2017high} than bulk $Pnma$ (low symmetry). This makes very interesting the study of the thermoelectric properties of monolayer SnSe in the $Pnmm$ (high 
symmetry) structure as in the bulk the high symmetry $Cmcm$ phase is a better thermoelectric material at high 
temperatures than the $Pnma$. \\

Experimentally, there are no measurements for the ferroelectric phase transition temperature. However, monolayer 
SnSe seems to crystallize in the $Pnm2_{1}$ structure at room temperature\cite{li2013single,chang2020controlled}, 
therefore, the ferroelectric transition temperature must be higher. According to theoretical studies applying 
molecular dynamics simulations\cite{mehboudi2016structural,barraza2018tuning} the transition temperature is around 
$200$ K. Another work applying DFT and an effective Hamiltonian claims\cite{fei2016ferroelectricity} that the 
transition occurs at around $320$ K. Regarding the thermoelectric properties in monolayer SnSe in the $Pnmm$ 
structure, there are no theoretical calculations of the lattice thermal conductivity as it has an imaginary phonon 
frequency at the $\Gamma$ point within the harmonic approximation (see Fig. \ref{harmonic-mono}), which hinders the 
calculation of $\kappa_{l}$. \\

In this chapter we study the vibrational properties of $Pnmm$ monolayer SnSe and the associated transition to the 
ferroelectric state. According to our calculations monolayer SnSe suffers a second-order phase transition to the 
ferroelectric $Pnm2_{1}$ phase at around $107$ K. We also try to calculate the lattice thermal conductivity of 
$Pnm2_{1}$ monolayer SnSe in order to compare it with the bulk, however, we will see that it is not possible due to 
a problem generated by the two dimensional character of the system. This last result will be an extra motivation for 
the last chapter of this thesis.

\section{Crystal structure and high symmetry points}

The $Pnmm$ and $Pnm2_{1}$ phases are rectangular and their two dimensional structure is shown in 
Fig. \ref{pnma-cmcm-mono}. The primitive cell of both structures contains 4 atoms in the primitive cell. The 
primitive lattice vectors of both structures are: $\mathbf{a}_{1}=(a,0)$ and $\mathbf{a}_{2}=(0,b)$, where $a$ and 
$b$ are the lattice constants. \\

The reciprocal lattice vectors of the $Cmcm$ primitive 1BZ are $\mathbf{b}_{1}=2\pi(1/a,0)$ and 
$\mathbf{b}_{2}=2\pi(0,1/b)$. The high symmetry points and their coordinates used in phonon dispersion figures are 
listed in table \ref{qpoints-mono}.
\begin{table}
\begin{center}
\begin{tabular*}{0.45\textwidth}{l c}
 \hline
 \hline
 %\multicolumn{4}{|c|}{Country List} \\
             Symmetry point  & Reduced $\mathbf{q}$ vector  \\
 \hline
 $X$                  &  0.5, 0.0 \\
 $Y$                  &  0.0, 0.5 \\
 $R$                  &  0.5, 0.5  \\
 $\Gamma$             &  0.0, 0.0  \\
 \hline
 \hline
\end{tabular*}
\end{center}
\caption{Reduced $\mathbf{q}$ vectors of the high symmetry points in the Brillouin zone of the $Pnmm$ and $Pnm2_{1}$ 
phases.}
\label{qpoints-mono}
\end{table}
The theoretical lattice parameters within LDA are $a=b=7.94$ $a_{0}$ and within PBE $a=b=8.17$ $a_{0}$. 

\section{Calculation details}

Our calculations are based on DFT using the QUANTUM-ESPRESSO\cite{giannozzi2009quantum} software package. Harmonic 
phonons were calculated within DFPT. Anharmonic phonons were calculated within the SSCHA. For the 
exchange-correlation interaction we use the Perdew-Burke-Ernzerhof (PBE) generalized gradient approximation and the 
local density approximation (LDA) with ultrasoft (US) and Projector Augmented Wave method (PAW) pseudopotentials 
respectively. We use a cutoff energy of 70 Ry and a grid of $16\times16$ $\boldsymbol{k}$ points to sample the first 
Brillouin zone. For the harmonic phonon calculations we use a $6\times6$ supercell for both harmonic and SSCHA 
calculations.

\section{Ferroelectric phase transition}

As in the bulk cases described in chapters \ref{Chapter5} and \ref{Chapter6}, 
symmetry\cite{orobengoa2009amplimodes,perez2010mode} dictates that it is possible to have a second-order phase 
transition between the $Pnmm$ and $Pnm2_{1}$ phases. In this case, The transition is dominated by the distortion 
pattern associated to a mode ($\Gamma_{1}$) at the zone center. The distortion of this 
mode is shown in Fig. \ref{transition-mono} and it corresponds to the imaginary phonon in 
Fig. \ref{harmonic-mono}. \\
\begin{figure}[h]
\includegraphics[width=\linewidth]{Figures/transition-mono.pdf}
	\caption{Distortion pattern associated to the $\Gamma_{1}$ mode.}
\label{transition-mono}
\end{figure}

The phonon mode that transforms bulk SnSe from the $Cmcm$ to $Pnma$ structure is different. The distortion is 
associated to a mode at the zone border. In the monolayer the second bilayer does not exist, that is why the mode is 
at the zone center. This fact makes that the bulk $Pnma$ is centrosymmetric but not the monolayer $Pnm2_{1}$, which 
has no inversion symmetry and it is a ferroelectric phase. \\

In a second-order displacive phase transition scenario, the transition temperature $T_{c}$ is defined as 
$\partial^{2}F/\partial Q^{2}(T=T_{c})=0$ where $Q$ is the order parameter that transforms the system continuously 
from the $Pnm2_{1}$ ($Q\ne0$, low symmetry) to the $Pnmm$ ($Q=0$, high symmetry) phase. As the distortion is 
dominated by the $\Gamma_{1}$ phonon, $\partial^{2}F/\partial Q^{2}(T)$ is proportional to 
$\Omega^{(F)2}_{\Gamma_{1}}(T)$, which can be calculated using Eq. \ref{free-energy-hessian}. \\

In the bulk case, in both LDA and PBE approximations the $Y_{1}$ is unstable within the harmonic approximation, 
which is the typical scenario for a second-order phase transition. In Fig. \ref{harmonic-mono} we show the 
harmonic phonon spectrum of monolayer SnSe in the $Pnmm$ structure within the LDA and PBE approximations. 
\begin{figure}[h]
\includegraphics[width=\linewidth]{Figures/harmonic-mono.eps}
\caption[Harmonic phonons of monolayer SnSe.]{Harmonic phonons of monolayer SnSe in the $Pnmm$ phase within the LDA 
and PBE approximations in the theoretical structure.}
\label{harmonic-mono}
\end{figure}
As we can see, the instability $\Gamma_{1}$ appears within PBE but not within LDA. This results again shows the huge 
volume dependence of the vibrational properties of SnSe, which was also discussed in chapter \ref{Chapter5}. 
According to this result, the second-order phase transition scenario only appears within PBE in the case of the 
monolayer. From now on we will only show results within PBE. \\

Fig. \ref{freq-transition-mono} shows $\Omega^{(F)2}_{Y_{1}}(T)$ within the PBE approximation.
\begin{figure}[h]
\includegraphics[width=\linewidth]{Figures/freq-mono.eps}
\caption[Phonon collapse in monolayer SnSe.]{$\Omega^{(F)2}_{\Gamma_{1}}$ as a function of temperature within PBE 
using the theoretical lattice parameters. We show the calculation for the $2\times2$, $4\times4$, and $6\times6$ 
supercells.}
\label{freq-transition-mono}
\end{figure}
As in the cases of bulk SnSe and SnS, the second derivative of the free energy is positive at high temperatures and 
decreases lowering the temperature. It becomes negative at the critical temperature 
$T_{c}=107$ K in the $6\times6$ supercell, which means that the $Pnmm$ phase is not any longer a minimum of the free 
energy and the structure distorts adopting the $Pnm2_{1}$ phase. The transition temperature is around 50 K in the 
$4\times4$ supercell and the transition does not occur in the $2\times2$ supercell. This supercell dependence 
suggest that the calculation may not be fully converged. It could also explain why we do not get a better 
agreement for the transition temperature in the bulk case. \\

Our results are consistent with other theoretical 
works\cite{mehboudi2016structural,barraza2018tuning,fei2016ferroelectricity}, where the same ferroelectric 
transition is predicted. From a quantitative point of view, our transition temperature of $107$ K is lower than the 
$\simeq180-200$ K value obtained in molecular dynamics simulations\cite{mehboudi2016structural,barraza2018tuning}. 
As already mentioned, the discrepancy probably arises due to the different exchange-correlation or the smaller 
supercell we have used to calculate the transition temperature.

\section{Problem for calculating lattice thermal conductivity}

As shown in chapters \ref{Chapter3}, \ref{Chapter5}, and \ref{Chapter6} the SSCHA phonons ($\Omega^{(S)}_{\mu}$) are 
a properly defined basis for the three phonon scattering phase space and provide an accurate lattice thermal 
conductivity. In Fig. \ref{sscha-freq-mono} we show the SSCHA phonons of monolayer SnSe in the $Pnmm$ structure at 
$300$ K, which is a temperature where the high symmetry phase is stable (see Fig. \ref{freq-transition-mono}).
\begin{figure}[h]
\includegraphics[width=\linewidth]{Figures/sscha-mono.eps}
\caption[Anharmonic phonons in monolayer SnSe.]{$\Omega^{(S)}_{\mu}(\boldsymbol{q})$ phonon spectrum at $300$ K 
within PBE using the theoretical lattice parameters.}
\label{sscha-freq-mono}
\end{figure}
As we can see there is a small instability in the lowest energy acoustic branch very close to the point $\Gamma$. 
This small instability could be an artifact of the Fourier interpolation, which is needed to get information about 
the acoustic branches at very small momenta. It could also be that the lattice parameter is too small once the 
fluctuations of the ions are taken into account. In principle, if we would be able to do a calculation in an 
infinitely big supercell and relax the lattice parameters including the ionic fluctuations, the problem would 
disappear, obviously, this is not possible. However, we could ask ourselves why this problem appears in the monolayer 
and not in the bulk. Actually, there is a very fundamental reason for this. As it 
is shown in appendix \ref{quadratic-dispersion}, the lowest energy acoustic branch in monolayer materials has a 
quadratic dispersion as a function of momentum within the harmonic approximation, which provides it with very low 
frequencies compared to the other two linear acoustic branches. The linear dispersion is not kept in the SSCHA phonon
dispersion, however, still the frequency of the lowest energy acoustic mode is substantially lower. Because of this 
reason we are not able to calculate the lattice thermal conductivity of monolayer SnSe as we need the phonon 
frequencies at small momenta in order to calculate the phonon line widths. In chapter \ref{Chapter8} we include a 
deep analysis about the anharmonic effects on the lowest energy acoustic branch of monolayer materials.

\section{Conclusions}

In conclusion, we have shown that a ferroelectric transition is possible from the high symmetry (centrosymmetric) 
$Pnmm$ phase to the low symmetry (no centrosymmetric) $Pnm2_{1}$ phase. We have seen that this phase transition 
is different to the one that appears in the bulk in which the two phases are centrosymmetric. We have seen that 
this phase transition actually happens at around $107$ K, result that is in qualitative agreement with previous 
theoretical calculations. Quantitative disagreement may be due to exchange correlation or supercell effects, which 
are very big in these kind of materials. Finally, we have tried to calculate the lattice thermal conductivity of 
$Pnmm$ monolayer SnSe but we have found problems related to the 2D character of the material, which make the lowest 
energy acoustic branch very soft. We will see in the next chapter what is the effect of anharmonic effects in this 
mode and its role on the mechanical stability of 2D materials.
