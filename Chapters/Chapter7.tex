% Chapter 1

\chapter{Monolayer SnSe} % Main chapter title

\label{Chapter7} % For referencing the chapter elsewhere, use \ref{Chapter1} 

%----------------------------------------------------------------------------------------

% Define some commands to keep the formatting separated from the content 
%\newcommand{\keyword}[1]{\textbf{#1}}
%\newcommand{\tabhead}[1]{\textbf{#1}}
%\newcommand{\code}[1]{\texttt{#1}}
%\newcommand{\file}[1]{\texttt{\bfseries#1}}
%\newcommand{\option}[1]{\texttt{\itshape#1}}

%----------------------------------------------------------------------------------------

\section{introduction}

Thin film ferroelectrics are the key for modern device applications\cite{lallart2011ferroelectrics}. However, it is 
known that there is a critical thickness for thin ferroelectrics, below which the depolarization field destroys the 
ferroelectric state\cite{batra1973new,zhong1994giant,dawber2005physics}. This effect decreases the transition 
temperature to the ferroelectric state as a function of the sample 
thickness\cite{fong2004ferroelectricity,fong2006stabilization} and sets a fundamental limit for technological  
applications. Layered Van der Waals materials have been proposed to be good candidates to overcome this 
limitation\cite{shirodkar2014emergence,fei2016ferroelectricity}. \\

Bulk monochalcogenides studied in chapters \ref{Chapter5} and \ref{Chapter6} are in this family of Van der Waals 
materials, which in principle, could be exfoliated to get monolayer materials. Actually, it has been experimetally 
shown that it is possible to synthesize monolayer SnSe\cite{li2013single,zhao2015controlled}. On top of this, it has 
been shown in monolayer SnTe, another monochalcogenide material, that the monolayer materials can show a robust 
ferroelectric state with a higher transition temperature than its bulk counterpart\cite{chang2016discovery}. These 
properties make very interesting the theoretical study of the ferroelectric transition in SnSe. \\

The structure and physical and thermoelectric properties of bulk SnSe have been explained in chapter \ref{Chapter5}. 
As can be seen in Fig. \ref{pnma-cmcm-mono}, monolayer SnSe has the same structure and space group of the bulk, but 
in the unit cell it only contains one bilayer with four atoms per unit cell (the bulk has two bilayers with 8 atoms). 
\begin{figure}[h]
\begin{center}
\includegraphics[width=0.8\linewidth]{Figures/monolayer-structure.pdf}
\caption{Monolayer SnSe crystal structure in the a) $Cmcm$ and b) $Pnma$ phases.}
\label{pnma-cmcm-mono}
\end{center}
\end{figure}
The electronic band gap is bigger in the monolayer\cite{wang2015thermoelectric,hu2017high}, due to quantum 
confinement effects, and the lattice thermal conductivity is of the same order of 
magnitude\cite{wang2015thermoelectric,hu2017high} compared to the bulk counterpart. According to several theoretical 
works, monolayer SnSe in the $Pnma$ structure could have a higher thermoelectric figure of 
merit\cite{wang2015thermoelectric,hu2017high} than bulk $Pnma$. This makes very interesting the study of the 
thermoelectric properties of monolayer SnSe in the $Cmcm$ structure as in the bulk it is a better thermoelectric 
material than the $Pnma$ phase at high temperatures. \\

Experimentally, there are no measurements for the ferroelectric phase transition temperature. However, monolayer 
SnSe seems to crystallize in the $Pnma$ structure at room temperature\cite{li2013single}, therefore, the 
ferroelectric transition temperature must be higher. According to theoretical studies applying molecular dynamics 
simulations\cite{mehboudi2016structural,barraza2018tuning} the transition temperature is around $200$ K. Another 
work applying DFT and an effective Hamiltonian claims\cite{fei2016ferroelectricity} that the transition occurs at 
around $320$ K. Regarding the thermoelectric properties in monolayer SnSe in the $Cmcm$ structure, there are no 
theoretical calculations of the lattice thermal conductivity as it has an imaginary phonon frequency at the $\Gamma$ 
point, which hinders the calculation of $\kappa_{l}$. \\

In this chapter we study the vibrational properties of $Cmcm$ monolayer SnSe and the associated transition to the 
ferroelectric state. According to our calculations monolayer SnSe suffers a second-order phase transition to the 
ferroelectric $Pnma$ phase at around $200$ K. We also try to calculate the lattice thermal conductivity of $Cmcm$ 
monolayer SnSe, howerver, we will see that it is not possible due to the two dimensional character of the system. 
This last result will be an extra  motivation for the last chapter of this thesis.

\section{Crystal structure and high symmetry points}

The $Cmcm$ and $Pnma$ phases are orthorhombic and their two dimensional structure is shown in 
Fig. \ref{pnma-cmcm-mono}. The primitive cell of both structures contains 4 atoms in the primitive cell. The 
primitive lattice vectors of both structures are: $\mathbf{a}_{1}=(a,0)$ and $\mathbf{a}_{2}=(0,b)$, where $a$ and 
$b$ are the lattice constants. \\

The reciprocal lattice vectors of the $Cmcm$ primitive 1BZ are $\mathbf{b}_{1}=2\pi(1/a,0)$ and 
$\mathbf{b}_{2}=2\pi(0,1/b)$. The high symmetry points and their coordinates used in phonon dispersion figures are 
listed in table \ref{qpoints-mono}.
\begin{table}
\begin{center}
\begin{tabular*}{0.45\textwidth}{l c}
 \hline
 \hline
 %\multicolumn{4}{|c|}{Country List} \\
             Symmetry point  & Reduced $\mathbf{q}$ vector  \\
 \hline
 $X$                  &  0.5, 0.0 \\
 $Y$                  &  0.0, 0.5 \\
 $R$                  &  0.5, 0.5  \\
 $\Gamma$             &  0.0, 0.0  \\
 \hline
 \hline
\end{tabular*}
\end{center}
\caption{Reduced $\mathbf{q}$ vectors of the high symmetry points in the Brillouin zone of the $Cmcm$ and $Pnma$ 
phases.}
\label{qpoints-mono}
\end{table}

\section{Calculation details}

Our calculations are based on DFT using the QUANTUM-ESPRESSO\cite{giannozzi2009quantum} software package. Harmonic 
phonons were calculated within DFPT. Anharmonic phonons were calculated within the SSCHA. For the 
exchange-correlation interaction we use the Perdew-Burke-Ernzerhof (PBE) generalized gradient approximation and the 
local density approximation (LDA) with ultrasoft (US) and Projector Augmented Wave method (PAW) pseudopotentials 
respectively. We use a cutoff energy of 70 Ry and a grid of $16\times16$  $\boldsymbol{k}$ points to sample the first 
Brillouin zone. For the harmonic phonon calculations we use a $6\times6$ supercell for both harmonic and SSCHA 
calculation.

\section{Ferroelectric phase transition}

As it has been already pointed out in chapters \ref{Chapter5} and \ref{Chapter6}, 
symmetry\cite{orobengoa2009amplimodes,perez2010mode} dictates that it is possible to have a second-order phase 
transition between the $Cmcm$ and $Pnma$ phases. In the case of the monolayer, The transition is dominated by the 
distortion pattern associated to a mode ($\Gamma_{1}$) at the zone center $\Gamma$ point. The distortion of this 
mode is shown in Fig. . 
Figure \ref{transition-mono} shows $\Omega^{(F)2}_{Y_{1}}(T)$ within the LDA and PBE approximations.
\begin{figure}[h]
\includegraphics[width=\linewidth]{Figures/transition-mono.pdf}
	\caption{Distortion pattern associated to the $\Gamma_{1}$ mode.}
\label{transition-mono}
\end{figure}
This mode is different in the bulk, where the distortion is associated to a mode at the 
zone border. In the monolayer the second bilayer does not exist, that is why the mode is at the zone center. This 
fact makes that the bulk $Pnma$ is centrosymmetryc but not the monolayer $Pnma$, which has no inversion symmetry 
and it is a ferroelectric phase. \\

In a second-order displacive phase transition scenario, the transition temperature $T_{c}$ is defined as 
$\partial^{2}F/\partial Q^{2}(T=T_{c})=0$ where $Q$ is the order parameter that transforms the system continuously 
from the $Pnma$ ($Q\ne0$) to the $Cmcm$ ($Q=0$) phase. As the distortion is dominated by the $\Gamma_{1}$ phonon, 
$\partial^{2}F/\partial Q^{2}(T)$ is proportional to $\Omega^{(F)2}_{\Gamma_{1}}(T)$, which can be calculated using 
Eq. \ref{free-energy-hessian}. \\

Figure \ref{freq-transition-mono} shows $\Omega^{(F)2}_{Y_{1}}(T)$ within the LDA and PBE approximations.
\begin{figure}[h]
\includegraphics[width=\linewidth]{Figures/freq-sns.eps}
        \caption[Phonon collapse in SnS.]{$\Omega^{(F)2}_{Y_{1}}$ as a function of temperature within LDA and PBE approximations using the experimental lattice parameters (circles). The solid lines correspond to a polynomial fit. We include the pressure component
$P_{zz}$, which is the pressure in the direction where the atoms move in the transition.  This pressure is calculated including the anharmonic vibrational energy within the SSCHA.}
\label{freq-transition-mono}
\end{figure}
As in the case of SnSe\cite{aseginolaza2019phonon}, the second derivative of the free energy is positive at high temperatures and decreases lowering the temperature. For both approximations, it becomes negative at the critical
temperature $T_c$, which means that the $Cmcm$ phase is not any longer a minimum of the free energy and the structure distorts adopting the $Pnma$ phase. $T_{c}$ strongly depends on the approximation of the exchange-correlation
functional: it is $600$ K for LDA and $465$ K for PBE. Our LDA calculation agrees better with the experimental value, around $900$ K\cite{chattopadhyay1986neutron}. We associate the discrepancy between LDA and PBE  to the different
pressures obtained in the transition direction, $P_{zz}$. In fact, as shown in the case of SnSe\cite{aseginolaza2019phonon}, $T_{c}$ depends strongly on the pressure in this $z$ direction. The pressure in
Figure \ref{transition} includes anharmonic vibrational effects on the energy following the procedure outlined in section \ref{scha-stress-section}. For the same lattice parameter LDA displays a much smaller pressure, as
generally LDA predicts smaller lattice volumes than PBE. The underestimation with respect to experiments may be attributed to the small supercell size used for the SSCHA calculations ($2\times2\times2$). Even if
experimentally $T_{c}$ is around $100$ K higher in SnS than in SnSe, our LDA calculations give basically the same transition temperature for both materials as $T_{c}=616$ K in SnSe according to our previous
calculations\cite{aseginolaza2019phonon}. However, within PBE SnSe does show a lower transition temperature since $T_c=299$ K for SnSe\cite{aseginolaza2019phonon}.
