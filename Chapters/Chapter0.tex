% Chapter 1

\chapter*{Introduction} % Main chapter title

\label{Chapter0} % For referencing the chapter elsewhere, use \ref{Chapter1} 
\addcontentsline{toc}{chapter}{Introduction}

%----------------------------------------------------------------------------------------

Thermoelectricity is a material property with an endless application list. It can be used to generate electricity 
from a temperature gradient, which can be waste heat coming from a combustion engine. It can be also applied in the 
reverse way to transform a voltage into a temperature gradient. This application in mainly exploited in the 
microchip cooling technology. A research line in the thermoelectric community is to find materials with a higher 
thermoelectric efficiency. Another research line tries to find and understand physical and chemical mechanisms that 
increase the thermoelectric efficiency. Even though these research lines have a long history, the main challenge is 
to find materials or mechanisms to apply these materials in an industrial way. The state of the art materials have 
very low efficiencies, which make them only useful in very specific applications. \\

The efficiency of a thermoelectric material is measured by the figure of merit $ZT=S^{2}\sigma T/\kappa$. $S$ is the
Seebeck coefficient, $\sigma$ the electrical conductivity and $\kappa$ the thermal conductivity. With a $ZT$ of 3 
in a broad temperature range, the thermoeletric materials would be competetitive against the heat engines in terms of
heat transformation into electricity. In Fig. \ref{ztvst} we show the $ZT$ of record materials as a function of the 
year when they were discovered or synthesized.
\begin{figure}[h]
\begin{center}
\includegraphics[width=0.9\linewidth]{Figures/ztvstemp.png}
\caption[Record $ZT$ materials]{Record $ZT$ materials as a function of the year.}
\label{ztvst}
\end{center}
\end{figure}
As we can see, there is not material with a $ZT$ of three. Even though there are some materials with a $ZT$ of 2 or 
higher, usually this only happens for very narrow temperature ranges. This limits the applications of these 
materials. \\

The main problem behind the increase of the $ZT$ is the fact that the physical magnitudes that are involved are 
correlated. This correlation goes against high $ZT$s. In Fig. \ref{decouple} we show a cartoon of the physical 
magnitudes as a function of the doping for a given temperature.
\begin{figure}[h]
\begin{center}
\includegraphics[width=0.9\linewidth]{Figures/decoupling.png}
\caption[Thermoelectric physical magnitudes]{Electrical conductivity and  Seebeck coefficient as a function of the 
carrier concentration in a typical semiconductor.}
\label{decouple}
\end{center}
\end{figure}
The power factor $PF$ is the product of the squared Seebeck coefficient and the electrical conductivity. The idea 
behind a good thermoelectric material is very simple. We need materials with a high power factor and low thermal 
conductivity, the so called electron crystals and phonon glasses. However, if we have a look to Fig. \ref{decouple}
we can see that the power factor has a clear limitation. The electrical conductivity increases as a function of the 
carrier concentration, however, the Seebeck coefficient has the opposite trend, which makes the PF to have a peak 
that cannot be increased. On top of this, the thermal conductivity of typical semiconductors increases as a 
function of the carrier concentration. This further decreases the carrier concentration at which the maximum $ZT$ 
arises. \\

In the last years, it has been shown that intrinsic semiconductors with an intrinsically low thermal conductivity 
are materials with a very high $ZT$. The total thermal conductivity is the sum of electronic and lattice thermal 
conductivities. In intrinsics semiconductor, the main contribution to the thermal conductivity comes from the 
lattice. Therefore, the challenge is to find intrinsic semiconductors with an instrinsically low lattice thermal 
conductivity. The best thermoelectric material in this family is SnSe with a $ZT_{max}$ of 2.6. This material is 
not only the best thermoelectric in this family, but the best thermoelectric material overall. \\

At this point, theoretical material science plays a crucial role. Theory could reduce the experimental effort by 
searching for materials with an intrinsically low lattice thermal conductivity. The calculation of the lattice 
thermal conductivity is a complicated task and it is computationally demanding. Anyway, nowadays it is included in 
the toolkit of any theoretical material scientist. However, it has been found, that the intrinsically low lattice 
thermal conductivity is associated, in many cases, to a strongly anharmonic lattice. This is the case, for instance, 
in SnSe. The strong anharmonicity not only causes low thermal conductivity, it also creates interesting properties,
such as, structural  phase transition. These structural phase transition, in many cases, make the thermoelectric 
materials more efficient. However, The structural phase transitions make materials unstable within the harmonic 
approximation, on top of which, all the theory for calculating the lattice thermal conductivity is built by using 
perturbation theory.

EXPLAIN WHAT WE CAN DO HERE. \\

EXPLAIN THAT THESE FRAMEWORK IS ALSO VERY INTERESTING IN 2D MATERIALS. WHICH CAN BE ALSO GOOD THERMOELECTRICS. ALSO 
INTERESTING IN THE MICROCHIP TECHNOLOGY.
