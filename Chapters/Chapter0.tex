% Chapter 1

\chapter*{Introduction} % Main chapter title

\label{Chapter0} % For referencing the chapter elsewhere, use \ref{Chapter1} 
\addcontentsline{toc}{chapter}{Introduction}

%----------------------------------------------------------------------------------------

Thermoelectricity is a material property with an endless application list. It can be used to generate electricity 
from a temperature gradient, which can be waste heat coming from a combustion engine. It can be also applied in the 
reverse way to transform a voltage into a temperature gradient. This application in mainly exploited in the 
microchip cooling technology. A research line in the thermoelectric community is to find materials with a higher 
thermoelectric efficiency. Another research line tries to find and understand physical and chemical mechanisms that 
increase the thermoelectric efficiency. Even though these research lines have a long history, the main challenge is 
to find materials or mechanisms to apply these materials in an industrial way. The state of the art materials have 
very low efficiencies, which make them only useful in very specific applications. \\

What is the problem? ELECTRICITY/HEAT. \\

EXPLAIN CURRENT TREND TO DECREASE THERMAL CONDUCTIVITY. \\

EXPLAIN WHAT ARE THE MAIN CHALLENGES IN INTERESTING THERMOELECTRIC CLOSE TO PHASE TRANSITIONS. \\

EXPLAIN WHAT WE CAN DO HERE. \\

EXPLAIN THAT THESE FRAMEWORK IS ALSO VERY INTERESTING IN 2D MATERIALS. WHICH CAN BE ALSO GOOD THERMOELECTRICS. ALSO 
INTERESTING IN THE MICROCHIP TECHNOLOGY.
