% Chapter 1

\chapter*{Introduction} % Main chapter title

\label{Chapter0} % For referencing the chapter elsewhere, use \ref{Chapter1} 
\addcontentsline{toc}{chapter}{Introduction}

%----------------------------------------------------------------------------------------

Thermoelectricity is a material property with an endless application list. It can be used to generate electricity 
from a temperature gradient\cite{li2009thermoelectric,yang2006thermoelectric}, which can potentially be the be waste heat coming from a combustion engine. It can be 
also applied in the reverse way to transform a voltage into a temperature gradient. This application is mainly 
exploited in the microchip cooling technology\cite{disalvo1999thermoelectric,zhao2014review}. The two fundamental 
research lines in the thermoelectric community are the following: 
\begin{enumerate}
\item Find and synthesize materials with a higher thermoelectric efficiency.
\item Find and understand physical and chemical mechanisms that increase the thermoelectric efficiency. 
\end{enumerate}	
Even though these research lines have a long history, the main challenge is to find materials or mechanisms to apply 
these materials in an industrial way. The state of the art materials have very low efficiencies, which make them 
only useful in very specific applications. \\

The efficiency of a thermoelectric material is measured by the figure of merit $ZT=S^{2}\sigma T/\kappa$. $S$ is the
Seebeck coefficient, $\sigma$ the electrical conductivity and $\kappa$ the thermal conductivity. With a $ZT$ of 3 
in a broad temperature range, the thermoelectric materials would be competetitive against heat engines in terms of
heat transformation into electricity\cite{zhang2015thermoelectric}. In Fig. \ref{ztvst} we show the maximum figure 
of merit $ZT_{max}$ of record thermoelectric materials as a function of the year when they were discovered or 
synthesized.
\begin{figure}[h]
\begin{center}
\includegraphics[width=0.9\linewidth]{Figures/ztvstemp.png}
\caption[Record $ZT$ materials]{Record $ZT$ materials as a function of the year. The figure is taken from 
reference\cite{zhang2015thermoelectric}.}
\label{ztvst}
\end{center}
\end{figure}
As we can see, there is no material with a $ZT_{max}$ as high as 3. Even though there are some materials with a 
$ZT_{max}$ of 2 or higher, usually this only happens for very narrow temperature ranges\cite{zhao2014ultralow}. This 
limits the applications of these materials. \\

The power factor $PF$ is the product of the squared Seebeck coefficient and the electrical conductivity. The idea
behind a good thermoelectric material is very simple. We need materials with a high power factor and low thermal
conductivity, the so called electron crystals and phonon glasses. The main problem behind the increase of the $ZT$ 
is the fact that the physical magnitudes that are involved are correlated among themselves. This correlation limits 
the $ZT$. In Fig. \ref{decouple} we show a cartoon of the physical magnitudes that appear in the numerator of $ZT$ 
as a function of the doping for a given temperature. We choose to plot the physical magnitudes as a function of the
doping because it is one of the simplest manipulations to be applied. 
\begin{figure}[h]
\begin{center}
\includegraphics[width=0.8\linewidth]{Figures/decoupling.png}
\caption[Thermoelectric physical magnitudes]{Electrical conductivity and  Seebeck coefficient as a function of the 
carrier concentration in a typical semiconductor.}
\label{decouple}
\end{center}
\end{figure}
If we have a look to Fig. \ref{decouple} we can see that the power factor has a clear limitation. The electrical 
conductivity increases as a function of the carrier concentration, however, the Seebeck coefficient has the opposite 
trend, which makes the $PF$ to have a peak that cannot be increased. On top of this, the thermal conductivity of 
typical semiconductors increases as a function of the carrier concentration. This further decreases the carrier 
concentration at which the maximum $ZT$ but does not open any possibility to increase the $ZT$. \\

There are different ways to increase the thermoelectric efficiency of a material by physical and chemical 
manipulation. Materials have been doped\cite{kim2013engineered,pei2011stabilizing,heremans2008enhancement} or
nanostructured\cite{vineis2010nanostructured,minnich2009bulk} in order to get a high power factor combined with a 
low thermal conductivity, yielding, i.e., $ZT\simeq 2.2$ in PbTe\cite{hsu2004cubic}. It has been shown that, in the 
proximity to a phase transition $ZT$ may also soar, as in the case of Cu$_{2}$Se\cite{liu2013ultrahigh}. In these 
cases the thermal conductivity abruptly decreases close to the transition, reaching $ZT$ values as high as $2.5$. \\

In the last years, it has been shown that intrinsic semiconductors with an intrinsically low thermal conductivity 
may be materials with a very high $ZT$\cite{zhao2014ultralow,he2018remarkable}. The total thermal conductivity is 
the sum of electronic and lattice thermal conductivities. In intrinsics semiconductor, the main contribution to the 
thermal conductivity comes from the lattice as there are not many available charge carriers at normal temperatures. 
Therefore, the challenge is to find intrinsic semiconductors with an instrinsically low lattice thermal conductivity. 
The best thermoelectric material in this family is SnSe\cite{zhao2014ultralow} with a $ZT_{max}$ of 2.6. This 
material is not only the best thermoelectric in this family, but the best thermoelectric material overall. \\

At this point, theoretical material science plays a crucial role. Theory could reduce the experimental effort by 
searching for materials with an intrinsically low lattice thermal conductivity. The first-principles calculation of 
the lattice thermal conductivity is a complicated task and it is computationally demanding\cite{broido2007intrinsic}. 
Anyway, nowadays it is included in the toolkit\cite{giannozzi2009quantum} of any theoretical material scientist. 
However, it has been found, that the intrinsically low lattice thermal conductivity is associated, in many cases, to 
a strongly anharmonic lattice\cite{zhao2014ultralow,ribeiro2018strong}. This is the case, for instance, in SnSe. The 
strong anharmonicity not only causes low thermal conductivity, it also creates interesting properties, such as, 
structural phase transitions. These structural phase transition, in many cases, make the thermoelectric materials 
more efficient\cite{liu2013ultrahigh}. However, materials that suffer structural phase transitions are, in many 
cases, unstable within the harmonic approximation, on top of which, all the theory for calculating the 
lattice thermal conductivity is built by using perturbation theory\cite{paulatto2013anharmonic}. The calculation of 
the transition temperature is, for the same reason, a complicated task. In the last years there have been several 
methods and implementations\cite{hellman2013temperature,ljungberg2013temperature,magduau2013identification} that try 
to overcome these problems. \\

The main purpose of this thesis is to understand and predict this structural phase transformation that occur in 
thermoelectric materials and study what is the role they play in the thermoelectric efficiency. Once the structural 
and vibrational properties of these materials are studied, another purpose is to calculate the lattice thermal 
conductivity and understand what makes it so low in these materials with high $ZT$. For this purpose we have chosen 
the most efficient thermoelectric material SnSe and its isoelectronic counterpart SnS. In order to account for the 
anharmonic effects at a non-pertubative level we have applied the so called stochastic self-consistent harmonic 
approximation\cite{errea2013first,errea2014anharmonic,bianco2017second,monacelli2018pressure} (SSCHA). \\

In many cases, this structural phase transitions break the inversion symmetry of the 
material\cite{ribeiro2018strong,chang2016discovery}. This opens the door to interesting physical properties, such 
as, ferroelectricity. It has been shown that both properties, high thermoelectric efficiency and ferroelectricity, 
can coexist in the same material as it happen in SnTe\cite{ribeiro2018strong,zhang2013high}. There is an 
experimental work\cite{chang2016discovery} showing that the dimensionality could dramatically change the transition 
temperature to the ferroelectric state. In the same work\cite{chang2016discovery} they show that 
monolayer SnTe has a robust ferroelectric state that can be switched with an electric field. Thin film ferroelectrics 
are key for modern device applications\cite{lallart2011ferroelectrics}. However, it is known that there is a critical 
thickness for thin ferroelectrics, below which the depolarization field destroys the ferroelectric 
state\cite{batra1973new,zhong1994giant,dawber2005physics}. This effect decreases the transition temperature to the 
ferroelectric state as a function of the sample thickness\cite{fong2004ferroelectricity,fong2006stabilization} and 
sets a fundamental limit for technological applications. Monolayer SnSe could suffer a similar phase transition. In 
this thesis we have tried to understand the ferroelectric phase transition in monolayer 
SnSe\cite{chang2020controlled}. We have also tried to study the thermal properties of monolayer SnSe. \\

The out-of-plane vibrational modes play a crucial role in the heat transfer in two dimensional materials. 
The understanding of the mechanical and thermal properties of any 2D material is far from trivial. Even the 
possibility of having crystalline order in 2D has been long 
questioned\cite{landau_statistical_physics,mermin1968crystalline}. It has been theoretically shown that the harmonic 
approximation fails in the understanding of the out-of-plane mode, leading to unphysical consequences. Motivated by 
this and the frustrated attempts to calculate the lattice thermal conductivity in monolayer SnSe, which could be 
affected by the mentioned problems, we have studied the anharmonic effects in the out-of-plane phonon modes of 
graphene. \\

Long made short, the main purpose of this thesis is to study the anharmonic effects in materials where the harmonic 
approximation and perturbation theory break down. For that purpose we have chosen technologically relevant 
candidates, such as, very efficient thermoelectric materials, 2D ferroelectrics, and graphene.
