% Chapter 1

\chapter{2D materials: Graphene} % Main chapter title

\label{Chapter8} % For referencing the chapter elsewhere, use \ref{Chapter1} 

%----------------------------------------------------------------------------------------

% Define some commands to keep the formatting separated from the content 
%\newcommand{\keyword}[1]{\textbf{#1}}
%\newcommand{\tabhead}[1]{\textbf{#1}}
%\newcommand{\code}[1]{\texttt{#1}}
%\newcommand{\file}[1]{\texttt{\bfseries#1}}
%\newcommand{\option}[1]{\texttt{\itshape#1}}

%----------------------------------------------------------------------------------------

\section{Introduction}

The discovery of graphene\cite{novoselov2004electric,novoselov2005two,novoselov2005twoo,zhang2005experimental} proved 
the existence of 2D materials and launched their science and technology. Graphene is already a reality in different 
industrial products\cite{Kong2019path} that benefit from its fantastic properties. In particular, its mechanical and 
thermal properties are crucial for many of its current and future applications. For instance, graphene's uneven 
strength, stiffness, and lightness\cite{lee2008measurement} has been used to make stronger but lighter macroscopic 
objects, such as tennis rackets, shoes, and so on. Graphene, due to its very high thermal 
conductivity\cite{ghosh2008extremely}, has been already incorporated into electronic devices like mobile phones for 
efficient heat dissipation. \\

The understanding of the mechanical and thermal properties of graphene, and in general of any 2D material, 
is, however, far from trivial. Even the possibility of having crystalline order in 2D has been long 
questioned\cite{landau_statistical_physics,mermin1968crystalline}. Indeed, the mean square displacement calculated 
in the harmonic approximation diverges in the long wave-length limit, which means that the larger the sample of a 2D 
material the bigger the atomic displacements, which prevents any crystalline order\cite{landau_statistical_physics}. 
In Fig. \ref{rms_harmonic} we show the atomic displacements of carbon atoms in graphene as a function of the sample 
size, or the supercell used for the calculation.
\begin{figure}[h]
\includegraphics[width=0.8\linewidth]{Figures/rms-harmonic.eps}
\caption[Mean square atomic displacements of carbon atoms in graphene calculated using the harmonic dispersion]{Mean 
	square atomic displacements of carbon atoms in graphene calculated using the harmonic dispersion as a 
	function of the supercell size for different temperatures.}
\label{rms_harmonic}
\end{figure}
As the figure clearly shows, the mean square atomic displacements diverge at finite temperatures as a function of 
the sample size. Mermin's theorem\cite{mermin1968crystalline} proves that, even without assuming the harmonic 
approximation, long-range crystalline order is not possible in a two-dimensional crystal described by a pair 
potential. Experimentally, however, crystalline order has been observed in suspended 
graphene\cite{meyer2007structure}, although it shows ripples that seem to be intrinsic\cite{fasolino2007intrinsic}. \\

Most of the problems related to the mechanical stability of graphene are caused by the quadratic dispersion of the 
acoustic out-of-plane (ZA) mode that is obtained in the harmonic approximation. The harmonic phonon spectrum of 
graphene is shown in Fig. \ref{harmonic-graphene}.
\begin{figure}[h]
\includegraphics[width=0.55\linewidth]{Figures/harmonic-graphene.eps}
\includegraphics[width=0.45\linewidth]{Figures/harmonic-graphene-logarithmic.eps}
\caption[Graphene harmonic phonons]{Left: Harmonic phonon spectrum of graphene. Right: Harmonic ZA phonon dispersion 
	of graphene in logarithmic scale. We include a quadratic line for reference.}
\label{harmonic-graphene}
\end{figure}
We can see that the LA and TA modes have a linear dispersion close to the point $\Gamma$ as it happens in 3D 
crystals. However, the ZA mode, which atomic displacements are directed in the out-of-plane direction, has a 
quadratic dispersion. This is very clear in the plot in logarithmic scale where we have included a quadratic line for 
reference. The quadratic dispersion is given by the rotational symmetry, together with the fact that in a strictly 
two-dimensional system force constants involving an in-plane and an out-of-plane displacement 
vanish\cite{katsnelson2013graphene}. The mathematical proof of the quadratic dispersion is given in 
section \ref{quadratic-dispersion}. \\

The quadratic dispersion also creates spurious divergences in physical properties calculated perturbativaly on top of 
the harmonic result. The phonon linewidths of the in-plane acoustic longitudinal (LA) and transverse (TA) phonons 
calculated perturbatively do not vanish as the momentum decreases\cite{paulatto2013anharmonic}, precisely, because of 
the $\sim q^2$ dispersion of the ZA modes\cite{bonini2012acoustic}. The linewidths of the LA and TA modes are shown 
in Fig. \ref{lata-harmonic-lw}. 
\begin{figure}[h]
\includegraphics[width=0.8\linewidth]{Figures/lw-graphene-harmonic.eps}
\caption[Graphene harmonic linewidths]{Linewidths (full width at half maximum, FWHM) of the LA and TA modes 
	calculated using the harmonic phonon spectrum for the three phonon scattering processes and 3BFC calculated 
	within the SCHA at $0$ K.}
\label{lata-harmonic-lw}
\end{figure}
The figure clearly shows how the linewidths do not vanish at small momenta. This yields to the conclusion that for 
small enough momentum phonons do not live long enough for vibrating one period and the quasiparticle picture is lost. 
Consequently, the thermal conductivity calculated with the perturbative phonon 
lifetimes\cite{fugallo2013ab,fugallo2014thermal} does not converge with the sample size\cite{bonini2012acoustic}.
The application of strain linearizes the dispersion of the ZA phonons and makes the linewidth of the LA and TA 
phonon modes vanish together with their frequency at small momenta, keeping the phonon 
picture\cite{bonini2012acoustic}. \\

It has been argued\cite{wang2016anharmonic,los2009scaling,katsnelson2013graphene,zakharchenko2009finite,mariani2008flexural,amorim2014thermodynamics,de2012bending} that the anharmonic coupling between in-plane and out-of-plane phonon 
modes renormalizes the dispersion of the ZA phonon modes, providing it with a linear term at small momenta that 
somewhat cures the pathologies. This conclusion has been drawn by Monte Carlo simulations with empirical 
potentials\cite{wang2016anharmonic,los2009scaling,katsnelson2013graphene,zakharchenko2009finite} and by using a 
membrane continuum  Hamiltonian that mimics the interaction among acoustic 
modes\cite{mariani2008flexural,amorim2014thermodynamics,de2012bending}. Anharmonic phonons expected experimentally 
should be calculated from the free energy Hessian, i.e., diagonalizing the  $[\frac{\partial F}{\partial \mathcal{R}_a \partial \mathcal{R}_b}]_{0}/\sqrt{M_aM_b}$ matrix, where $F$ is the free energy calculated including anharmonic 
effects and $\boldsymbol{\mathcal{R}}$ the centroid positions that determine the most probable ionic 
positions\cite{bianco2017second}. If this is the case, as $F$ obeys the same symmetry properties as $V$, a similar 
$\sim q^2$ dispersion would be expected for the ZA mode even if anharmonic effects are included in the calculation 
of $F$. Actually, measurements done with helium diffraction show a ZA mode with a quadratic 
dispersion\cite{al2016acoustic,al2015helium,al2018resolving}, though the linearization regime may not be seen and 
substrate effects may be important. The remaining question is thus whether the ZA modes really have a quadratic 
dispersion, and, if it is so, how the mechanical stability and thermal properties of graphene can be explained. \\

In this work we show that a quadratic dispersion of the ZA mode is actually expected for graphene provided that it 
is calculated from the Hessian of the anharmonic free energy $F$, and that it is compatible with the absence of 
divergences. We estimate $F$ within the self-consistent harmonic approximation (SCHA). We apply the SCHA in its 
stochastic implementation (SSCHA) making use of a machine learning atomistic potential trained with density 
functional theory\cite{rowe2018development}. We also solve the SCHA equations in a membrane continuum Hamiltonian 
which provides results at smaller momenta.

\section{Quadratic dispersion of ZA modes within the harmonic approximation}
\label{quadratic-dispersion}

In this section we prove that the dispersion of the ZA phonon modes close to the point $\Gamma$ is quadratic in the 
harmonic approximation. \\

For that purpose, by using the inversion symmetry and the ASR in Eq. \ref{acoustic-sum-rule} we will rewrite Eq. \ref{2bfc-fourier-simple} in section \ref{harmonic-approximation} as
\begin{multline}
 \phi_{s_{1}s_{2}}^{\alpha_{1}\alpha_{2}}(\mathbf{q})=\sum_{\mathbf{T}}\phi_{s_{1}s_{2}}^{\alpha_{1}\alpha_{2}}(\mathbf{T},0)e^{-i\mathbf{q}\cdot\mathbf{T}}=\\\frac{1}{2}\sum_{\mathbf{T}}\phi_{s_{1}s_{2}}^{\alpha_{1}\alpha_{2}}(\mathbf{T},0)(e^{-i\mathbf{q}\cdot\mathbf{T}}+e^{i\mathbf{q}\cdot\mathbf{T}}-2)=-2\sum_{\mathbf{T}}\phi_{s_{1}s_{2}}^{\alpha_{1}\alpha_{2}}(\mathbf{T},0)sin^{2}(\mathbf{q}\cdot\mathbf{T}/2).
\end{multline}
And by Taylor expanding this equation close to the point $\Gamma$ we get
\begin{equation}
\label{need-zero}
\phi_{s_{1}s_{2}}^{\alpha_{1}\alpha_{2}}(\mathbf{q})\simeq -\frac{1}{2}\sum_{\alpha\beta}q_{\alpha}q_{\beta}\sum_{\mathbf{T}}\phi_{s_{1}s_{2}}^{\alpha_{1}\alpha_{2}}(\mathbf{T})T_{\alpha}T_{\beta}+O(q^{4}),
\end{equation} 
$\alpha$, $\beta$ being $x$, $y$, and $z$ Cartesian components. \\

The fact that a material is bidimensional and the equilibrium atomic positions are contained in the $z=0$ plane,
makes that all the 2BFC of the type $\phi_{s_{1}s_{2}}^{xz}(\mathbf{T})$, $\phi_{s_{1}s_{2}}^{yz}(\mathbf{T})$ are zero. This is translated into the fact that the out-of-plane $z$ direction is an eigendirection of the system and the 
harmonic frequency of the ZA mode is given by
\begin{equation} 
\omega_{ZA}^{2}(\mathbf{q})=\sum_{s_{2}}D_{1s_{2}}^{zz}(\mathbf{q}). 
\end{equation} 
Therefore, by looking at Eq. \ref{need-zero} we can see that for proving that the dispersion of the ZA modes is quadratic we need to prove that 
$\sum_{\mathbf{T}}\phi_{s_{1}s_{2}}^{zz}(\mathbf{T})T_{\alpha}T_{\beta}=0$. Actually, this equality is enforced by the rotational invariance of the system. \\

In conclusion, the 2D character of the 2BFC and the rotational invariance of the system make the harmonic dispersion of the ZA modes close to the point $\Gamma$ to be quadratic.

\section{Empirical potential benchmark and calculation parameters}

For calculating the forces needed in the SCHA minimization\cite{errea2014anharmonic} we have used an empirical 
potential trained with machine learning and density functional theory (DFT) forces. The details about the machine 
learning training are explained in Ref. \cite{rowe2018development}. Here we have benchmarked the ability of 
the potential to account for the anharmonic effects. For that purpose we have applied the SCHA method by using 
DFT and empirical forces in a $2\times2$ supercell and we have checked the anharmonic effects in the optical 
modes at the $\Gamma$ point. The machine learning potential is trained with the exchange-correlation in 
Ref. \cite{dion2004van} and for the DFT calculations we have applied a PBE\cite{perdew1996generalized} 
ultrasoft pseudopotential\cite{vanderbilt1990soft} with Van der Walls corrections\cite{barone2009role}. The 
results are shown in Figs. \ref{benchmark-spectrum} and \ref{benchmark}.
\begin{figure}[ht]
\includegraphics[width=0.99\linewidth]{Figures/abinito-vs-empirical.eps}
\caption[Harmonic phonon spectrum of graphene calculated with the empirical                                 
        potential and $ab$ $initio$.]{Harmonic phonon spectrum of graphene calculated with the empirical potential 
	and $ab$ $initio$. The calculations are done in a $6\times6$ supercell.}
\label{benchmark-spectrum}
\end{figure}
\begin{figure}[ht]
\includegraphics[width=0.49\linewidth]{Figures/bm1.eps}
\includegraphics[width=0.49\linewidth]{Figures/bm2.eps}
\caption[Harmonic, SCHA and physical frequencies (static) using the DFT and machine learning (ML) forces.]{Harmonic, 
	SCHA and physical frequencies (static) using the DFT and machine learning (ML) forces. The left panel shows 
	the in-plane optical frequency at the $\Gamma$ point and the right panel the out-of-plane one.}
\label{benchmark}
\end{figure}
As we can see in Fig. \ref{benchmark-spectrum}, the two potentials provide very similar harmonic phonons, or what is 
the same, very similar forces. Due to the different exchange correlation functional there is a slight offset in 
Fig. \ref{benchmark}, however, the anharmonic lineshifts in both $\boldsymbol{D}^{(S)}$ and $\boldsymbol{D}^{(F)}$ 
are very well captured within the empirical potential. \\

For the self-consistent DFT calculations we have used a plane wave cutoff of $70$ Ry and a $700$ Ry cutoff for the 
density. For the Brillouin zone integration we have used a Monkhorst pack grid\cite{monkhorst1976special} of 
$32\times32$ points with a Gaussian smearing of $0.02$ Ry. For the linewidth calculations in the main text we have 
used a grid of $400\times400$ grid with a Gaussian smearing of $1$ cm$^{-1}$. For the stress calculation in order 
to account for the thermal expansion we have used a $10\times10$ supercell. We have used the same supercell for 
the SCHA and physical frequency calculations. For the linewidth calculations we have used 3BFC calculated in 
$3\times3$ supercells. We have tested all the calculations with denser grids and bigger supercells.

\section{Graphene without stress}

In order to calculate phonon spectra in unstrained graphene at any temperature, we calculate the SCHA stress tensor 
following the procedure in section \ref{scha-stress-section} and pick the lattice parameter that sets it to zero at 
each temperature. For that purpose, we calculate the stress in ranges of $0.0005$ $\AA$ and then we interpolate the 
result, which is fitted very good with a linear function. The lattice parameter calculated in this way includes 
anharmonic effects as well as the effect of quantum and thermal fluctuations. In order to properly account for 
thermal expansion, all the phonon spectra shown in this work that are obtained with the atomistic potential are 
calculated with the lattice parameter that gives a null SCHA stress at each temperature. The harmonic spectra on the 
contrary is always calculated at the lattice parameter that minimizes $U$. The temperature dependence of the lattice 
parameter and the thermal expansion coefficient are shown in Fig. \ref{lattice}.
\begin{figure}[ht]
\includegraphics[width=0.99\linewidth]{Figures/lattice.eps}
\caption[Lattice parameter of graphene as a function of temperature]{a) Lattice parameter of graphene as a function 
	of temperature obtained with the SCHA using a machine learning atomistic potential. Both quantum (black) and 
	classical (blue) calculations are included. The classical result is calculated setting $\hbar=0$ in the SCHA 
	free energy. The temperature-independent frozen nuclei (FN) result corresponds to the lattice parameter that 
	minimizes the Born Oppenheimer potential $U$. Results obtained by Rowe et al.\cite{rowe2018development} are 
	also inlcuded for comparison. Solid black and blue lines correspond to cubic fits. The black dashed line 
	corresponds to the quasiharmonic result. b) Thermal expansion coefficient (CTE) calculated as 
	$CTE=\frac{1}{A}\frac{\partial A}{\partial T}$, $A$ being the area of the membrane. Red line is directly 
	taken from \cite{rowe2018development}. Black and blue lines are calculated using the cubic fits in a).}
\label{lattice}
\end{figure}
We include the results of Rowe et al.\cite{rowe2018development}, which do not account for quantum effects as they 
are done with molecular dynamics (MD). They use the same potential as we do. For comparison we also include SCHA 
calculations in he classical limit and within the quasiharmonic (QH) approximation. Our quantum calculations 
correctly capture the negative thermal expansion of graphene up to $\sim$ 750 K that has been estimated in previous 
theoretical works\cite{rowe2018development,zakharchenko2009finite}. Our quantum calculations show a larger lattice 
parameter and a more pronounced thermal expansion (in absolute value) at low temperatures. This is not surprising as 
classical calculations neglect quantum fluctuations and, consequently, underestimate the fluctuations associated to 
the high-energy optical modes (the highest energy phonon modes require temperatures of around 2000 K to be thermally 
populated). Actually, the classical result approaches the quantum at high temperatures. This remarks the importance 
of considering quantum effects in the evaluation of thermodynamical properties of graphene. Our classical results 
and the MD calculations of Rowe et al.\cite{rowe2018development} are in agreement (within their error of $0.003$ 
$\AA$) at low temperatures. However, at high temperatures they deviate and our calculations approach the quantum 
result as it should be. The difference may be due to the non-trivial calculation of the lattice parameter within MD. 
We have also included the quasiharmonic (QH) result. The lattice parameter within the QH approximation is calculated 
by calculating the harmonic free energy for different temperatures and lattice paramters. The lattice parameter 
dependence comes from the different harmonic phonons at different lattice parameters. Then, we take the lattice 
parameter that minimizes the harmonic free energy at each temperature. It is worth noting that this approximation is 
not valid to calculate the thermal expansion of graphene due to the imaginary phonon frequencies that appear close 
to $\Gamma$ for the ZA mode already at $500$ K. In the shadowed region in Fig. \ref{lattice} the QH is not 
valid. This result questions the results obtained with this approach that yield a negative thermal expansion at all 
temperatures\cite{mounet2005first}.

\section{Graphene phonons}

In Fig. \ref{spectrum-graphene} we compare the harmonic phonon spectra with the one obtained from the SCHA dynamical 
matrix $\boldsymbol{D}^{(S)}$ as well as the spectra obtained from the Hessian of the SCHA free 
energy $\boldsymbol{D}^{(F)}$. 
\begin{figure*}[ht]
\includegraphics[width=0.32\linewidth]{Figures/T0.eps}
\includegraphics[width=0.32\linewidth]{Figures/T300.eps}
\includegraphics[width=0.32\linewidth]{Figures/T2000.eps}
	\caption[Graphene harmonic and anharmonic phonons]{Harmonic phonon spectra together with the SCHA phonons 
	$\boldsymbol{D}^{(S)}$ (labeled as ``SCHA'') and those obtained from the Hessian of the SCHA free energy 
	$\boldsymbol{D}^{(F)}$ as (labeled as ``Physical''). Results at 0 K (a), 300 K (b), and 2000 K (c) are shown. 
	(d), (e) and (f) show only the dispersion of the ZA mode close to $\Gamma$ in logarithmic scale at 0 K, 300 
	K, and 2000 K, respectively. In the small panels the dispersion corresponds to the $\Gamma$M direction. For 
	reference, the M point is at $1.4662$ $\AA^{-1}$ at $0$ K, at $1.4671$ $\AA^{-1}$ at $300$ K, and at $1.4652$ 
	$\AA^{-1}$ at $200$ K. The harmonic result (solid black) is computed at the lattice parameter that minimizes 
	$U$, while the other results include thermal expansion (see Fig. \ref{lattice}). The dashed black lines 
	correspond to harmonic calculations including thermal expansion (TE). All these calculations are performed 
	with the machine learning atomistic potential.}
\label{spectrum-graphene}
\end{figure*}
The main conclusion is that while the dispersion of the ZA modes obtained from $\boldsymbol{D}^{(S)}$ is linearized, 
the physical phonons given by the Hessian of $F$ ($\boldsymbol{D}^{(F)}$) become close to a quadratic dispersion and 
approach the harmonic dispersion, as expected by symmetry. The SCHA ZA frequencies suffer a blue-shift with respect 
to the harmonic ones, but are red-shifted once the Hessian is calculated. Both shifts are bigger when the temperature 
is increased, but the quadratic behavior of the ZA modes is always recovered regardless of the temperature. With 
respect to the optical modes, we observe that the frequencies red-shift with increasing temperature in agreement with 
experiments and theory\cite{calizo2007temperature,bonini2007phonon}. It is noteworthy that the phonon frequencies 
obtained from the Hessian of $F$ should only be understood as the physical phonons in the static 
limit (see sections \ref{perturbation-theory-third}, \ref{pertubative-limit-sscha}, and \ref{dynamical-sscha}). 
However, as shown in Fig. \ref{static-dynamic}, 
\begin{figure}[ht]
\includegraphics[width=0.32\linewidth]{Figures/T0-za.eps}
\includegraphics[width=0.32\linewidth]{Figures/T300-za.eps}
\includegraphics[width=0.32\linewidth]{Figures/T2000-za.eps}
	\caption[Graphene harmonic and anharmonic ZA phonons]{Harmonic, SCHA, and physical phonons (static and 
	dynamic) calculated at 0 K a), 300 K b), and 2000 K c).}
\label{static-dynamic}
\end{figure}
the static phonons agree with those derived from the dynamical (see section \ref{dynamical-sscha}) theory 
and are thus good representatives of the physical phonons expected experimentally. \\

\section{Mean square displacements and linewidths}

Even if the anharmonic correction to the phonon spectra may look small in Fig. \ref{spectrum-graphene}, it has a 
huge impact on the mechanical and thermal properties of graphene. As shown in Fig. \ref{problems}, 
\begin{figure*}[ht]
\includegraphics[width=0.49\linewidth]{Figures/rms.eps}
\includegraphics[width=0.49\linewidth]{Figures/lw-graphene.eps}
\caption[SCHA mean square atomic displacements and linewidths]{(a) Mean square atomic displacement of carbon atoms 
	in graphene calculated using the harmonic and the SCHA density matrices. The harmonic result is always 
	computed at the lattice parameter that minimizes $U$, while the SCHA results include thermal expansion (see 
	Fig. \ref{lattice}). (b) Linewidths (full width at half maximum) of LA and TA phonon modes at $300$ K 
	calculated within the harmonic approximation and the SCHA. Results in both figures are obtained with the 
	machine learning atomistic potential.}
\label{problems}
\end{figure*}
when calculating the mean square atomic displacement with the SCHA density matrix as 
$\langle \mathbf{u}^2 \rangle_{\rho_{\mathcal{H}}}$, the dramatic divergences with the sample size obtained in the 
harmonic case are critically suppressed, clearly showing the contribution of anharmonicity to the crystalline order 
of graphene. The divergences are suppressed precisely because the SCHA phonon frequencies obtained from 
$\boldsymbol{D}^{(S)}$ that build $\rho_{\mathcal{H}}$ are linear at small momenta (see 
Fig. \ref{spectrum-graphene}). Thus, even if the phonons obtained from the SCHA free energy Hessian 
$\boldsymbol{D}^{(F)}$ are quadratic, the fact that the density matrix used to compute thermodynamic properties is 
built with linearized phonons is responsible for curing the divergences. \\

The SCHA phonon frequencies obtained from $\boldsymbol{D}^{(S)}$ also provide an appropriate basis to calculate 
phonon linewidths\cite{bianco2017second,aseginolaza2019phonon}. The results presented in Fig. \ref{problems} clearly 
show that the linearization of the SCHA frequencies dramatically changes the linewidth of the LA and TA modes at 
small momenta by making them smaller as momentum decreases. This result recovers the quasiparticle picture for these 
modes. Similar results are obtained in Ref. \cite{bonini2012acoustic}. However, in this work linewidths of LA 
and TA modes at small momenta only vanish when strain is applied. Somewhat strain has an analogous effect to 
anharmonicity by linearizing the ZA dispersion. The problem is that strain linearizes both SCHA and physical phonons. 
We show here that there is no need of strain to have physically well-defined phonon-linewidths provided by linear 
SCHA phonons together with quadratic physical phonons. Anharmonicity is responsible for it.

\section{SCHA applied to a continuum membrane Hamiltonian}

In order to obtain results at very small momenta and reinforce the conclusions drawn with the atomistic calculations 
with the machine learning potential, we also solve the SCHA equations in a continuum membrane Hamiltonian. This model 
has been widely used in the literature to describe graphene as an elastic membrane as well as to account for the 
coupling between in-plane and out-of plane acoustic 
modes\cite{mariani2008flexural,amorim2014thermodynamics,de2012bending}. The most general rotationally invariant 
continuum model potential for phonons in free-standing 2D membranes up to the fourth-order with respect to the 
phonon fields has the following form:
\begin{multline}
 \label{membrane-potential}
 U=\frac{1}{2}\int_{\Omega}d^{2}x[\kappa(\partial^{2}h)^{2}+C^{ijkl}\partial_{i}u_{j}\partial_{k}u_{l}+C^{ijkl}\partial_{i}u_{j}\partial_{k}h\partial_{l}h+\\ +\frac{C^{ijkl}}{4}\partial_{i}h\partial_{j}h\partial_{
 k}h\partial_{l}h+\frac{C^{ijkl}}{2}\partial_{i}\boldsymbol{u}\cdot\partial_{j}\boldsymbol{u}\partial_{k}h\partial_{l}h+\\+C^{ijkl}\partial_{i}u_{j}\partial_{k}\boldsymbol{u}\cdot\partial_{l}\boldsymbol{u}+
 \frac{C^{ijkl}}{4}\partial_{i}\boldsymbol{u}\cdot\partial_{j}\boldsymbol{u}\partial_{k}\boldsymbol{u}\cdot\partial_{l}\boldsymbol{u}].
\end{multline}
$\boldsymbol{u}(\boldsymbol{x})$ and $h(\boldsymbol{x})$ are the in-plane and out-of-plane displacement 
fields, respectively, and $\boldsymbol{x}$ is the 2D position vector in the membrane. $\kappa$ is the bending 
rigidity of the membrane and the tensor 
$C^{ijkl}=\lambda\delta^{ij}\delta^{kl}+\mu(\delta^{ik}\delta^{jl}+\delta^{il}\delta^{jk})$ contains the Lame's 
coefficients $\lambda$ and $\mu$ and Kronecker deltas. As it is a continuum model it only accounts for acoustic 
modes and the second-order terms in the potential are the harmonic terms. The harmonic frequencies in normal 
coordinates space are $\omega_{ZA}(q)=\sqrt{\phi_{h}(\boldsymbol{q})/\rho}$, 
$\omega_{LA}(q)=\sqrt{(\phi_{LA}(\boldsymbol{q})/\rho}$, and 
$\omega_{TA}(q)=\sqrt{\phi_{TA}(\boldsymbol{q})/\rho}$, where $\rho$ is the mass density of the membrane and 
$\phi_{ZA}(\boldsymbol{q})=\kappa |\boldsymbol{q}|^{4}$, 
$\phi_{LA}(\boldsymbol{q})=(\lambda+\mu)|\boldsymbol{q}|^{2}$, and
$\phi_{TA}(\boldsymbol{q})=\mu|\boldsymbol{q}|^{2}$ are the harmonic 2BFC. LA and TA correspond to the in-plane 
longitudinal and transversal phonons and ZA corresponds to the out-of-plane phonons. The higher-order terms account 
for the phonon-phonon anharmonic interactions. The thermal expansion is included in this formalism by changing the 
in-plane derivatives $\partial_{i}u_{j}\rightarrow \partial_{i}u_{j}+\delta^{ij}\delta a$, with 
$\delta a=(a-a_{0})a_{0}$, $a_{0}$ being the lattice parameter that minimizes $U$. \\

It is not possible to apply the SCHA analytically in Eq. \ref{membrane-potential}. As far as we know, the simplest 
approximation that allows applying the SCHA analytically is to neglect fourth-order terms including in-plane phonon 
fields. In that case the potential can be written as
\begin{multline}
\label{membrane-potential-tommaso}
U=\frac{1}{2}\int_{\Omega}d^{2}x[C^{ijkl}\partial_{i}u_{j}\partial_{k}u_{l}+\kappa(\partial^{2}h)^{2}+
C^{ijkl}\partial_{i}u_{j}\partial_{k}h\partial_{l}h+\frac{C^{ijkl}}{4}\partial_{i}h\partial_{j}h\partial_{k}h\partial_{l}h]+\\+2\Omega(\lambda+\mu)(\delta a)^{2}+\delta a(\lambda+\mu)\int_{\Omega}d^{2}x\partial_{k}h\partial_{k}h.
\end{multline}
By Fourier transforming this potential in $\boldsymbol{q}$ space and applying Eq. \ref{scha-equation}, we arrive to 
the SCHA equations (see appendix for further derivations)
\begin{equation}
\label{tommaso-eq1}
\frac{\partial\mathcal{F}}{\partial\delta a}=0\rightarrow\delta a=-\frac{1}{4\Omega}\sum_{\boldsymbol{q}}|\boldsymbol{q}|^{2}g[\Omega_{ZA}^{(S)}(\boldsymbol{q})],
\end{equation}   
\begin{equation}
\label{tommaso-eq2}
\frac{\partial\mathcal{F}}{\partial\Phi_{ZA}(\boldsymbol{q})}=0\rightarrow\Phi_{h}(\boldsymbol{q})=\kappa|\boldsymbol{q}|^{4}+2\delta a(\lambda+\mu)|\boldsymbol{q}|^{2}+\frac{\lambda+2\mu}{2\Omega}\sum_{\boldsymbol{k}}g[\Omega_{ZA}^{(S)}(\boldsymbol{k})][|\boldsymbol{q}|^{2}|\boldsymbol{k}|^{2}+2(\boldsymbol{q}\cdot\boldsymbol{k})^{2}].
\end{equation}
$\Phi_{ZA}(\boldsymbol{q})$ are the SCHA 2BFC that correspond to the out-of-plane modes and the frequencies are 
$\Omega_{ZA}^{(S)}(\boldsymbol{q})=\sqrt{\Phi_{ZA}(\boldsymbol{q})/\rho}$. The function $g$ is defined as
$g(x)=coth(x/2T)/2\rho x$, $T$ being the temperature. Already from Eq. \ref{tommaso-eq2} it can be seen that the 
non-zero $\delta a$ provides the SCHA frequency with a linear term. By inserting Eq. \ref{tommaso-eq1} in 
Eq. \ref{tommaso-eq2} the SCHA 2BFC can be written as 
$\Phi_{ZA}(\boldsymbol{q})=\kappa|\boldsymbol{q}|^{4}+\gamma|\boldsymbol{q}|^{2}$. The expression for $\gamma$ is 
given in appendix. Therefore, we have shown that the SCHA flexural acoustic (ZA) modes have a linear dispersion 
close to the point $\Gamma$. \\

For calculating the physical phonons, given by the second derivative of the free energy, an analogous formula to 
Eq. \ref{free-energy-hessian} can be found within the membrane formalism (see appendix). By applying this formula 
we arrive to the following expression for the physical phonons at $T=0$
\begin{equation}
 \Phi_{h}^{(F)}(\boldsymbol{q})=\kappa|\boldsymbol{q}|^{2}+(\gamma-\sigma)|\boldsymbol{q}|^{2}+
 O(|\boldsymbol{q}|^{4}).
\end{equation} 
The formula of $\sigma$ is given in the appendix. The positive number $\sigma$ makes the linear term in the physical 
phonon frequencies $40\%$ smaller than in the SCHA case, however, it does not remove it. Actually, it can be argued 
that this non-zero linear term in the physical frequencies arises because the potential in 
Eq. \ref{membrane-potential-tommaso} is not rotationally invariant. In order to clarify these results we have 
applied the SCHA numerically in the full potential. \\

By taking the full potential in Eq. \ref{membrane-potential}, Fourier transforming it and applying 
Eq. \ref{scha-equation} we arrive to the SCHA equations for the rotationally invariant membrane
\begin{multline}
 \frac{\partial\mathcal{F}(\mathcal{V})}{\partial\delta a}=0=2\Omega(2\delta a+3\delta a^{2}+\delta a^{3})(\lambda+
\mu)+\frac{1}{2}\sum_{\boldsymbol{q}}g[\Omega_{SCHA}^{(ZA)}(\boldsymbol{q})]2(1+\delta a)(\lambda+\mu)|\boldsymbol{
 q}|^{2}\\+\frac{1}{2}\sum_{\boldsymbol{q}}g[\Omega_{SCHA}^{(LA)}(\boldsymbol{q})][2(1+\delta a)(\lambda+2\mu)|\boldsymbol{q}|^{2}+2(1+\delta a)(\lambda+\mu)|\boldsymbol{q}|^{2}]+\\\frac{1}{2}\sum_{\boldsymbol{q}}g[\Omega_{
 SCHA}^{(TA)}(\boldsymbol{q})][2(1+\delta a)\mu|\boldsymbol{q}|^{2}+2(1+\delta a/2)(\lambda+\mu)|\boldsymbol{q}|^{2}],
\end{multline}
\begin{multline}
 \Phi_{ZA}(\boldsymbol{q})=\kappa|\boldsymbol{q}|^{4}+2(1+\delta a/2)\delta a(\lambda+\mu)|\boldsymbol{q}|^{2}+\frac{\lambda+2\mu}{2\Omega}\sum_{\boldsymbol{k}}g[\Omega_{SCHA}^{(ZA)}(\boldsymbol{k})][|\boldsymbol{q}|^{2}|\boldsymbol{
 k}|^{2}+2(\boldsymbol{q}\cdot\boldsymbol{k})^{2}]+\\\frac{1}{2\Omega}\sum_{\boldsymbol{k}}\{g[\Omega_{SCHA}^{(LA)}(\boldsymbol{k})]+g[\Omega_{SCHA}^{(TA)}(\boldsymbol{k})]\}[\lambda|\boldsymbol{q}|^{2}|\boldsymbol{k}|^{2}+2\mu(
 \boldsymbol{q}\cdot\boldsymbol{k})^{2}],
\end{multline}
\begin{multline}
 \Phi_{LA}(\boldsymbol{q})=(\lambda+2\mu)|\boldsymbol{q}|^{2}+2(1+\delta a/2)\delta a(\lambda+2\mu)|\boldsymbol{q}|^{2}+2(1+\delta a/2)\delta a(\lambda+\mu)|\boldsymbol{q}|^{2}\\+\frac{1}{2\Omega}\sum_{\boldsymbol{k}}g[\Omega_{SCHA}^{
 (ZA)}(\boldsymbol{k})][\lambda|\boldsymbol{q}|^{2}|\boldsymbol{k}|^{2}+2\mu(\boldsymbol{q}\cdot\boldsymbol{k})^{2}]+\\\frac{1}{4\Omega}\sum_{\boldsymbol{k}}\{4g[\Omega_{SCHA}^{(TA)}(\boldsymbol{k})][\lambda(\boldsymbol{q}\cdot\boldsymbol{
 k})^{2}+\mu|\boldsymbol{q}|^{2}|\boldsymbol{k}|^{2}+\mu(\boldsymbol{q}\cdot\boldsymbol{k})^{2}](\hat{\boldsymbol{q}_{\perp}}\cdot\hat{\boldsymbol{k}})+ \\ 2g[\Omega_{SCHA}^{(TA)}(\boldsymbol{k})][\lambda|\boldsymbol{q}|^{2}|\boldsymbol{
 k}|^{2}+2\mu(\boldsymbol{q}\cdot\boldsymbol{k})^{2}]+ \\ 2g[\Omega_{SCHA}^{(LA)}(\boldsymbol{k})][\lambda|\boldsymbol{q}|^{2}|\boldsymbol{k}|^{2}+2\mu(\boldsymbol{q}\cdot\boldsymbol{k})^{2}]+ \\ 4g[\Omega_{SCHA}^{(LA)}(\boldsymbol{
 k})][\lambda(\boldsymbol{q}\cdot\boldsymbol{k})^{2}+\mu|\boldsymbol{q}|^{2}|\boldsymbol{k}|^{2}+\mu(\boldsymbol{q}\cdot\boldsymbol{k})^{2}](\hat{\boldsymbol{q}}\cdot\hat{\boldsymbol{k}})\}
\end{multline}
and,
\begin{multline}
 \Phi_{TA}(\boldsymbol{q})=\mu|\boldsymbol{q}|^{2}+2(1+\delta a/2)\delta a\mu|\boldsymbol{q}|^{2}+2(1+\delta a/2)\delta a(\lambda+\mu)|\boldsymbol{q}|^{2}\\+\frac{1}{2\Omega}\sum_{\boldsymbol{k}}g[\Omega_{SCHA}^{
 (ZA)}(\boldsymbol{k})][\lambda|\boldsymbol{q}|^{2}|\boldsymbol{k}|^{2}+2\mu(\boldsymbol{q}\cdot\boldsymbol{k})^{2}]+\\\frac{1}{4\Omega}\sum_{\boldsymbol{k}}\{
 4g[\Omega_{SCHA}^{(TA)}(\boldsymbol{k})][\lambda(\boldsymbol{q}\cdot\boldsymbol{k})^{2}+\mu|\boldsymbol{q}|^{2}|\boldsymbol{k}|^{2}+\mu(\boldsymbol{q}\cdot\boldsymbol{k})^{2}](\hat{\boldsymbol{q}_{\perp}}\cdot\hat{
 \boldsymbol{k}_{\perp}})]+\\4g[\Omega_{SCHA}^{(LA)}(\boldsymbol{k})][\lambda(\boldsymbol{q}\cdot\boldsymbol{k})^{2}+\mu|\boldsymbol{q}|^{2}|\boldsymbol{k}|^{2}+\mu(\boldsymbol{q}\cdot\boldsymbol{k})^{2}](\hat{\boldsymbol{
 q}_{\perp}}\cdot\hat{\boldsymbol{k}})+ \\ 2g[\Omega_{SCHA}^{(TA)}(\boldsymbol{k})][\lambda|\boldsymbol{q}|^{2}|\boldsymbol{k}|^{2}+2\mu(\boldsymbol{q}\cdot\boldsymbol{k})^{2}]  \}.
\end{multline}
We have solved these equation by using the Newton-Raphson method\cite{ypma1995historical} in a circular discretized grid of $60\times60$ $\boldsymbol{q}$ 
points. The results are shown in Fig. \ref{membrane-results}. 
\begin{figure}[h]
\includegraphics[width=\linewidth]{Figures/membrane.eps}
\caption[ZA harmonic and anharmonic phonons in the membrane model.]{a) Frequency of the ZA mode in the harmonic 
	approximation, within the SCHA and obtained from the Hessian of the SCHA (labeled as ``Physical'') at 0 K in 
	the membrane model. We name rotationally invariant (RI) the results considering the full potential in 
	Eq. \ref{membrane-potential}. We name no rotationally invariant (No RI) the results neglecting the last three 
	terms in Eq. \ref{membrane-potential}. b) Frequency of the ZA mode in the harmonic approximation, within the 
	SCHA and obtained from the Hessian of the SCHA at $0$ K and $100$ K in the membrane model. We also include 
	the SCHA and physical phonons at $100$ K without considering thermal expansion (NTE).}
\label{membrane-results}
\end{figure}
All conclusions drawn with the atomistic model are confirmed and put in solid grounds. Again the ZA phonons obtained 
from the SCHA force constants get linearized at small momenta and the linearization is bigger for higher 
temperatures. These results are consistent with the anharmonic linearization obtained for this mode in previous 
calculations\cite{mariani2008flexural,amorim2014thermodynamics,de2012bending}. However, when the phonons are 
calculated from the free energy Hessian, the ZA phonon frequencies get basically on top of the harmonic values 
recovering a quadratic dispersion. This means that the physical phonons have a quadratic dispersion for small momenta 
in an unstrained membrane, as it is expected by symmetry. We also show that accounting correctly for the thermal 
expansion is crucial to recover the $\sim q^2$ behavior as shown in Fig. \ref{membrane-results}(b). The membrane 
potential is able to capture the negative thermal expansion at small temperatures, as it is shown in 
Fig. \ref{cte-membrane}. 
\begin{figure}[ht]
\includegraphics[width=0.8\linewidth]{Figures/cte-membrane.eps}
\caption{$\delta a$ as a function of temperature in the membrane model.}
\label{cte-membrane}
\end{figure}
Finally, it is important to remark that a fully rotationally invariant potential is needed to recover the quadratic 
dispersion (see Fig. \ref{membrane-results}(b)). If the last three terms in Eq. \ref{membrane-potential} are 
neglected, which breaks the rotational symmetry of the potential, the quadratic dispersion is not recovered. This 
means that it is very important to keep these terms in the potential to correctly estimate the mechanical properties 
of membranes even if they are usually neglected\cite{mariani2008flexural,amorim2014thermodynamics,de2012bending}. 

\section{Conclusions}

In conclusion, we show that anharmonic effects are crucial to mechanically stabilize graphene and to guarantee its 
phonon modes make physical sense at small momenta. Moreover, we determine that, despite the relevance of anharmonic 
effects, the out-of-plane acoustic modes should show a quadratic dispersion experimentally. We estimate anharmonic 
effects within the self-consistent harmonic approximation both with an atomistic machine learning potential and with 
a membrane model, obtaining consistent results in both cases. Our results show how all the divergences of the 
harmonic and quasiharmonic approximations are cured by anharmonicity. These conclusions can be extrapolated to any 
strictly 2D material and will have a large impact on the understanding if their mechanical and thermal properties.
