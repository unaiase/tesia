% Chapter 1

\chapter{2D materials: Graphene} % Main chapter title

\label{Chapter8} % For referencing the chapter elsewhere, use \ref{Chapter1} 

%----------------------------------------------------------------------------------------

% Define some commands to keep the formatting separated from the content 
%\newcommand{\keyword}[1]{\textbf{#1}}
%\newcommand{\tabhead}[1]{\textbf{#1}}
%\newcommand{\code}[1]{\texttt{#1}}
%\newcommand{\file}[1]{\texttt{\bfseries#1}}
%\newcommand{\option}[1]{\texttt{\itshape#1}}

%----------------------------------------------------------------------------------------

\section{Introduction}

The discovery of graphene\cite{novoselov2004electric,novoselov2005two,novoselov2005twoo,zhang2005experimental} proved 
the existence of 2D materials and launched their science and technology. Graphene is already a reality in different 
industrial products\cite{Kong2019path} that benefit from its fantastic properties. In particular, its mechanical and 
thermal properties are crucial for many of its current and future applications. For instance, uneven strength, 
stiffness, and lightness\cite{lee2008measurement} of graphene have been used to make stronger but lighter macroscopic 
objects, such as tennis rackets, shoes, and so on. Graphene, due to its very high thermal 
conductivity\cite{ghosh2008extremely}, has been already incorporated into electronic devices like mobile phones for 
efficient heat dissipation. \\

The understanding of the mechanical and thermal properties of graphene, and in general of any 2D material, 
is, however, far from trivial. Even the possibility of having crystalline order in 2D has been long 
questioned\cite{landau_statistical_physics,mermin1968crystalline}. Indeed, the root mean square displacement 
calculated within the harmonic approximation diverges in the long wave-length limit, which means that the larger the 
sample of a 2D material the bigger the atomic displacements, preventing any crystalline 
order\cite{landau_statistical_physics}. In Fig. \ref{rms_harmonic} we show the atomic displacements of carbon atoms 
in graphene as a function of the sample size, or the supercell used for the calculation.
\begin{figure}[h]
\includegraphics[width=0.8\linewidth]{Figures/rms-harmonic.eps}
\caption[Root mean square atomic displacements of carbon atoms in graphene calculated using the harmonic 
	dispersion]{Root mean square atomic displacements of carbon atoms in graphene calculated using the harmonic 
	dispersion as a function of the supercell size for different temperatures.}
\label{rms_harmonic}
\end{figure}
As the figure clearly shows, the root mean square atomic displacements diverge at finite temperatures as a function 
of the sample size. Mermin's theorem\cite{mermin1968crystalline} proves that, even without assuming the harmonic 
approximation, long-range crystalline order is not possible in a two-dimensional crystal described by a pair 
potential. Experimentally, however, crystalline order has been observed in suspended 
graphene\cite{meyer2007structure}, although it shows ripples that seem to be intrinsic\cite{fasolino2007intrinsic}. \\

Most of the problems related to the mechanical stability of graphene are caused by the quadratic dispersion of the 
acoustic out-of-plane (ZA) mode that is obtained in the harmonic approximation. The harmonic phonon spectrum of 
graphene is shown in Fig. \ref{harmonic-graphene}.
\begin{figure}[h]
\includegraphics[width=0.55\linewidth]{Figures/harmonic-graphene.eps}
\includegraphics[width=0.45\linewidth]{Figures/harmonic-graphene-logarithmic.eps}
\caption[Graphene harmonic phonons]{Left: Harmonic phonon spectrum of graphene. Right: Harmonic ZA phonon dispersion 
	of graphene in logarithmic scale. We include a quadratic line for reference.}
\label{harmonic-graphene}
\end{figure}
We can see that the LA and TA modes have a linear dispersion close to the point $\Gamma$ as it happens in 3D 
crystals. However, the ZA mode, with atomic displacements along the out-of-plane direction, has a 
quadratic dispersion. This is very clear in the logarithmic scale plot, where we have included a quadratic line for 
reference. The quadratic dispersion is given by the rotational symmetry, together with the fact that in a strictly 
two-dimensional system force constants involving an in-plane and an out-of-plane displacement 
vanish\cite{katsnelson2013graphene}. The mathematical proof of the quadratic dispersion is given in 
appendix \ref{quadratic-dispersion}. \\

The quadratic dispersion also creates spurious divergences in physical properties calculated perturbativaly on top of 
the harmonic result. The phonon linewidths of the in-plane acoustic longitudinal (LA) and transverse (TA) phonons 
calculated perturbatively do not vanish as the momentum decreases\cite{paulatto2013anharmonic}, precisely, because of 
the $\sim q^2$ dispersion of the ZA modes\cite{bonini2012acoustic}. The linewidths of the LA and TA modes are shown 
in Fig. \ref{lata-harmonic-lw} (a). 
\begin{figure}[h]
\includegraphics[width=0.8\linewidth]{Figures/lw-graphene-harmonic.eps}
	\caption[Graphene linewidths calculated within perturbation theory]{(a) Linewidths (full width at half 
	maximum, FWHM) of the LA and TA modes calculated using perturbation theory at 300 K. (b) FWHM divided by the 
	phonon frequency. We shadow the region where $FWHM/\omega>1$. All the calculations are done in the membrane 
	model in Eq. \ref{membrane-potential}.}
\label{lata-harmonic-lw}
\end{figure}
The figure clearly shows how the linewidths do not vanish at small momenta. This yields to the conclusion that for 
small enough momentum phonons do not live long enough for vibrating one period and the quasiparticle picture is lost.
Consequently, the thermal conductivity calculated with the perturbative phonon 
lifetimes\cite{fugallo2013ab,fugallo2014thermal} does not converge with the sample size\cite{bonini2012acoustic}. In 
Fig. \ref{lata-harmonic-lw} (b) we can see the FWHM divided by the phonon frequency. When this number is around 1 the 
phonon picture is lost. Therefore, the longitudinal and transversal acoustic phonons with a frequency smaller than 
$\simeq1$ cm$^{-1}$ do not propagate in graphene. The mentioned region is shadowed in the figure. This means that 
sound does not propagate in graphene because the highest frequency that the human ear can heard is around $10^{-4}$ 
cm$^{-1}$. The application of strain linearizes the dispersion of the ZA phonons and makes the linewidth of the LA 
and TA phonon modes vanish together with their frequency at small momenta, keeping the phonon 
picture\cite{bonini2012acoustic}. \\

It has been argued\cite{wang2016anharmonic,los2009scaling,katsnelson2013graphene,zakharchenko2009finite,mariani2008flexural,amorim2014thermodynamics,de2012bending} that the anharmonic coupling between in-plane and out-of-plane phonon 
modes renormalizes the dispersion of the ZA phonon modes, providing it with a linear term at small momenta that 
somewhat cures the pathologies. This conclusion has been drawn by Monte Carlo simulations with empirical 
potentials\cite{wang2016anharmonic,los2009scaling,katsnelson2013graphene,zakharchenko2009finite} and by using a 
membrane continuum  Hamiltonian that mimics the interaction among acoustic 
modes\cite{mariani2008flexural,amorim2014thermodynamics,de2012bending}. Anharmonic phonons expected experimentally 
should be calculated from the phonon spectral function. For low energy modes, as it happens for the ZA mode, the 
phonon peaks of the spectral function coincide with the phonons defined from the
free energy Hessian, i.e., diagonalizing the  $[\frac{\partial F}{\partial \mathcal{R}_a \partial \mathcal{R}_b}]_{0}/\sqrt{m_am_b}$ matrix, where $F$ is the free energy calculated including anharmonic 
effects and $\boldsymbol{\mathcal{R}}$ the centroid positions that determine the most probable ionic 
positions\cite{bianco2017second}. As shown in section \ref{pertubative-limit-sscha}, the phonons given by the free 
energy Hessian are equivalent to the phonons calculated from the spectral function in the static limit. We will name 
the phonons defined from the phonon spectral function $\Omega_{\mu}(\boldsymbol{q})$ "Physical" phonons and the 
phonons defined from the free energy Hessian "Physical static" phonons. 
If phonons are calculated from the free energy Hessian, as $F$ obeys the same symmetry properties as $U$, a similar 
$\sim q^2$ dispersion would be expected for the ZA mode even if anharmonic effects are included in the calculation 
of $F$. Actually, measurements done with helium diffraction show a ZA mode with a quadratic 
dispersion\cite{al2016acoustic,al2015helium,al2018resolving}, though the linearization regime may not be seen and 
substrate effects may be important. The remaining question is thus, whether the ZA modes really have a quadratic 
dispersion, and, if it is so, how the mechanical stability and thermal properties of graphene can be explained. \\

In this work we show that a quadratic dispersion of the ZA mode is actually expected for graphene provided that it 
is calculated from the phonon spectral function, and that it is compatible with sound propagation. We estimate $F$ 
within the self-consistent harmonic approximation (SCHA). We apply the SCHA in its 
stochastic implementation (SSCHA) making use of a machine learning atomistic potential trained with density 
functional theory\cite{rowe2018development}. We also solve the SCHA equations in a membrane continuum Hamiltonian 
which provides results at smaller momenta.

\section{Crystal structure}

The direct lattice of graphene has a hexagonal structure with two atoms per unit cell. The two vectors of the direct 
lattice in Cartesian coordinates are the following:
\begin{itemize}
 \item $\mathbf{a}_{1}=a(1,0)$,
 \item $\mathbf{a}_{2}=a\left(-\frac{1}{2},\frac{\sqrt{3}}{2}\right)$,
\end{itemize}
where one atom is at the origin and the other in $a(0,1/\sqrt{3})$. Then, the reciprocal lattice is:
\begin{itemize}
 \item $\mathbf{b}_{1}=\frac{2\pi}{a}\left(1,\frac{1}{\sqrt{3}}\right)$,
 \item $\mathbf{b}_{2}=\frac{2\pi}{a}\left(0,\frac{2}{\sqrt{3}}\right)$.
\end{itemize}
The reciprocal lattice of graphene is equivalent to the direct one but rotated by $60$ degrees. The high symmetry 
points in the first Brillouin zone of graphene are $K=\frac{2\pi}{a}\left(\frac{1}{3},\frac{1}{\sqrt{3}}\right)$ and 
$M=\frac{2\pi}{a}\left(0,\frac{1}{\sqrt{3}}\right)$. We can see the Wigner-Seitz cell and the 1BZ of graphene in 
Fig. \ref{wigner}.
\begin{figure}[htb]
\includegraphics[width=0.45\linewidth]{Figures/ws.pdf}
\includegraphics[width=0.45\linewidth]{Figures/brillouin.pdf}
\caption{Wigner-Seitz cell and first Brillouin zone of graphene.}
\label{wigner}
\end{figure}

\section{Empirical potential benchmark and calculation parameters}

For calculating the forces needed in the SSCHA minimization\cite{errea2014anharmonic} we have used an empirical 
potential trained with machine learning and density functional theory (DFT) forces. The details about the machine 
learning training are explained in Ref. \cite{rowe2018development}. Here we have benchmarked the ability of 
the potential to account for the anharmonic effects. For that purpose we have applied the SSCHA method by using 
DFT and empirical forces in a $2\times2$ supercell and we have checked the anharmonic effects in the optical 
modes at the $\Gamma$ point. The machine learning potential is trained with the exchange-correlation in 
Ref. \cite{dion2004van} and for the DFT calculations we have applied a PBE\cite{perdew1996generalized} 
ultrasoft pseudopotential\cite{vanderbilt1990soft} with Van der Walls corrections\cite{barone2009role}. The 
results are shown in Figs. \ref{benchmark-spectrum} and \ref{benchmark}.
\begin{figure}[ht]
\includegraphics[width=0.99\linewidth]{Figures/abinito-vs-empirical.eps}
\caption[Harmonic phonon spectrum of graphene calculated with the empirical                                 
        potential and $ab$ $initio$.]{Harmonic phonon spectrum of graphene calculated with the empirical potential 
	and $ab$ $initio$. The calculations are done in a $6\times6$ supercell.}
\label{benchmark-spectrum}
\end{figure}
\begin{figure}[ht]
\includegraphics[width=0.49\linewidth]{Figures/bm1.eps}
\includegraphics[width=0.49\linewidth]{Figures/bm2.eps}
\caption[Harmonic, SSCHA and physical static frequencies using the DFT and machine learning (ML) forces.]{Harmonic, 
	SSCHA and physical static frequencies using the DFT and machine learning (ML) forces. The left panel shows 
	the in-plane optical frequency at the $\Gamma$ point and the right panel the out-of-plane one.}
\label{benchmark}
\end{figure}
As we can see in Fig. \ref{benchmark-spectrum}, the two potentials provide very similar harmonic phonons, or what is 
the same, very similar forces. Due to the different exchange correlation functional, there is a slight offset in 
Fig. \ref{benchmark}, however, the anharmonic lineshifts in both $\boldsymbol{D}^{(S)}$ and $\boldsymbol{D}^{(F)}$ 
are very well captured within the empirical potential. \\

For the self-consistent DFT calculations we have used a plane wave cutoff of $70$ Ry and a $700$ Ry cutoff for the 
density. For the Brillouin zone integration we have used a Monkhorst pack grid\cite{monkhorst1976special} of 
$32\times32$ points with a Gaussian smearing of $0.02$ Ry. For the linewidth calculations in the main text we have 
used a grid of $400\times400$ grid with a Gaussian smearing of $1$ cm$^{-1}$. For the stress calculation in order 
to account for the thermal expansion we have used a $10\times10$ supercell. We have used the same supercell for 
the SSCHA and physical frequency calculations. For the linewidth calculations we have used 3BFC calculated in 
$3\times3$ supercells. We have tested all the calculations with denser grids and bigger supercells.

\section{Graphene without stress}

In order to calculate phonon spectra in unstrained graphene at any temperature, we calculate the SSCHA stress tensor 
following the procedure in section \ref{scha-stress-section} and pick the lattice parameter that sets it to zero at 
each temperature. For that purpose, we calculate the stress in ranges of $0.0005$ \AA \hspace{0.1cm} and then we 
interpolate the result, which is fitted very accurately with a linear function. The lattice parameter calculated in 
this way includes anharmonic effects as well as the effect of quantum and thermal fluctuations. In order to properly 
account for thermal expansion, all the phonon spectra shown in this work that are obtained with the atomistic 
potential are calculated with the lattice parameter that gives a null SSCHA stress at each temperature. The harmonic 
spectra on the contrary is always calculated at the lattice parameter that minimizes $U$. The temperature dependence 
of the lattice parameter and the thermal expansion coefficient are shown in Fig. \ref{lattice}.
\begin{figure}[ht]
\includegraphics[width=0.99\linewidth]{Figures/lattice.eps}
	\caption[Lattice parameter of graphene as a function of temperature]{(a) Lattice parameter of graphene as a 
	function of temperature obtained with the SSCHA using a machine learning atomistic potential. Both quantum 
	(black) and classical (blue) calculations are included. The classical result is calculated setting $\hbar=0$ 
	in the SSCHA free energy. The temperature-independent frozen nuclei (FN) result corresponds to the lattice 
	parameter that minimizes the Born Oppenheimer potential $U$. Results obtained by Rowe et 
	al.\cite{rowe2018development} are also included for comparison. Solid black and blue lines correspond to 
	cubic fits. The black dashed line corresponds to the quasiharmonic result (QHA). (b) Thermal expansion 
	coefficient (CTE) calculated as $CTE=\frac{1}{A}\frac{\partial A}{\partial T}$, $A$ being the area of the 
	membrane. Red line is directly taken from \cite{rowe2018development}. Black and blue lines are calculated 
	using the cubic fits in (a).}
\label{lattice}
\end{figure}
We include the results of Rowe et al.\cite{rowe2018development}, which do not account for quantum effects as they 
are done with molecular dynamics (MD). They use the same potential as we do. For comparison we also include SSCHA 
calculations in the classical limit and within the quasiharmonic approximation\cite{bonini2007phonon} (QHA). Our 
quantum calculations correctly capture the negative thermal expansion of graphene up to $\sim$ 750 K that has been 
estimated in previous theoretical works\cite{rowe2018development,zakharchenko2009finite}. Our quantum calculations 
show a larger lattice parameter and a more pronounced thermal expansion (in absolute value) at low temperatures. This 
is not surprising as classical calculations neglect quantum fluctuations and, consequently, underestimate the 
fluctuations associated to the high-energy optical modes (the highest energy phonon modes require temperatures of 
around 2000 K to be thermally populated). Actually, the classical result approaches the quantum one at high 
temperatures. This remarks the importance of considering quantum effects in the evaluation of thermodynamic 
properties of graphene. Our classical results and the MD calculations of Rowe et al.\cite{rowe2018development} are 
in agreement (within their error of $0.003$ \AA) at low temperatures. However, at high temperatures, our classical 
results and the ones of Rowe et al.\cite{rowe2018development} deviate. \\

It has been shown that at high temperatures graphene may be in a corrugated 
state\cite{pozzo2011thermal,los2009scaling}. The corrugation is not captured in our formalism as the centroids are 
forced to stay in the plane of the membrane. In order to clarify this disagreement we have performed MD calculation 
with the same potential at 2000 K. We have performed MD calculation in the microcanonical ensemble at fixed lattice 
parameters 2.465 \AA, 2.47 \AA, and 2.48 \AA. The result with $a=2.48$ \AA \hspace{0.1cm} is shown in Fig. \ref{md1}.
\begin{figure}[ht]
\includegraphics[width=0.99\linewidth]{Figures/md1.eps}
\caption[Molecular dynamics run.]{Temperature and in-plane pressure Cartesian components as a function of
the step number. Each steps corresponds to 0.001 ps. The Temperatures is in K and the pressure in Bars. The lattice 
parameter is fixed to 2.48 \AA. The simulation is done in the microcanonical ensemble.}
\label{md1}
\end{figure}
We start the MD simulation from a SSCHA configuration at 4000 K. Due to the equipartition theorem, the system reaches
a temperature of 2000 K and stays stable. We can see in the figure that the temperature and pressure are stable. By 
taking averages, we can calculate the pressure for different lattice parameters and see at what lattice parameter 
the pressure is 0. We show the results in Fig. \ref{md2}. 
\begin{figure}[ht]
\includegraphics[width=0.99\linewidth]{Figures/md2.eps}
\caption[Molecular dynamics results.]{SSCHA and MD in-plane pressure as a function of the lattice parameter at 
2000 K. The results of Rowe et al. is included as a vertical line.}
\label{md2}
\end{figure}
As we can see, our MD results agree with the result of Rowe et al. This suggests that graphene is in a corrugated 
state at high temperatures and the corrugation is not captured within our framework. As we will see later, we see 
the flat phase is metastable and, therefore, we suggest that there may be a first-order phase transition at high 
temperatures from the flat to the corrugated state. \\

In Fig. \ref{lattice} we have also included the quasiharmonic (QHA) result. The lattice parameter within 
the QHA is obtained by calculating the harmonic free energy for different temperatures and lattice parameters. The 
lattice parameter dependence comes from the different harmonic phonons at different lattice parameters. Then, we take 
the lattice parameter that minimizes the harmonic free energy at each temperature. It is worth noting that this 
approximation is not valid to calculate the thermal expansion of graphene due to the imaginary phonon frequencies 
that appear close to $\Gamma$ for the ZA mode already at $500$ K. In the shadowed region in Fig. \ref{lattice} the 
QHA is not valid. These features question the validity of the results obtained with this approach that yield a 
negative thermal expansion at all temperatures\cite{mounet2005first}.

\section{Graphene phonons}

In Fig. \ref{spectrum-graphene} we compare the ZA harmonic phonon spectra with the one obtained from the SSCHA 
dynamical matrix $\boldsymbol{D}^{(S)}$ as well as the spectra obtained from the SSCHA spectral 
function $\Omega_{\mu}(\boldsymbol{q})$. 
\begin{figure*}[ht]
\includegraphics[width=0.32\linewidth]{Figures/T0.eps}
\includegraphics[width=0.32\linewidth]{Figures/T300.eps}
\includegraphics[width=0.32\linewidth]{Figures/T2000.eps}
	\caption[Graphene harmonic and anharmonic ZA phonons]{Harmonic ZA phonon spectra in logarithmic scale 
	together with the SSCHA phonons, $\boldsymbol{D}^{(S)}$, (labeled as ``SSCHA'') and those obtained from the 
	phonon spectral function, $\Omega_{\mu}(\boldsymbol{q})$, (labeled as ``Physical''). Results at 0 K (a), 300 
	K (b), and 2000 K (c) are shown. (d), (e) and (f) show the frequency divided by the squared momentum. The 
	dispersion corresponds to the 	$\Gamma$M direction. For 
	reference, the M point is at $1.4662$ $\AA^{-1}$ at $0$ K, at $1.4671$ $\AA^{-1}$ at $300$ K, and at $1.4652$ 
	$\AA^{-1}$ at $2000$ K. The harmonic result (solid black) is computed at the lattice parameter that minimizes 
	$U$, while the other results include thermal expansion (see Fig. \ref{lattice}). The dashed black lines 
	correspond to harmonic calculations including thermal expansion (TE). All these calculations are performed 
	with the machine learning atomistic potential.}
\label{spectrum-graphene}
\end{figure*}
The main conclusion is that while the dispersion of the ZA modes obtained from $\boldsymbol{D}^{(S)}$ is linearized, 
the physical phonons become close to a quadratic dispersion and approach the harmonic dispersion, as expected by 
symmetry. The SSCHA ZA frequencies suffer a blue-shift with respect to the harmonic ones, but are red-shifted once 
the spectral function is calculated. Both shifts are bigger when the temperature is increased, but the quadratic 
behavior of the ZA modes is always recovered regardless of the temperature. The quadratic dispersion is clear 
from panels (d), (e), and (f), where the frequency is divided by the squared momentum, therefore, the lines with a
quadratic dispersion become flat. We show in Fig. \ref{static-dynamic}
\begin{figure}[ht]
\includegraphics[width=0.32\linewidth]{Figures/T0-za.eps}
\includegraphics[width=0.32\linewidth]{Figures/T300-za.eps}
\includegraphics[width=0.32\linewidth]{Figures/T2000-za.eps}
        \caption[Graphene static and dynamic phonons]{Harmonic, SSCHA, and physical phonons (static and 
	dynamic) calculated at 0 K (a), 300 K (b), and 2000 K (c).}
\label{static-dynamic}
\end{figure}
that the static phonons ($\Omega^{(F)}_{\mu}(\boldsymbol{q})$) agree with those derived from the dynamical (see 
section \ref{dynamical-sscha}) theory ($\Omega_{\mu}(\boldsymbol{q})$) and are thus good representatives of the 
physical phonons expected experimentally. \\

In Fig. \ref{e2g} we make a similar study for the highest energy optical $E2_{g}$ mode at the $\Gamma$ point. This 
mode is interesting because the theoretical results can be compared to Raman experiments\cite{linas2015interplay}. 
\begin{figure}[ht]
\includegraphics[width=0.49\linewidth]{Figures/e2g1.eps}
\includegraphics[width=0.49\linewidth]{Figures/e2g2.eps}
	\caption[E2g mode temperature dependent frequency shift]{(a) SCHA and physical $E2_{g}$ phonon frequency 
	shift as a function of temperature. We include experimental results in graphene\cite{linas2015interplay} and 
	graphite\cite{kagi1994proper,tan1999intrinsic}. (b) The same as (a) without experiments and including the 
	physical phonons without thermal expansion (Physical NTE) and the physical static phonons.}
\label{e2g}
\end{figure}
By looking at Fig. \ref{e2g} (a), we observe that both SCHA and physical frequencies red-shift with increasing 
temperature in agreement with experiments and theory\cite{calizo2007temperature,bonini2007phonon}. Actually, for 
this mode the difference between the SCHA and physical frequencies is not so crucial. By looking at Fig. \ref{e2g} 
(b) we can see how big are the dynamical effects and the effect of the thermal expansion. For this high energy mode 
it is crucial to include the dynamical effects, otherwise the shift is overestimated. The thermal expansion provides 
a positive shift at low temperatures. 

\section{Root mean square displacements and linewidths}

Even if the anharmonic correction to the phonon spectra may look small in Fig. \ref{spectrum-graphene}, it has a 
huge impact on the mechanical and thermal properties of graphene. As shown in Fig. \ref{problems} (a), 
\begin{figure*}[ht]
\includegraphics[width=0.49\linewidth]{Figures/rms.eps}
\includegraphics[width=0.49\linewidth]{Figures/lw-graphene.eps}
\caption[SSCHA root mean square atomic displacements and linewidths]{(a) Root mean square atomic displacement of 
	carbon atoms in graphene calculated using the harmonic and the SSCHA density matrices. The harmonic result 
	is always computed at the lattice parameter that minimizes $U$, while the SSCHA results include thermal 
	expansion (see Fig. \ref{lattice}). (b) Linewidths (full width at half maximum) of LA and TA phonon modes 
	at $300$ K calculated within perturbation theory and the SSCHA. Results in both figures are obtained with 
	the machine learning atomistic potential.}
\label{problems}
\end{figure*}
when calculating the root mean square atomic displacement with the SSCHA density matrix as 
$\sqrt{\langle \mathbf{u}^2 \rangle}_{\rho_{\mathcal{H}}}$, the dramatic divergences with the sample size obtained 
in the harmonic case are partially suppressed, clearly showing the contribution of anharmonicity to the crystalline 
order of graphene. The divergences are reduced precisely because the SSCHA phonon frequencies obtained from 
$\boldsymbol{D}^{(S)}$ that build $\rho_{\mathcal{H}}$ are linear at small momenta (see 
Fig. \ref{spectrum-graphene}). Thus, even if the phonons obtained from the SSCHA free energy Hessian 
$\boldsymbol{D}^{(F)}$ are quadratic, the fact that the density matrix used to compute thermodynamic properties is 
built with linearized phonons is responsible for reducing the divergences. \\

The SSCHA phonon frequencies obtained from $\boldsymbol{D}^{(S)}$ also provide an appropriate basis to calculate 
phonon linewidths\cite{bianco2017second,aseginolaza2019phonon}. The results presented in Fig. \ref{problems} (b) 
clearly show that the linearization of the SSCHA frequencies dramatically changes the linewidth of the LA and TA 
modes at small momenta by making them smaller as momentum decreases. This result recovers the quasiparticle picture 
for these modes. Similar results are obtained in Ref. \cite{bonini2012acoustic}. However, in this work linewidths of 
LA and TA modes at small momenta only vanish when strain is applied. Somewhat strain has an analogous effect to 
anharmonicity by linearizing the ZA dispersion. The problem is that strain linearizes physical phonons. We show here 
that there is no need of strain to have physically well-defined phonon linewidths provided by 
linear SCHA phonons together with quadratic physical phonons. Anharmonicity is responsible for it. The sound 
propagation in graphene and the ratio FWHM$/\omega$ will be discussed in the next section together with the results 
in the membrane model.

\section{SCHA applied to a continuum membrane Hamiltonian}

In order to obtain results at very small momenta and reinforce the conclusions drawn with the atomistic calculations 
with the machine learning potential, we also solve the SCHA equations in a continuum membrane Hamiltonian. This model 
has been widely used in the literature to describe graphene as an elastic membrane as well as to account for the 
coupling between in-plane and out-of plane acoustic 
modes\cite{mariani2008flexural,amorim2014thermodynamics,de2012bending}. The most general rotationally invariant 
continuum model potential for phonons in free-standing 2D membranes up to the fourth-order with respect to the 
phonon fields has the following form:
\begin{multline}
 \label{membrane-potential}
 U=\frac{1}{2}\int_{\Omega}d^{2}x[\kappa(\partial^{2}h)^{2}+C^{ijkl}\partial_{i}u_{j}\partial_{k}u_{l}+C^{ijkl}\partial_{i}u_{j}\partial_{k}h\partial_{l}h+\\ +\frac{C^{ijkl}}{4}\partial_{i}h\partial_{j}h\partial_{
 k}h\partial_{l}h+\frac{C^{ijkl}}{2}\partial_{i}\boldsymbol{u}\cdot\partial_{j}\boldsymbol{u}\partial_{k}h\partial_{l}h+\\+C^{ijkl}\partial_{i}u_{j}\partial_{k}\boldsymbol{u}\cdot\partial_{l}\boldsymbol{u}+
 \frac{C^{ijkl}}{4}\partial_{i}\boldsymbol{u}\cdot\partial_{j}\boldsymbol{u}\partial_{k}\boldsymbol{u}\cdot\partial_{l}\boldsymbol{u}].
\end{multline}
$\boldsymbol{u}(\boldsymbol{x})$ and $h(\boldsymbol{x})$ are the in-plane and out-of-plane displacement 
fields, respectively, and $\boldsymbol{x}$ is the 2D position vector in the membrane. $\kappa$ is the bending 
rigidity of the membrane and the tensor 
$C^{ijkl}=\lambda\delta^{ij}\delta^{kl}+\mu(\delta^{ik}\delta^{jl}+\delta^{il}\delta^{jk})$ contains the Lame's 
coefficients, $\lambda$ and $\mu$, and Kronecker deltas. As it is a continuum model it only accounts for acoustic 
modes and the second-order terms in the potential are the harmonic terms. The harmonic frequencies in normal 
coordinates space are $\omega_{ZA}(q)=\sqrt{\phi_{ZA}(\boldsymbol{q})/\rho}$, 
$\omega_{LA}(q)=\sqrt{(\phi_{LA}(\boldsymbol{q})/\rho}$, and 
$\omega_{TA}(q)=\sqrt{\phi_{TA}(\boldsymbol{q})/\rho}$, where $\rho$ is the mass density of the membrane and 
$\phi_{ZA}(\boldsymbol{q})=\kappa |\boldsymbol{q}|^{4}$, 
$\phi_{LA}(\boldsymbol{q})=(\lambda+\mu)|\boldsymbol{q}|^{2}$, and
$\phi_{TA}(\boldsymbol{q})=\mu|\boldsymbol{q}|^{2}$ are the harmonic 2BFC. LA and TA correspond to the in-plane 
longitudinal and transversal phonons and ZA corresponds to the out-of-plane phonons. The higher-order terms account 
for the phonon-phonon anharmonic interactions. The thermal expansion is included in this formalism by changing the 
in-plane derivatives $\partial_{i}u_{j}\rightarrow \partial_{i}u_{j}+\delta^{ij}\delta a$, with 
$\delta a=(a-a_{0})a_{0}$, $a_{0}$ being the lattice parameter that minimizes $U$. \\

It is not possible to apply the SCHA analytically in Eq. \ref{membrane-potential}. As far as we know, the simplest 
approximation that allows applying the SCHA analytically is to neglect fourth-order terms including in-plane phonon 
fields. In that case, the potential can be written as
\begin{multline}
\label{membrane-potential-tommaso}
U=\frac{1}{2}\int_{\Omega}d^{2}x[C^{ijkl}\partial_{i}u_{j}\partial_{k}u_{l}+\kappa(\partial^{2}h)^{2}+
C^{ijkl}\partial_{i}u_{j}\partial_{k}h\partial_{l}h+\frac{C^{ijkl}}{4}\partial_{i}h\partial_{j}h\partial_{k}h\partial_{l}h]+\\+2\Omega(\lambda+\mu)(\delta a)^{2}+\delta a(\lambda+\mu)\int_{\Omega}d^{2}x\partial_{k}h\partial_{k}h.
\end{multline}
By Fourier transforming this potential in $\boldsymbol{q}$ space and applying Eq. \ref{scha-equation}, we arrive to 
the SCHA equations (see appendix \ref{appendixA} for further derivations)
\begin{equation}
\label{tommaso-eq1}
\frac{\partial\mathcal{F}}{\partial\delta a}=0\rightarrow\delta a=-\frac{1}{4\Omega}\sum_{\boldsymbol{q}}|\boldsymbol{q}|^{2}g[\Omega_{ZA}^{(S)}(\boldsymbol{q})],
\end{equation}   
\begin{equation}
\label{tommaso-eq2}
\frac{\partial\mathcal{F}}{\partial\Phi_{ZA}(\boldsymbol{q})}=0\rightarrow\Phi_{ZA}(\boldsymbol{q})=\kappa|\boldsymbol{q}|^{4}+2\delta a(\lambda+\mu)|\boldsymbol{q}|^{2}+\frac{\lambda+2\mu}{2\Omega}\sum_{\boldsymbol{k}}g[\Omega_{ZA}^{(S)}(\boldsymbol{k})][|\boldsymbol{q}|^{2}|\boldsymbol{k}|^{2}+2(\boldsymbol{q}\cdot\boldsymbol{k})^{2}].
\end{equation}
$\Phi_{ZA}(\boldsymbol{q})$ are the SCHA 2BFC that correspond to the out-of-plane modes and the frequencies are 
$\Omega_{ZA}^{(S)}(\boldsymbol{q})=\sqrt{\Phi_{ZA}(\boldsymbol{q})/\rho}$. The function $g$ is defined as
$g(x)=coth(x/2T)/2\rho x$, $T$ being the temperature. Already from Eq. \ref{tommaso-eq2} it can be seen that the 
non-zero $\delta a$ provides the SCHA frequency with a linear term. By inserting Eq. \ref{tommaso-eq1} in 
Eq. \ref{tommaso-eq2} the SCHA 2BFC can be written as 
$\Phi_{ZA}(\boldsymbol{q})=\kappa|\boldsymbol{q}|^{4}+\gamma|\boldsymbol{q}|^{2}$. The expression for $\gamma$ is 
given in appendix \ref{appendixA}. Therefore, we have shown that the SCHA flexural acoustic (ZA) modes have a linear 
dispersion close to the point $\Gamma$. \\

For calculating the physical static phonons, given by the second derivative of the free energy, an analogous formula 
to Eq. \ref{free-energy-hessian} can be found within the membrane formalism (see appendix \ref{appendixA}). By 
applying this formula we arrive to the following expression for the physical 2BFC at $T=0$
\begin{equation}
 \Phi_{ZA}^{(F)}(\boldsymbol{q})=\kappa|\boldsymbol{q}|^{2}+(\gamma-\sigma)|\boldsymbol{q}|^{2}+
 O(|\boldsymbol{q}|^{4}).
\end{equation} 
The formula of $\sigma$ is given in the appendix. The positive number $\sigma$ makes the linear term in the physical 
phonon frequencies $40\%$ smaller than in the SCHA case, however, it does not remove it. Actually, it can be argued 
that this non-zero linear term in the physical static frequencies arises because the potential in 
Eq. \ref{membrane-potential-tommaso} is not rotationally invariant. In order to clarify these results we have 
applied the SCHA numerically in the full potential. \\

By taking the full potential in Eq. \ref{membrane-potential}, Fourier transforming it and applying 
Eq. \ref{scha-equation} we arrive to the SCHA equations for the rotationally invariant membrane
\begin{multline}
 \frac{\partial\mathcal{F}(\mathcal{V})}{\partial\delta a}=0=2\Omega(2\delta a+3\delta a^{2}+\delta a^{3})(\lambda+
\mu)+\frac{1}{2}\sum_{\boldsymbol{q}}g[\Omega_{ZA}^{(S)}(\boldsymbol{q})]2(1+\delta a)(\lambda+\mu)|\boldsymbol{
 q}|^{2}\\+\frac{1}{2}\sum_{\boldsymbol{q}}g[\Omega_{LA}^{(S)}(\boldsymbol{q})][2(1+\delta a)(\lambda+2\mu)|\boldsymbol{q}|^{2}+2(1+\delta a)(\lambda+\mu)|\boldsymbol{q}|^{2}]+\\\frac{1}{2}\sum_{\boldsymbol{q}}g[\Omega_{
 TA}^{(S)}(\boldsymbol{q})][2(1+\delta a)\mu|\boldsymbol{q}|^{2}+2(1+\delta a/2)(\lambda+\mu)|\boldsymbol{q}|^{2}],
\end{multline}
\begin{multline}
 \Phi_{ZA}(\boldsymbol{q})=\kappa|\boldsymbol{q}|^{4}+2(1+\delta a/2)\delta a(\lambda+\mu)|\boldsymbol{q}|^{2}+\frac{\lambda+2\mu}{2\Omega}\sum_{\boldsymbol{k}}g[\Omega_{ZA}^{(S)}(\boldsymbol{k})][|\boldsymbol{q}|^{2}|\boldsymbol{
 k}|^{2}+2(\boldsymbol{q}\cdot\boldsymbol{k})^{2}]+\\\frac{1}{2\Omega}\sum_{\boldsymbol{k}}\{g[\Omega_{LA}^{(S)}(\boldsymbol{k})]+g[\Omega_{TA}^{(S)}(\boldsymbol{k})]\}[\lambda|\boldsymbol{q}|^{2}|\boldsymbol{k}|^{2}+2\mu(
 \boldsymbol{q}\cdot\boldsymbol{k})^{2}],
\end{multline}
\begin{multline}
 \Phi_{LA}(\boldsymbol{q})=(\lambda+2\mu)|\boldsymbol{q}|^{2}+2(1+\delta a/2)\delta a(\lambda+2\mu)|\boldsymbol{q}|^{2}+2(1+\delta a/2)\delta a(\lambda+\mu)|\boldsymbol{q}|^{2}\\+\frac{1}{2\Omega}\sum_{\boldsymbol{k}}g[\Omega_{ZA}^{
 (S)}(\boldsymbol{k})][\lambda|\boldsymbol{q}|^{2}|\boldsymbol{k}|^{2}+2\mu(\boldsymbol{q}\cdot\boldsymbol{k})^{2}]+\\\frac{1}{4\Omega}\sum_{\boldsymbol{k}}\{4g[\Omega_{TA}^{(S)}(\boldsymbol{k})][\lambda(\boldsymbol{q}\cdot\boldsymbol{
 k})^{2}+\mu|\boldsymbol{q}|^{2}|\boldsymbol{k}|^{2}+\mu(\boldsymbol{q}\cdot\boldsymbol{k})^{2}](\hat{\boldsymbol{q}_{\perp}}\cdot\hat{\boldsymbol{k}})+ \\ 2g[\Omega_{TA}^{(S)}(\boldsymbol{k})][\lambda|\boldsymbol{q}|^{2}|\boldsymbol{
 k}|^{2}+2\mu(\boldsymbol{q}\cdot\boldsymbol{k})^{2}]+ \\ 2g[\Omega_{LA}^{(S)}(\boldsymbol{k})][\lambda|\boldsymbol{q}|^{2}|\boldsymbol{k}|^{2}+2\mu(\boldsymbol{q}\cdot\boldsymbol{k})^{2}]+ \\ 4g[\Omega_{LA}^{(S)}(\boldsymbol{
 k})][\lambda(\boldsymbol{q}\cdot\boldsymbol{k})^{2}+\mu|\boldsymbol{q}|^{2}|\boldsymbol{k}|^{2}+\mu(\boldsymbol{q}\cdot\boldsymbol{k})^{2}](\hat{\boldsymbol{q}}\cdot\hat{\boldsymbol{k}})\}
\end{multline}
and,
\begin{multline}
 \Phi_{TA}(\boldsymbol{q})=\mu|\boldsymbol{q}|^{2}+2(1+\delta a/2)\delta a\mu|\boldsymbol{q}|^{2}+2(1+\delta a/2)\delta a(\lambda+\mu)|\boldsymbol{q}|^{2}\\+\frac{1}{2\Omega}\sum_{\boldsymbol{k}}g[\Omega_{ZA}^{
 (S)}(\boldsymbol{k})][\lambda|\boldsymbol{q}|^{2}|\boldsymbol{k}|^{2}+2\mu(\boldsymbol{q}\cdot\boldsymbol{k})^{2}]+\\\frac{1}{4\Omega}\sum_{\boldsymbol{k}}\{
 4g[\Omega_{TA}^{(S)}(\boldsymbol{k})][\lambda(\boldsymbol{q}\cdot\boldsymbol{k})^{2}+\mu|\boldsymbol{q}|^{2}|\boldsymbol{k}|^{2}+\mu(\boldsymbol{q}\cdot\boldsymbol{k})^{2}](\hat{\boldsymbol{q}_{\perp}}\cdot\hat{
 \boldsymbol{k}_{\perp}})]+\\4g[\Omega_{LA}^{(S)}(\boldsymbol{k})][\lambda(\boldsymbol{q}\cdot\boldsymbol{k})^{2}+\mu|\boldsymbol{q}|^{2}|\boldsymbol{k}|^{2}+\mu(\boldsymbol{q}\cdot\boldsymbol{k})^{2}](\hat{\boldsymbol{
 q}_{\perp}}\cdot\hat{\boldsymbol{k}})+ \\ 2g[\Omega_{TA}^{(S)}(\boldsymbol{k})][\lambda|\boldsymbol{q}|^{2}|\boldsymbol{k}|^{2}+2\mu(\boldsymbol{q}\cdot\boldsymbol{k})^{2}]  \}.
\end{multline}
We have solved these equation by using the Newton-Raphson method\cite{ypma1995historical} in a circular discretized grid of $60\times60$ $\boldsymbol{q}$ 
points. The results are shown in Fig. \ref{membrane-results}. 
\begin{figure}[h]
\includegraphics[width=\linewidth]{Figures/membrane.eps}
	\caption[ZA harmonic and anharmonic phonons in the membrane model.]{(a) Frequency of the ZA mode divided by 
	the squared momentum in the harmonic approximation, within the SCHA and obtained from the Hessian of the SCHA
	(labeled as ``Physical static'') at 0 K in the membrane model. We name rotationally invariant (RI) the 
	results considering the full potential in Eq. \ref{membrane-potential}. We name no rotationally invariant (No 
	RI) the results neglecting the last three terms in Eq. \ref{membrane-potential}. (b) Frequency of the ZA mode 
	in the harmonic approximation, within the SCHA and obtained from the Hessian of the SCHA at $0$ K and $300$ K
	in the membrane model. We also include the SCHA and physical phonons at $300$ K without considering thermal 
	expansion (NTE).}
\label{membrane-results}
\end{figure}
All conclusions drawn with the atomistic model are confirmed and put in solid grounds. Again the ZA phonons obtained 
from the SCHA force constants get linearized at small momenta and the linearization is bigger for higher 
temperatures. These results are consistent with the anharmonic linearization obtained for this mode in previous 
calculations\cite{mariani2008flexural,amorim2014thermodynamics,de2012bending}. However, when the phonons are 
calculated from the free energy Hessian, the ZA phonon frequencies get basically on top of the harmonic values 
recovering a quadratic dispersion. This means that the physical phonons have a quadratic dispersion for small momenta 
in an unstrained membrane, as it is expected by symmetry. We also show that accounting correctly for the thermal 
expansion is crucial to recover the $\sim q^2$ behavior as shown in Fig. \ref{membrane-results} (b). The membrane 
potential is able to capture the negative thermal expansion at small temperatures, as it is shown in 
Fig. \ref{cte-membrane}. 
\begin{figure}[ht]
\includegraphics[width=0.8\linewidth]{Figures/cte-membrane.eps}
\caption{$\delta a$ as a function of temperature in the membrane model.}
\label{cte-membrane}
\end{figure}
Finally, it is important to remark that a fully rotationally invariant potential is needed to recover the quadratic 
dispersion (see Fig. \ref{membrane-results} (b)). If the last three terms in Eq. \ref{membrane-potential} are 
neglected, which breaks the rotational symmetry of the potential, the quadratic dispersion is not recovered. This 
means that it is very important to keep these terms in the potential to correctly estimate the mechanical properties 
of membranes even if they are usually neglected\cite{mariani2008flexural,amorim2014thermodynamics,de2012bending}. \\

In Fig. \ref{lw-membrane} we show the linewidths of the LA and TA modes calculated within the membrane model. We 
also include the atomistic results for comparison.
\begin{figure}[ht]
\includegraphics[width=0.8\linewidth]{Figures/lwmembrane.eps}
	\caption[Atomistic and membrane linewidths calculated within the SSCHA and within perturbation theory in 
	graphene]{(a) Atomistic and membrane linewidths calculated within the SSCHA and within perturbation theory 
	at 300 K. (b) Membrane linewidths at 300 K divided by the phonon frequency calculated within the SCHA and 
	perturbation theory.}
\label{lw-membrane}
\end{figure}
The results clearly show that the linearization of the SCHA frequencies dramatically changes the linewidth of the LA 
and TA modes at small momenta by making them smaller as momentum decreases. This result is equivalent in the 
atomistic case. When the ratio between the FWHM and the frequency of the mode is 1 or bigger, the quasiparticle 
picture is lost. According to our calculations in Fig. \ref{lw-membrane} (b) this value is reached in the 0.001-0.002 
\AA$^{-1}$ momentum range when the linewidth is calculated on top of the harmonic dispersion within perturbation 
theory. We shadow the region where the membrane LA mode is not well defined. However, when the linewidth is 
calculated within the SCHA, the ratio never gets bigger than 0.05. These results recover the quasiparticle picture 
for these modes and are in agreement with experiments\cite{wang2008brillouin}, where they clearly see that in this 
momentum range (0.001-0.002\AA$^{-1}$) the quasiparticle picture is hold. It also means that sound can propagate in 
graphene, which does not happen within perturbation theory. \\

In the last part of this section we will prove that neglecting the fourth-order tensor in the SSCHA self-energy and 
assuming the Lorentzian approximation are good approximations. For studying the effect of the fourth-order tensor 
in the SSCHA self-energy we Taylor expand Eq. \ref{full-sscha-se}
\begin{multline}
 \boldsymbol{\Pi}(z)=\mathbf{M}^{-\frac{1}{2}}\overset{(3)}{\mathbf{\Phi}}\mathbf{\Lambda}(z)[\mathbf{1}-\overset{(4)}{\mathbf{\Phi}}\mathbf{\Lambda}(z)]^{-1}
 \overset{(3)}{\mathbf{\Phi}}\mathbf{M}^{-\frac{1}{2}}\simeq\\\mathbf{M}^{-\frac{1}{2}}\overset{(3)}{\mathbf{\Phi}}\mathbf{\Lambda}(z)\overset{(3)}{\mathbf{\Phi}}\mathbf{M}^{-\frac{1}{2}}+\mathbf{M}^{-\frac{1}{2}}\overset{(3)}{\mathbf{\Phi}}\mathbf{\Lambda}(z)\overset{(4)}{\mathbf{\Phi}}\mathbf{\Lambda}(z)\overset{(3)}{\mathbf{\Phi}}\mathbf{M}^{-\frac{1}{2}}+\dots,
\end{multline}
and we calculate the contribution of the term containing the fourth-order tensor to the linewidth. We also calculate 
the spectral function with and without including the frequency dependence of the self energy. We show the results in 
Fig. \ref{lw_membrane}.
\begin{figure}[ht]
\includegraphics[width=0.49\linewidth]{Figures/v4-membrane.eps}
\includegraphics[width=0.45\linewidth]{Figures/spf-lorentzian-membrane.eps}
	\caption[Fourth-order term contribution to the linewidth in graphene. Spectral function within the 
	Lorentzian and non-Lorentzian approximations]{(a) Linewidth (full width at half maximum, FWHM) contribution 
	of the term containing the fourth-order tensor of the LA mode calculated in the membrane model at 100 K using 
	the harmonic and SCHA phonons. The value of the smearing is in the legend. (b) Spectral function of the LA 
	mode with momentum 0.01 \AA$^{-1}$ at 100 K with and without considering the frequency dependence of the 
	self energy.}
\label{lw_membrane}
\end{figure}
The figure clearly shows that the contribution of the fourth-order tensor is at least one order of magnitude smaller 
than the main term and, what it is more important, it also decays as momentum decreases. The figure also shows that 
the Lorentzian approximation is justified for the acoustic modes in graphene.

\section{Conclusions}

In conclusion, we show that anharmonic effects are crucial to mechanically stabilize graphene and to guarantee its 
phonon modes make physical sense and propagate sound at small momenta. Moreover, we determine that, despite the 
relevance of anharmonic effects, the out-of-plane acoustic modes should show a quadratic dispersion experimentally. 
We estimate anharmonic effects within the self-consistent harmonic approximation both with an atomistic machine 
learning potential and with a membrane model, obtaining consistent results in both cases. Our results show how all 
the divergences in the atomic displacements reduce and sound can propagate in graphene. These conclusions can 
be extrapolated to any strictly 2D material and will have a large impact on the understanding if their mechanical 
and thermal properties.
