% Chapter 1

\chapter{2D materials: Graphene} % Main chapter title

\label{Chapter8} % For referencing the chapter elsewhere, use \ref{Chapter1} 

%----------------------------------------------------------------------------------------

% Define some commands to keep the formatting separated from the content 
%\newcommand{\keyword}[1]{\textbf{#1}}
%\newcommand{\tabhead}[1]{\textbf{#1}}
%\newcommand{\code}[1]{\texttt{#1}}
%\newcommand{\file}[1]{\texttt{\bfseries#1}}
%\newcommand{\option}[1]{\texttt{\itshape#1}}

%----------------------------------------------------------------------------------------

\section{Introduction}


\section{Quadratic dispersion of ZA modes within the harmonic approximation}

In this section we prove that the dispersion of the ZA phonon modes close to the point $\Gamma$ is quadratic in the 
harmonic approximation. \\

For that purpose, by using the inversion symmetry and the ASR in Eq. \ref{acoustic-sum-rule} we will rewrite Eq. \ref{2bfc-fourier-simple} in section \ref{harmonic-approximation} as
\begin{multline}
 \phi_{s_{1}s_{2}}^{\alpha_{1}\alpha_{2}}(\mathbf{q})=\sum_{\mathbf{T}}\phi_{s_{1}s_{2}}^{\alpha_{1}\alpha_{2}}(\mathbf{T},0)e^{-i\mathbf{q}\cdot\mathbf{T}}=\\\frac{1}{2}\sum_{\mathbf{T}}\phi_{s_{1}s_{2}}^{\alpha_{1}\alpha_{2}}(\mathbf{T},0)(e^{-i\mathbf{q}\cdot\mathbf{T}}+e^{i\mathbf{q}\cdot\mathbf{T}}-2)=-2\sum_{\mathbf{T}}\phi_{s_{1}s_{2}}^{\alpha_{1}\alpha_{2}}(\mathbf{T},0)sin^{2}(\mathbf{q}\cdot\mathbf{T}/2).
\end{multline}
And by Taylor expanding this equation close to the point $\Gamma$ we get
\begin{equation}
\label{need-zero}
\phi_{s_{1}s_{2}}^{\alpha_{1}\alpha_{2}}(\mathbf{q})\simeq -\frac{1}{2}\sum_{\alpha\beta}q_{\alpha}q_{\beta}\sum_{\mathbf{T}}\phi_{s_{1}s_{2}}^{\alpha_{1}\alpha_{2}}(\mathbf{T})T_{\alpha}T_{\beta}+O(q^{4}),
\end{equation} 
$\alpha$, $\beta$ being $x$, $y$, and $z$ Cartesian components. \\

The fact that a material is bidimensional and the equilibrium atomic positions are contained in the $z=0$ plane,
makes that all the 2BFC of the type $\phi_{s_{1}s_{2}}^{xz}(\mathbf{T})$, $\phi_{s_{1}s_{2}}^{yz}(\mathbf{T})$ are zero. This is translated into the fact that the out-of-plane $z$ direction is an eigendirection of the system and the 
harmonic frequency of the ZA mode is given by
\begin{equation} 
\omega_{ZA}^{2}(\mathbf{q})=D_{11}^{zz}(\mathbf{q})+D_{12}^{zz}(\mathbf{q}). 
\end{equation} 
Therefore, by looking at Eq. \ref{need-zero} we can see that for proving that the dispersion of the ZA modes is quadratic we need to prove that 
$\sum_{\mathbf{T}}\phi_{s_{1}s_{2}}^{zz}(\mathbf{T})T_{\alpha}T_{\beta}=0$. Actually, this equality is enforced by the rotational invariance of the system. \\

In conclusion, the 2D character of the 2BFC and the rotational invariance of the system make the harmonic dispersion of the ZA modes close to the point $\Gamma$ to be quadratic.

\section{SCHA applied to a continuum membrane Hamiltonian}

In this section we apply the SCHA in a continuum membrane Hamiltonian in order to get some results at very small 
momenta and reinforce our conclusions from the atomistic model. This model has been very much used in the 
literature to acount for the coupling between in-plane and out-of plane acoustic modes of 
graphene\cite{mariani2008flexural,amorim2014thermodynamics,de2012bending}. The most general rotationally invariant 
continuum model potential for phonons in free-standing 2D membranes up to the fourth-order with respect to the 
phonon fields has the following form:
\begin{multline}
 \label{membrane-potential}
 V=\frac{1}{2}\int_{\Omega}d^{2}x[\kappa(\partial^{2}h)^{2}+C^{ijkl}\partial_{i}u_{j}\partial_{k}u_{l}+C^{ijkl}\partial_{i}u_{j}\partial_{k}h\partial_{l}h+\\ +\frac{C^{ijkl}}{4}\partial_{i}h\partial_{j}h\partial_{
 k}h\partial_{l}h+\frac{C^{ijkl}}{2}\partial_{i}\boldsymbol{u}\cdot\partial_{j}\boldsymbol{u}\partial_{k}h\partial_{l}h+\\+C^{ijkl}\partial_{i}u_{j}\partial_{k}\boldsymbol{u}\cdot\partial_{l}\boldsymbol{u}+
 \frac{C^{ijkl}}{4}\partial_{i}\boldsymbol{u}\cdot\partial_{j}\boldsymbol{u}\partial_{k}\boldsymbol{u}\cdot\partial_{l}\boldsymbol{u}].
\end{multline}
$\boldsymbol{u}(\boldsymbol{x})$ and $h(\boldsymbol{x})$ are the in-plane and out-of-plane displacement 
fields, respectively, and $\boldsymbol{x}$ is the 2D position vector in the membrane. $\kappa$ is the bending 
rigidity of the membrane and the tensor 
$C^{ijkl}=\lambda\delta^{ij}\delta^{kl}+\mu(\delta^{ik}\delta^{jl}+\delta^{il}\delta^{jk})$ contains the Lame 
coefficients $\lambda$ and $\mu$ and Kronecker deltas. As it is a continuum model it only accounts for acoustic 
modes and the second-order terms in the potential are the harmonic terms. The harmonic frequencies are 
$\omega_{ZA}(q)=\sqrt{\kappa/\rho}q^{2}$, $\omega_{LA}(q)=\sqrt{(\lambda+\mu)/\rho}q$, and 
$\omega_{TA}(q)=\sqrt{\mu\rho} q$, $\rho$ being the mass density of the membrane. The higher-order terms account 
for the phonon-phonon anharmonic interactions. The thermal expansion is included in this formalism by changing the 
in-plane derivatives $\partial_{i}u_{j}\rightarrow \partial_{i}u_{j}+\delta^{ij}\delta a$, with 
$\delta a=(a-a_{0})a_{0}$, $a_{0}$ being the lattice parameter that minimizes $V$. \\

It is not possible to apply the SCHA analytically in Eq. \ref{membrane-potential}. The simplest approximation that 
allows applying the SCHA analytically is to neglect fourth-order terms including in-plane phonon fields. In that 
case the potential can be written as
\begin{multline}
\label{membrane-potential-tommaso}
V=\frac{1}{2}\int_{\Omega}d^{2}x[C^{ijkl}\partial_{i}u_{j}\partial_{k}u_{l}+\kappa(\partial^{2}h)^{2}+
C^{ijkl}\partial_{i}u_{j}\partial_{k}h\partial_{l}h+\frac{C^{ijkl}}{4}\partial_{i}h\partial_{j}h\partial_{k}h\partial_{l}h]+\\+2\Omega(\lambda+\mu)(\delta a)^{2}+\delta a(\lambda+\mu)\int_{\Omega}d^{2}x\partial_{k}h\partial_{k}h.
\end{multline}
By Fourier transforming this potential in $\boldsymbol{q}$ space and applying Eq. \ref{scha-equation}, we arrive to 
the SCHA equations (see appendix for further derivations)
\begin{equation}
\label{tommaso-eq1}
\frac{\partial\mathcal{F}}{\partial\delta a}=0\rightarrow\delta a=-\frac{1}{4\Omega}\sum_{\boldsymbol{q}}|\boldsymbol{q}|^{2}g[\Omega_{SCHA}^{(h)}(\boldsymbol{q})],
\end{equation}   
\begin{equation}
\label{tommaso-eq2}
\frac{\partial\mathcal{F}}{\partial\Phi_{h}(\boldsymbol{q})}=0\rightarrow\Phi_{h}(\boldsymbol{q})=\kappa|\boldsymbol{q}|^{4}+2\delta a(\lambda+\mu)|\boldsymbol{q}|^{2}+\frac{\lambda+2\mu}{2\Omega}\sum_{\boldsymbol{k}}g[\Omega_{SCHA}^{(h)}(\boldsymbol{k})][|\boldsymbol{q}|^{2}|\boldsymbol{k}|^{2}+2(\boldsymbol{q}\cdot\boldsymbol{k})^{2}].
\end{equation}
$\Phi_{h}(\boldsymbol{q})$ are the SCHA 2BFC that correspond to the out-of-plane modes and the frequencies are 
$\Omega_{SCHA}^{(h)}(\boldsymbol{q})=\sqrt{\Phi_{h}(\boldsymbol{q})/\rho}$. The function $g$ is defined as
$g(x)=coth(x/2T)/2\rho x$, $T$ being the temperature. Already from Eq. \ref{tommaso-eq2} it can be seen that the 
non-zero $\delta a$ provides the SCHA frequency with a linear term. By inserting Eq. \ref{tommaso-eq1} in 
Eq. \ref{tommaso-eq2} the SCHA 2BFC can be written as 
$\Phi_{h}(\boldsymbol{q})=\kappa|\boldsymbol{q}|^{4}+\gamma|\boldsymbol{q}|^{2}$. The expression for $\gamma$ is given in appendix. Therefore, we have shown that the SCHA flexural acoustic modes have a linear dispersion close to the points $\Gamma$. \\

For calculating the physical phonons, given by the second derivative of the free energy, an analogous formula to 
Eq. \ref{free-energy-hessian} can be found within the membrane formalism (see appendix). By applying this formula 
we arrive to the following expression for the physical phonons at $T=0$
\begin{equation}
 \Phi_{F}^{(h)}(\boldsymbol{q})=\kappa|\boldsymbol{q}|^{2}+(\gamma-\sigma)|\boldsymbol{q}|^{2}+
 O(|\boldsymbol{q}|^{4}).
\end{equation} 
The formula of $\sigma$ is given in the appendix. The positive number $\sigma$ makes the linear term in the physical 
phonon frequencies $40\%$ smaller than in the SCHA case, however, it does not remove it. Actually, it can be argued 
that this non-zero linear term in the physical frequencies arises because the potential in 
Eq. \ref{membrane-potential-tommaso} is not rotationally invariant. In order to clarify these results we have 
applied the SCHA numerically in the full potential. \\

By taking the full potential in Eq. \ref{membrane-potential}, Fourier transforming it and applying 
Eq. \ref{scha-equation} we arrive to the SCHA equations for the rotationally invariant membrane
\begin{multline}
 \frac{\partial\mathcal{F}(\mathcal{V})}{\partial\delta a}=0=2\Omega(2\delta a+3\delta a^{2}+\delta a^{3})(\lambda+
\mu)+\frac{1}{2}\sum_{\boldsymbol{q}}g[\Omega_{SCHA}^{(h)}(\boldsymbol{q})]2(1+\delta a)(\lambda+\mu)|\boldsymbol{
 q}|^{2}\\+\frac{1}{2}\sum_{\boldsymbol{q}}g[\Omega_{SCHA}^{(LA)}(\boldsymbol{q})][2(1+\delta a)(\lambda+2\mu)|\boldsymbol{q}|^{2}+2(1+\delta a)(\lambda+\mu)|\boldsymbol{q}|^{2}]+\\\frac{1}{2}\sum_{\boldsymbol{q}}g[\Omega_{
 SCHA}^{(TA)}(\boldsymbol{q})][2(1+\delta a)\mu|\boldsymbol{q}|^{2}+2(1+\delta a/2)(\lambda+\mu)|\boldsymbol{q}|^{2}],
\end{multline}
\begin{multline}
 \Phi_{h}(\boldsymbol{q})=\kappa|\boldsymbol{q}|^{4}+2(1+\delta a/2)\delta a(\lambda+\mu)|\boldsymbol{q}|^{2}+\frac{\lambda+2\mu}{2\Omega}\sum_{\boldsymbol{k}}g[\Omega_{SCHA}^{(h)}(\boldsymbol{k})][|\boldsymbol{q}|^{2}|\boldsymbol{
 k}|^{2}+2(\boldsymbol{q}\cdot\boldsymbol{k})^{2}]+\\\frac{1}{2\Omega}\sum_{\boldsymbol{k}}\{g[\Omega_{SCHA}^{(LA)}(\boldsymbol{k})]+g[\Omega_{SCHA}^{(TA)}(\boldsymbol{k})]\}[\lambda|\boldsymbol{q}|^{2}|\boldsymbol{k}|^{2}+2\mu(
 \boldsymbol{q}\cdot\boldsymbol{k})^{2}],
\end{multline}
\begin{multline}
 \Phi_{LA}(\boldsymbol{q})=(\lambda+2\mu)|\boldsymbol{q}|^{2}+2(1+\delta a/2)\delta a(\lambda+2\mu)|\boldsymbol{q}|^{2}+2(1+\delta a/2)\delta a(\lambda+\mu)|\boldsymbol{q}|^{2}\\+\frac{1}{2\Omega}\sum_{\boldsymbol{k}}g[\Omega_{SCHA}^{
 (h)}(\boldsymbol{k})][\lambda|\boldsymbol{q}|^{2}|\boldsymbol{k}|^{2}+2\mu(\boldsymbol{q}\cdot\boldsymbol{k})^{2}]+\\\frac{1}{4\Omega}\sum_{\boldsymbol{k}}\{4g[\Omega_{SCHA}^{(TA)}(\boldsymbol{k})][\lambda(\boldsymbol{q}\cdot\boldsymbol{
 k})^{2}+\mu|\boldsymbol{q}|^{2}|\boldsymbol{k}|^{2}+\mu(\boldsymbol{q}\cdot\boldsymbol{k})^{2}](\hat{\boldsymbol{q}_{\perp}}\cdot\hat{\boldsymbol{k}})+ \\ 2g[\Omega_{SCHA}^{(TA)}(\boldsymbol{k})][\lambda|\boldsymbol{q}|^{2}|\boldsymbol{
 k}|^{2}+2\mu(\boldsymbol{q}\cdot\boldsymbol{k})^{2}]+ \\ 2g[\Omega_{SCHA}^{(LA)}(\boldsymbol{k})][\lambda|\boldsymbol{q}|^{2}|\boldsymbol{k}|^{2}+2\mu(\boldsymbol{q}\cdot\boldsymbol{k})^{2}]+ \\ 4g[\Omega_{SCHA}^{(LA)}(\boldsymbol{
 k})][\lambda(\boldsymbol{q}\cdot\boldsymbol{k})^{2}+\mu|\boldsymbol{q}|^{2}|\boldsymbol{k}|^{2}+\mu(\boldsymbol{q}\cdot\boldsymbol{k})^{2}](\hat{\boldsymbol{q}}\cdot\hat{\boldsymbol{k}})\}
\end{multline}
and,
\begin{multline}
 \Phi_{TA}(\boldsymbol{q})=\mu|\boldsymbol{q}|^{2}+2(1+\delta a/2)\delta a\mu|\boldsymbol{q}|^{2}+2(1+\delta a/2)\delta a(\lambda+\mu)|\boldsymbol{q}|^{2}\\+\frac{1}{2\Omega}\sum_{\boldsymbol{k}}g[\Omega_{SCHA}^{
 (h)}(\boldsymbol{k})][\lambda|\boldsymbol{q}|^{2}|\boldsymbol{k}|^{2}+2\mu(\boldsymbol{q}\cdot\boldsymbol{k})^{2}]+\\\frac{1}{4\Omega}\sum_{\boldsymbol{k}}\{
 4g[\Omega_{SCHA}^{(TA)}(\boldsymbol{k})][\lambda(\boldsymbol{q}\cdot\boldsymbol{k})^{2}+\mu|\boldsymbol{q}|^{2}|\boldsymbol{k}|^{2}+\mu(\boldsymbol{q}\cdot\boldsymbol{k})^{2}](\hat{\boldsymbol{q}_{\perp}}\cdot\hat{
 \boldsymbol{k}_{\perp}})]+\\4g[\Omega_{SCHA}^{(LA)}(\boldsymbol{k})][\lambda(\boldsymbol{q}\cdot\boldsymbol{k})^{2}+\mu|\boldsymbol{q}|^{2}|\boldsymbol{k}|^{2}+\mu(\boldsymbol{q}\cdot\boldsymbol{k})^{2}](\hat{\boldsymbol{
 q}_{\perp}}\cdot\hat{\boldsymbol{k}})+ \\ 2g[\Omega_{SCHA}^{(TA)}(\boldsymbol{k})][\lambda|\boldsymbol{q}|^{2}|\boldsymbol{k}|^{2}+2\mu(\boldsymbol{q}\cdot\boldsymbol{k})^{2}]  \}.
\end{multline}
The subscripts $LA$, $TA$ correspond to the longitudinal and transversal acoustic modes. We have solved these 
equation by using the Newton-Raphson method in a circular discretized grid of $60\times60$ $\boldsymbol{q}$ 
points. The results is shown in Fig. \ref{membrane-results}. 
\begin{figure}[h]
\includegraphics[width=\linewidth]{Figures/membrane.eps}
\caption{Harmonic, SCHA and physical (static) ZA phonon frequencies at $T=0$ K and $T=100$ K in the membrane model.}
\label{membrane-results}
\end{figure}
As we can see, the SCHA frequencies are linear at any temperature and the linear term is bigger for higher 
temperatures. The physical phonons are on top of the harmonic values, which means that the correction for 
calculating the physical phonons is bigger for higher temperatures. This result clearly shows that the physical 
phonons are quadratic, as it is expected by symmetry. 
