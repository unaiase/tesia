% Chapter 1

\chapter{2D materials: Graphene} % Main chapter title

\label{Chapter8} % For referencing the chapter elsewhere, use \ref{Chapter1} 

%----------------------------------------------------------------------------------------

% Define some commands to keep the formatting separated from the content 
%\newcommand{\keyword}[1]{\textbf{#1}}
%\newcommand{\tabhead}[1]{\textbf{#1}}
%\newcommand{\code}[1]{\texttt{#1}}
%\newcommand{\file}[1]{\texttt{\bfseries#1}}
%\newcommand{\option}[1]{\texttt{\itshape#1}}

%----------------------------------------------------------------------------------------

\section{Introduction}

The discovery of graphene\cite{novoselov2004electric,novoselov2005two,novoselov2005twoo,zhang2005experimental} proved 
the existence of 2D materials and launched their science and technology. Graphene is already a reality in different 
industrial products\cite{Kong2019path} that benefit from its fantastic properties. In particular, its mechanical and 
thermal properties are crucial for many of its current and future applications. For instance, graphene's uneven 
strength, stiffness, and lightness\cite{lee2008measurement} has been used to make stronger but lighter macroscopic 
objects, such as tennis rackets, shoes, and so on. Graphene, due to its very high thermal 
conductivity\cite{ghosh2008extremely}, has been already incorporated into electronic devices like mobile phones for 
efficient heat dissipation. \\

The understanding of the mechanical and thermal properties of graphene, and in general of any 2D material, 
is, however, far from trivial. Even the possibility of having crystalline order in 2D has been long 
questioned\cite{landau_statistical_physics,mermin1968crystalline}. Indeed, the mean square displacement calculated 
in the harmonic approximation diverges in the long wave-length limit, which means that the larger the sample of a 2D 
material the bigger the atomic displacements, which prevents any crystalline order\cite{landau_statistical_physics}. 
In Fig. \ref{rms_harmonic} we show the atomic displacements of carbon atoms in graphene as a function of the sample 
size, or the supercell used for the calculation.
\begin{figure}[h]
\includegraphics[width=0.8\linewidth]{Figures/rms-harmonic.eps}
\caption{Atomic displacements of carbon atoms in graphene calculated using the harmonic dispersion as a function 
	of the supercell size for different temperatures.}
\label{rms_harmonic}
\end{figure}
As the figure clearly shows, the mean square atomic displacements diverge at finite temperatures as a function of 
the sample size. Mermin's theorem\cite{mermin1968crystalline} proves that, even without assuming the harmonic 
approximation, long-range crystalline order is not possible in a two-dimensional crystal described by a pair 
potential. Experimentally, however, crystalline order has been observed in suspended 
graphene\cite{meyer2007structure}, although it shows ripples that seem to be intrinsic\cite{fasolino2007intrinsic}. \\

Most of the problems related to the mechanical stability of graphene are caused by the quadratic dispersion of the 
acoustic out-of-plane (ZA) mode that is obtained in the harmonic approximation. The harmonic phonon spectrum of 
graphene is shown in Fig. \ref{harmonic-graphene}.
\begin{figure}[h]
\includegraphics[width=0.55\linewidth]{Figures/harmonic-graphene.eps}
\includegraphics[width=0.45\linewidth]{Figures/harmonic-graphene-logarithmic.eps}
\caption{Left: Harmonic phonon spectrum of graphene. Right: Harmonic ZA phonon dispersion of graphene in logarithmic 
	scale. We include a quadratic line for reference.}
\label{harmonic-graphene}
\end{figure}
We can see that the LA and TA modes have a linear dispersion close to the point $\Gamma$ as it happens in 3D 
crystals. However, the ZA mode, which atomic displacements are directed in the out-of-plane direction, has a 
quadratic dispersion. This is very clear in the plot in logarithmic scale where we have included a quadratic line for 
reference. The quadratic dispersion is given by the rotational symmetry, together with the fact that in a strictly 
two-dimensional system force constants involving an in-plane and an out-of-plane displacement 
vanish\cite{katsnelson2013graphene}. The mathematical proof of the quadratic dispersion is given in 
section \ref{quadratic-dispersion}. \\

The quadratic dispersion also creates spurious divergences in physical properties calculated perturbativaly on top of 
the harmonic result. The phonon linewidths of the in-plane acoustic longitudinal (LA) and transverse (TA) phonons 
calculated perturbatively do not vanish as the momentum decreases\cite{paulatto2013anharmonic}, precisely, because of 
the $\sim q^2$ dispersion of the ZA modes\cite{bonini2012acoustic}. The linewidths of the LA and TA modes are shown 
in Fig. \ref{lata-harmonic-lw}. 
\begin{figure}[h]
\includegraphics[width=0.8\linewidth]{Figures/lw-graphene-harmonic.eps}
\caption{Linewidths (full width at half maximum, FWHM) of the LA and TA modes calculated using the harmonic phonon 
spectrum for the three phonon scattering processes and 3BFC calculated within the SCHA at $0$ K.}
\label{lata-harmonic-lw}
\end{figure}
The figure clearly shown how the linewidths do not vanish at small momenta. This yields to the conclusion that for 
small enough momentum phonons do not live long enough for vibrating one period and the quasiparticle picture is lost. 
Consequently, the thermal conductivity calculated with the perturbative phonon 
lifetimes\cite{fugallo2013ab,fugallo2014thermal} does not converge with the sample size\cite{bonini2012acoustic}.
The application of strain linearizes the dispersion of the ZA phonons and makes the linewidth of the LA and TA 
phonon modes vanish together with their frequency at small momenta, keeping the phonon 
picture\cite{bonini2012acoustic}. \\

It has been argued\cite{wang2016anharmonic,los2009scaling,katsnelson2013graphene,zakharchenko2009finite,mariani2008flexural,amorim2014thermodynamics,de2012bending} that the anharmonic coupling between in-plane and out-of-plane phonon 
modes renormalizes the dispersion of the ZA phonon modes, providing it with a linear term at small momenta that 
somewhat cures the pathologies. This conclusion has been drawn by Monte Carlo simulations with empirical 
potentials\cite{wang2016anharmonic,los2009scaling,katsnelson2013graphene,zakharchenko2009finite} and by using a 
membrane continuum  Hamiltonian that mimics the interaction among acoustic 
modes\cite{mariani2008flexural,amorim2014thermodynamics,de2012bending}. Anharmonic phonons expected experimentally 
should be calculated from the free energy Hessian, i.e., diagonalizing the  $[\frac{\partial F}{\partial \mathcal{R}_a \partial \mathcal{R}_b}]_{0}/\sqrt{M_aM_b}$ matrix, where $F$ is the free energy calculated including anharmonic 
effects and $\boldsymbol{\mathcal{R}}$ the centroid positions that determine the most probable ionic 
positions\cite{bianco2017second}. If this is the case, as $F$ obeys the same symmetry properties as $V$, a similar 
$\sim q^2$ dispersion would be expected for the ZA mode even if anharmonic effects are included in the calculation 
of $F$. Actually, measurements done with helium diffraction show a ZA mode with a quadratic 
dispersion\cite{al2016acoustic,al2015helium,al2018resolving}, though the linearization regime may not be seen and 
substrate effects may be important. The remaining question is thus whether the ZA modes really have a quadratic 
dispersion, and, if it is so, how the mechanical stability and thermal properties of graphene can be explained. \\

In this work we show that a quadratic dispersion of the ZA mode is actually expected for graphene provided that it 
is calculated from the Hessian of the anharmonic free energy $F$, and that it is compatible with the absence of 
divergences. We estimate $F$ within the self-consistent harmonic approximation (SCHA). We apply the SCHA in its 
stochastic implementation (SSCHA) making use of a machine learning atomistic potential trained with density 
functional theory\cite{rowe2018development}. We also solve the SCHA equations in a membrane continuum Hamiltonian 
which provides results at smaller momenta.

\section{Quadratic dispersion of ZA modes within the harmonic approximation}
\label{quadratic-dispersion}

In this section we prove that the dispersion of the ZA phonon modes close to the point $\Gamma$ is quadratic in the 
harmonic approximation. \\

For that purpose, by using the inversion symmetry and the ASR in Eq. \ref{acoustic-sum-rule} we will rewrite Eq. \ref{2bfc-fourier-simple} in section \ref{harmonic-approximation} as
\begin{multline}
 \phi_{s_{1}s_{2}}^{\alpha_{1}\alpha_{2}}(\mathbf{q})=\sum_{\mathbf{T}}\phi_{s_{1}s_{2}}^{\alpha_{1}\alpha_{2}}(\mathbf{T},0)e^{-i\mathbf{q}\cdot\mathbf{T}}=\\\frac{1}{2}\sum_{\mathbf{T}}\phi_{s_{1}s_{2}}^{\alpha_{1}\alpha_{2}}(\mathbf{T},0)(e^{-i\mathbf{q}\cdot\mathbf{T}}+e^{i\mathbf{q}\cdot\mathbf{T}}-2)=-2\sum_{\mathbf{T}}\phi_{s_{1}s_{2}}^{\alpha_{1}\alpha_{2}}(\mathbf{T},0)sin^{2}(\mathbf{q}\cdot\mathbf{T}/2).
\end{multline}
And by Taylor expanding this equation close to the point $\Gamma$ we get
\begin{equation}
\label{need-zero}
\phi_{s_{1}s_{2}}^{\alpha_{1}\alpha_{2}}(\mathbf{q})\simeq -\frac{1}{2}\sum_{\alpha\beta}q_{\alpha}q_{\beta}\sum_{\mathbf{T}}\phi_{s_{1}s_{2}}^{\alpha_{1}\alpha_{2}}(\mathbf{T})T_{\alpha}T_{\beta}+O(q^{4}),
\end{equation} 
$\alpha$, $\beta$ being $x$, $y$, and $z$ Cartesian components. \\

The fact that a material is bidimensional and the equilibrium atomic positions are contained in the $z=0$ plane,
makes that all the 2BFC of the type $\phi_{s_{1}s_{2}}^{xz}(\mathbf{T})$, $\phi_{s_{1}s_{2}}^{yz}(\mathbf{T})$ are zero. This is translated into the fact that the out-of-plane $z$ direction is an eigendirection of the system and the 
harmonic frequency of the ZA mode is given by
\begin{equation} 
\omega_{ZA}^{2}(\mathbf{q})=\sum_{s_{2}}D_{1s_{2}}^{zz}(\mathbf{q}). 
\end{equation} 
Therefore, by looking at Eq. \ref{need-zero} we can see that for proving that the dispersion of the ZA modes is quadratic we need to prove that 
$\sum_{\mathbf{T}}\phi_{s_{1}s_{2}}^{zz}(\mathbf{T})T_{\alpha}T_{\beta}=0$. Actually, this equality is enforced by the rotational invariance of the system. \\

In conclusion, the 2D character of the 2BFC and the rotational invariance of the system make the harmonic dispersion of the ZA modes close to the point $\Gamma$ to be quadratic.

\section{Graphene without stress}

In order to calculate phonon spectra in unstrained graphene at any temperature, we calculate the SCHA stress tensor 
following the procedure in section \ref{scha-stress-section} and pick the lattice parameter that sets it to zero at 
each temperature. For that purpose, we calculate the stress in ranges of $0.0005$ $\AA$ and then we interpolate the 
result, which is fitted very good with a linear function. The lattice parameter calculated in this way includes 
anharmonic effects as well as the effect of quantum and thermal fluctuations. In order to properly account for 
thermal expansion, all the phonon spectra shown in this work that are obtained with the atomistic potential are 
calculated with the lattice parameter that gives a null SCHA stress at each temperature. The harmonic spectra on the 
contrary is always calculated at the lattice parameter that minimizes $U$. The temperature dependence of the lattice 
parameter and the thermal expansion coefficient are shown in Fig. \ref{lattice}.
\begin{figure}[ht]
\includegraphics[width=0.99\linewidth]{Figures/lattice.eps}
\caption{a) Lattice parameter of graphene as a function of temperature obtained with the SCHA using a machine 
	learning atomistic potential. Both quantum (black) and classical (blue) calculations are included. The 
	classical result is calculated setting $\hbar=0$ in the SCHA free energy. The temperature-independent frozen 
	nuclei (FN) result corresponds to the lattice parameter that minimizes the Born Oppenheimer potential $U$. 
	Results obtained by Rowe et al.\cite{rowe2018development}are also inlcuded for comparison. Solid black and 
	blue lines correspond to cubic fits. The black dashed line corresponds to the quasiharmonic result. b) 
	Thermal expansion coefficient (CTE) calculated as $CTE=\frac{1}{A}\frac{\partial A}{\partial T}$, $A$ being 
	the area of the membrane. Red line is directly taken from \cite{rowe2018development}. Black and blue lines 
	are calculated using the cubic fits in a).}
\label{lattice}
\end{figure}
We include the results of Rowe et al.\cite{rowe2018development}, which do not account for quantum effects as they 
are done with molecular dynamics (MD). They use the same potential as we do. For comparison we also include SCHA 
calculations in he classical limit and within the quasiharmonic (QH) approximation. Our quantum calculations 
correctly capture the negative thermal expansion of graphene up to $\sim$ 750 K that has been estimated in previous 
theoretical works\cite{rowe2018development,zakharchenko2009finite}. Our quantum calculations show a larger lattice 
parameter and a more pronounced thermal expansion (in absolute value) at low temperatures. This is not surprising as 
classical calculations neglect quantum fluctuations and, consequently, underestimate the fluctuations associated to 
the high-energy optical modes (the highest energy phonon modes require temperatures of around 2000 K to be thermally 
populated). Actually, the classical result approaches the quantum at high temperatures. This remarks the importance 
of considering quantum effects in the evaluation of thermodynamical properties of graphene. Our classical results 
and the MD calculations of Rowe et al.\cite{rowe2018development} are in agreement (within their error of $0.003$ 
$\AA$) at low temperatures. However, at high temperatures they deviate and our calculations approach the quantum 
result as it should be. The difference may be due to the non-trivial calculation of the lattice parameter within MD. 
We have also included the quasiharmonic (QH) result. The lattice parameter within the QH approximation is calculated 
by calculating the harmonic free energy for different temperatures and lattice paramters. The lattice parameter 
dependence comes from the different harmonic phonons at different lattice parameters. Then, we take the lattice 
parameter that minimizes the harmonic free energy at each temperature. It is worth noting that this approximation is 
not valid to calculate the thermal expansion of graphene due to the imaginary phonon frequencies that appear close 
to $\Gamma$ for the ZA mode already at $500$ K. In the shadowed region in Fig. \ref{lattice} the QH is not 
valid. This result questions the results obtained with this approach that yield a negative thermal expansion at all 
temperatures\cite{mounet2005first}.

\section{Graphene phonons}

In Fig. \ref{spectrum-graphene} we compare the harmonic phonon spectra with the one obtained diagonalizing the SCHA 
effective force constants $\boldsymbol{\Phi}$ as well as the spectra obtained from the Hessian of the SCHA free 
energy. 
\begin{figure*}[ht]
\includegraphics[width=0.32\linewidth]{Figures/T0.eps}
\includegraphics[width=0.32\linewidth]{Figures/T300.eps}
\includegraphics[width=0.32\linewidth]{Figures/T2000.eps}
\caption{Harmonic phonon spectra together with the SCHA phonons (labeled as ``SCHA'') and those obtained from the 
	Hessian of the SCHA free energy as (labeled as ``Physical''). Results at 0 K (a), 300 K (b), and 2000 K (c) 
	are shown. (d), (e) and (f) show only the dispersion of the ZA mode close to $\Gamma$ in logarithmic scale 
	at 0 K, 300 K, and 2000 K, respectively. In the small panels the dispersion corresponds to the $\Gamma$M 
	direction. For reference, the M point is at $1.4662$ $\AA^{-1}$ at $0$ K, at $1.4671$ $\AA^{-1}$ at $300$ K, 
	and at $1.4652$ $\AA^{-1}$ at $200$ K. The harmonic result (solid black) is computed at the lattice parameter 
	that minimizes $U$, while the other results include thermal expansion (see Fig. \ref{lattice}). The dashed 
	black lines correspond to harmonic calculations including thermal expansion (TE). All these calculations are 
	performed with the machine learning atomistic potential.}
\label{spectrum-graphene}
\end{figure*}
The main conclusion is that while the dispersion of the ZA modes obtained from $\boldsymbol{\Phi}$ is linearized, 
the physical phonons given by the Hessian of $F$ become close to a quadratic dispersion and approach the harmonic 
dispersion, as expected by symmetry. The SCHA ZA frequencies suffer a blue-shift with respect to the harmonic ones, 
but are red-shifted once the Hessian is calculated. Both shifts are bigger when the temperature is increased, but 
the quadratic behavior of the ZA modes is always recovered regardless of the temperature. With respect to the optical 
modes, we observe that the frequencies red-shift with increasing temperature in agreement with experiments and 
theory\cite{calizo2007temperature,bonini2007phonon}. It is noteworthy that the phonon frequencies obtained from the 
Hessian of $F$ should only be understood as the physical phonons in the static limit\cite{bianco2017second}. However, 
as shown in Fig. \ref{static-dynamic}, 
\begin{figure}[ht]
\includegraphics[width=0.32\linewidth]{Figures/T0-za.eps}
\includegraphics[width=0.32\linewidth]{Figures/T300-za.eps}
\includegraphics[width=0.32\linewidth]{Figures/T2000-za.eps}
\caption{Harmonic, SCHA, and physical phonons (static and dynamic) calculated at 0 K a), 300 K b), and 2000 K c).}
\label{static-dynamic}
\end{figure}
the static phonons agree with those derived from the dynamical (see section \ref{perturbation-theory-third}) theory 
and are thus good representatives of the physical phonons expected experimentally.

\section{SCHA applied to a continuum membrane Hamiltonian}

In this section we apply the SCHA in a continuum membrane Hamiltonian in order to get some results at very small 
momenta and reinforce our conclusions from the atomistic model. This model has been very much used in the 
literature to acount for the coupling between in-plane and out-of plane acoustic modes of 
graphene\cite{mariani2008flexural,amorim2014thermodynamics,de2012bending}. The most general rotationally invariant 
continuum model potential for phonons in free-standing 2D membranes up to the fourth-order with respect to the 
phonon fields has the following form:
\begin{multline}
 \label{membrane-potential}
 V=\frac{1}{2}\int_{\Omega}d^{2}x[\kappa(\partial^{2}h)^{2}+C^{ijkl}\partial_{i}u_{j}\partial_{k}u_{l}+C^{ijkl}\partial_{i}u_{j}\partial_{k}h\partial_{l}h+\\ +\frac{C^{ijkl}}{4}\partial_{i}h\partial_{j}h\partial_{
 k}h\partial_{l}h+\frac{C^{ijkl}}{2}\partial_{i}\boldsymbol{u}\cdot\partial_{j}\boldsymbol{u}\partial_{k}h\partial_{l}h+\\+C^{ijkl}\partial_{i}u_{j}\partial_{k}\boldsymbol{u}\cdot\partial_{l}\boldsymbol{u}+
 \frac{C^{ijkl}}{4}\partial_{i}\boldsymbol{u}\cdot\partial_{j}\boldsymbol{u}\partial_{k}\boldsymbol{u}\cdot\partial_{l}\boldsymbol{u}].
\end{multline}
$\boldsymbol{u}(\boldsymbol{x})$ and $h(\boldsymbol{x})$ are the in-plane and out-of-plane displacement 
fields, respectively, and $\boldsymbol{x}$ is the 2D position vector in the membrane. $\kappa$ is the bending 
rigidity of the membrane and the tensor 
$C^{ijkl}=\lambda\delta^{ij}\delta^{kl}+\mu(\delta^{ik}\delta^{jl}+\delta^{il}\delta^{jk})$ contains the Lame 
coefficients $\lambda$ and $\mu$ and Kronecker deltas. As it is a continuum model it only accounts for acoustic 
modes and the second-order terms in the potential are the harmonic terms. The harmonic frequencies are 
$\omega_{ZA}(q)=\sqrt{\kappa/\rho}q^{2}$, $\omega_{LA}(q)=\sqrt{(\lambda+\mu)/\rho}q$, and 
$\omega_{TA}(q)=\sqrt{\mu\rho} q$, $\rho$ being the mass density of the membrane. The higher-order terms account 
for the phonon-phonon anharmonic interactions. The thermal expansion is included in this formalism by changing the 
in-plane derivatives $\partial_{i}u_{j}\rightarrow \partial_{i}u_{j}+\delta^{ij}\delta a$, with 
$\delta a=(a-a_{0})a_{0}$, $a_{0}$ being the lattice parameter that minimizes $V$. \\

It is not possible to apply the SCHA analytically in Eq. \ref{membrane-potential}. The simplest approximation that 
allows applying the SCHA analytically is to neglect fourth-order terms including in-plane phonon fields. In that 
case the potential can be written as
\begin{multline}
\label{membrane-potential-tommaso}
V=\frac{1}{2}\int_{\Omega}d^{2}x[C^{ijkl}\partial_{i}u_{j}\partial_{k}u_{l}+\kappa(\partial^{2}h)^{2}+
C^{ijkl}\partial_{i}u_{j}\partial_{k}h\partial_{l}h+\frac{C^{ijkl}}{4}\partial_{i}h\partial_{j}h\partial_{k}h\partial_{l}h]+\\+2\Omega(\lambda+\mu)(\delta a)^{2}+\delta a(\lambda+\mu)\int_{\Omega}d^{2}x\partial_{k}h\partial_{k}h.
\end{multline}
By Fourier transforming this potential in $\boldsymbol{q}$ space and applying Eq. \ref{scha-equation}, we arrive to 
the SCHA equations (see appendix for further derivations)
\begin{equation}
\label{tommaso-eq1}
\frac{\partial\mathcal{F}}{\partial\delta a}=0\rightarrow\delta a=-\frac{1}{4\Omega}\sum_{\boldsymbol{q}}|\boldsymbol{q}|^{2}g[\Omega_{SCHA}^{(h)}(\boldsymbol{q})],
\end{equation}   
\begin{equation}
\label{tommaso-eq2}
\frac{\partial\mathcal{F}}{\partial\Phi_{h}(\boldsymbol{q})}=0\rightarrow\Phi_{h}(\boldsymbol{q})=\kappa|\boldsymbol{q}|^{4}+2\delta a(\lambda+\mu)|\boldsymbol{q}|^{2}+\frac{\lambda+2\mu}{2\Omega}\sum_{\boldsymbol{k}}g[\Omega_{SCHA}^{(h)}(\boldsymbol{k})][|\boldsymbol{q}|^{2}|\boldsymbol{k}|^{2}+2(\boldsymbol{q}\cdot\boldsymbol{k})^{2}].
\end{equation}
$\Phi_{h}(\boldsymbol{q})$ are the SCHA 2BFC that correspond to the out-of-plane modes and the frequencies are 
$\Omega_{SCHA}^{(h)}(\boldsymbol{q})=\sqrt{\Phi_{h}(\boldsymbol{q})/\rho}$. The function $g$ is defined as
$g(x)=coth(x/2T)/2\rho x$, $T$ being the temperature. Already from Eq. \ref{tommaso-eq2} it can be seen that the 
non-zero $\delta a$ provides the SCHA frequency with a linear term. By inserting Eq. \ref{tommaso-eq1} in 
Eq. \ref{tommaso-eq2} the SCHA 2BFC can be written as 
$\Phi_{h}(\boldsymbol{q})=\kappa|\boldsymbol{q}|^{4}+\gamma|\boldsymbol{q}|^{2}$. The expression for $\gamma$ is given in appendix. Therefore, we have shown that the SCHA flexural acoustic modes have a linear dispersion close to the points $\Gamma$. \\

For calculating the physical phonons, given by the second derivative of the free energy, an analogous formula to 
Eq. \ref{free-energy-hessian} can be found within the membrane formalism (see appendix). By applying this formula 
we arrive to the following expression for the physical phonons at $T=0$
\begin{equation}
 \Phi_{F}^{(h)}(\boldsymbol{q})=\kappa|\boldsymbol{q}|^{2}+(\gamma-\sigma)|\boldsymbol{q}|^{2}+
 O(|\boldsymbol{q}|^{4}).
\end{equation} 
The formula of $\sigma$ is given in the appendix. The positive number $\sigma$ makes the linear term in the physical 
phonon frequencies $40\%$ smaller than in the SCHA case, however, it does not remove it. Actually, it can be argued 
that this non-zero linear term in the physical frequencies arises because the potential in 
Eq. \ref{membrane-potential-tommaso} is not rotationally invariant. In order to clarify these results we have 
applied the SCHA numerically in the full potential. \\

By taking the full potential in Eq. \ref{membrane-potential}, Fourier transforming it and applying 
Eq. \ref{scha-equation} we arrive to the SCHA equations for the rotationally invariant membrane
\begin{multline}
 \frac{\partial\mathcal{F}(\mathcal{V})}{\partial\delta a}=0=2\Omega(2\delta a+3\delta a^{2}+\delta a^{3})(\lambda+
\mu)+\frac{1}{2}\sum_{\boldsymbol{q}}g[\Omega_{SCHA}^{(h)}(\boldsymbol{q})]2(1+\delta a)(\lambda+\mu)|\boldsymbol{
 q}|^{2}\\+\frac{1}{2}\sum_{\boldsymbol{q}}g[\Omega_{SCHA}^{(LA)}(\boldsymbol{q})][2(1+\delta a)(\lambda+2\mu)|\boldsymbol{q}|^{2}+2(1+\delta a)(\lambda+\mu)|\boldsymbol{q}|^{2}]+\\\frac{1}{2}\sum_{\boldsymbol{q}}g[\Omega_{
 SCHA}^{(TA)}(\boldsymbol{q})][2(1+\delta a)\mu|\boldsymbol{q}|^{2}+2(1+\delta a/2)(\lambda+\mu)|\boldsymbol{q}|^{2}],
\end{multline}
\begin{multline}
 \Phi_{h}(\boldsymbol{q})=\kappa|\boldsymbol{q}|^{4}+2(1+\delta a/2)\delta a(\lambda+\mu)|\boldsymbol{q}|^{2}+\frac{\lambda+2\mu}{2\Omega}\sum_{\boldsymbol{k}}g[\Omega_{SCHA}^{(h)}(\boldsymbol{k})][|\boldsymbol{q}|^{2}|\boldsymbol{
 k}|^{2}+2(\boldsymbol{q}\cdot\boldsymbol{k})^{2}]+\\\frac{1}{2\Omega}\sum_{\boldsymbol{k}}\{g[\Omega_{SCHA}^{(LA)}(\boldsymbol{k})]+g[\Omega_{SCHA}^{(TA)}(\boldsymbol{k})]\}[\lambda|\boldsymbol{q}|^{2}|\boldsymbol{k}|^{2}+2\mu(
 \boldsymbol{q}\cdot\boldsymbol{k})^{2}],
\end{multline}
\begin{multline}
 \Phi_{LA}(\boldsymbol{q})=(\lambda+2\mu)|\boldsymbol{q}|^{2}+2(1+\delta a/2)\delta a(\lambda+2\mu)|\boldsymbol{q}|^{2}+2(1+\delta a/2)\delta a(\lambda+\mu)|\boldsymbol{q}|^{2}\\+\frac{1}{2\Omega}\sum_{\boldsymbol{k}}g[\Omega_{SCHA}^{
 (h)}(\boldsymbol{k})][\lambda|\boldsymbol{q}|^{2}|\boldsymbol{k}|^{2}+2\mu(\boldsymbol{q}\cdot\boldsymbol{k})^{2}]+\\\frac{1}{4\Omega}\sum_{\boldsymbol{k}}\{4g[\Omega_{SCHA}^{(TA)}(\boldsymbol{k})][\lambda(\boldsymbol{q}\cdot\boldsymbol{
 k})^{2}+\mu|\boldsymbol{q}|^{2}|\boldsymbol{k}|^{2}+\mu(\boldsymbol{q}\cdot\boldsymbol{k})^{2}](\hat{\boldsymbol{q}_{\perp}}\cdot\hat{\boldsymbol{k}})+ \\ 2g[\Omega_{SCHA}^{(TA)}(\boldsymbol{k})][\lambda|\boldsymbol{q}|^{2}|\boldsymbol{
 k}|^{2}+2\mu(\boldsymbol{q}\cdot\boldsymbol{k})^{2}]+ \\ 2g[\Omega_{SCHA}^{(LA)}(\boldsymbol{k})][\lambda|\boldsymbol{q}|^{2}|\boldsymbol{k}|^{2}+2\mu(\boldsymbol{q}\cdot\boldsymbol{k})^{2}]+ \\ 4g[\Omega_{SCHA}^{(LA)}(\boldsymbol{
 k})][\lambda(\boldsymbol{q}\cdot\boldsymbol{k})^{2}+\mu|\boldsymbol{q}|^{2}|\boldsymbol{k}|^{2}+\mu(\boldsymbol{q}\cdot\boldsymbol{k})^{2}](\hat{\boldsymbol{q}}\cdot\hat{\boldsymbol{k}})\}
\end{multline}
and,
\begin{multline}
 \Phi_{TA}(\boldsymbol{q})=\mu|\boldsymbol{q}|^{2}+2(1+\delta a/2)\delta a\mu|\boldsymbol{q}|^{2}+2(1+\delta a/2)\delta a(\lambda+\mu)|\boldsymbol{q}|^{2}\\+\frac{1}{2\Omega}\sum_{\boldsymbol{k}}g[\Omega_{SCHA}^{
 (h)}(\boldsymbol{k})][\lambda|\boldsymbol{q}|^{2}|\boldsymbol{k}|^{2}+2\mu(\boldsymbol{q}\cdot\boldsymbol{k})^{2}]+\\\frac{1}{4\Omega}\sum_{\boldsymbol{k}}\{
 4g[\Omega_{SCHA}^{(TA)}(\boldsymbol{k})][\lambda(\boldsymbol{q}\cdot\boldsymbol{k})^{2}+\mu|\boldsymbol{q}|^{2}|\boldsymbol{k}|^{2}+\mu(\boldsymbol{q}\cdot\boldsymbol{k})^{2}](\hat{\boldsymbol{q}_{\perp}}\cdot\hat{
 \boldsymbol{k}_{\perp}})]+\\4g[\Omega_{SCHA}^{(LA)}(\boldsymbol{k})][\lambda(\boldsymbol{q}\cdot\boldsymbol{k})^{2}+\mu|\boldsymbol{q}|^{2}|\boldsymbol{k}|^{2}+\mu(\boldsymbol{q}\cdot\boldsymbol{k})^{2}](\hat{\boldsymbol{
 q}_{\perp}}\cdot\hat{\boldsymbol{k}})+ \\ 2g[\Omega_{SCHA}^{(TA)}(\boldsymbol{k})][\lambda|\boldsymbol{q}|^{2}|\boldsymbol{k}|^{2}+2\mu(\boldsymbol{q}\cdot\boldsymbol{k})^{2}]  \}.
\end{multline}
The subscripts $LA$, $TA$ correspond to the longitudinal and transversal acoustic modes. We have solved these 
equation by using the Newton-Raphson method in a circular discretized grid of $60\times60$ $\boldsymbol{q}$ 
points. The results is shown in Fig. \ref{membrane-results}. 
\begin{figure}[h]
\includegraphics[width=\linewidth]{Figures/membrane.eps}
\caption{Harmonic, SCHA and physical (static) ZA phonon frequencies at $T=0$ K and $T=100$ K in the membrane model.}
\label{membrane-results}
\end{figure}
As we can see, the SCHA frequencies are linear at any temperature and the linear term is bigger for higher 
temperatures. The physical phonons are on top of the harmonic values, which means that the correction for 
calculating the physical phonons is bigger for higher temperatures. This result clearly shows that the physical 
phonons are quadratic, as it is expected by symmetry. 
