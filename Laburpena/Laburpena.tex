% Appendix A

\chapter*{Laburpena} % Main appendix title

Efektu termoelektrikoak aplikazio ugari ditu. Tenperatura gradiente bat boltai elektriko batean bihur dezake Seebeck 
efektuaren bitartez. Alderantzizko norantzan boltai bat tenperatura gradiente batean bilakatzea ere posible da. 
Material termoelektriko onenek ere oso eraginkortasun baxua dute aipaturako prozesuetan. Arrazoi honengatik oso kasu 
espezifikoetan erabiltzen den efektua da efektu termoelektrikoa. Esate baterako, historikoki asko erabili den 
aplikazio bat NASA-ren espazio ontzietako radioisotopo sorgailu elektrikoak izan dira. Arrazoi hauek direla eta, 
material termoelektrikoetan gehien jarratizen diren ikerkuntza lerroak ondorengoak dira:
\begin{enumerate}
\item Material termoelektriko eraginkorragoak bilatu eta sintetizatu.
\item Eraginkortasun termoelektrikoa handitzen duten mekanismo fisiko eta kimiko berriak bilatu. 
\end{enumerate}
Material termoelektriko baten eraginkortasuna $figure$ $of$ $merit$ delakoak neurtzen du. Ondorengoa da bere formula:
\begin{equation}
\nonumber
ZT=S^{2}\sigma T/\kappa.
\end{equation}
$ZT$ dimentsiorik gabeko magnitude bat da. $S$ Seebeck koefizientea da, $\sigma$ eroankortasun elektrikoa, T 
tenperatura eta $\kappa$ eroankortasun termikoa. Materialen $ZT$a 3 ingurukoa izango balitz tenperatura eremu handi 
batean, material termoelektrikoak konpetentzia egingo liekete errekuntza motorrei. Baina, orokorrean materialen $ZT$a 
2 baino baxuagoa da eta oso tenperature eremu mugatuetan. Propietate hauek asko mugatzen dituzte material hauen 
aplikazioak. \\

Potentzia faktorea $PF=S^{2}\sigma$ Seebeck koefizientearen karratuaren eta eroankortasun elektrikoaren arteko
biderkadura da. Material termoelektriko onak bilatzeko ideia oso sinplea da, potentzia faktore handia behar dugu 
eroankortasun termiko baxuarekin batera. Ideia sinple honen arazorik handiena propietateen arteko korrelazioa da. 
Esate baterako, materialak dopatu egiten baditugu, eroankortasun elektrikoa handitu egiten da baina, Seebeck 
koefizientea txikitu. \\

Modu desberdinak daude materialen eraginkortasun termoelektrikoa handitzeko. Materialak dopatu edo nanoegituratu 
egin daitezke, potentzia faktore handi bat lortzeko eroankortasun termiko baxu batekin batera. Esperimentalki ikusi 
da ere fase trantsizioetatik gertu $ZT$a asko handitu daitekeela 2.5 inguruko balioak lortuz. 
