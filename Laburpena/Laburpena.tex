% Appendix A

\chapter*{Laburpena} % Main appendix title

Efektu termoelektrikoak aplikazio ugari ditu. Tenperatura gradiente bat boltai elektriko bihur dezake Seebeck 
efektuaren bitartez. Alderantzizko norantzan, boltai elektriko bat tenperatura gradiente batean bilakatzea ere 
posible da. Material termoelektriko onenek ere oso eraginkortasun baxua dute aipatutako prozesuetan. Arrazoi 
honengatik oso kasu espezifikoetan erabiltzen den efektua da efektu termoelektrikoa. Esate baterako, historikoki 
asko erabili den aplikazio bat NASA-ren espazio ontzietako radioisotopo sorgailu 
elektrikoa\cite{yang2006thermoelectric} izan da. Arrazoi hauek direla eta, material termoelektrikoetan gehien 
jarraitzen diren ikerkuntza lerroak ondorengoak dira:
\begin{enumerate}
\item Material termoelektriko eraginkorragoak bilatu eta sintetizatu.
\item Eraginkortasun termoelektrikoa handitzen duten mekanismo fisiko eta kimiko berriak bilatu eta aztertu. 
\end{enumerate}
Material termoelektriko baten eraginkortasuna $figure$ $of$ $merit$ delakoak neurtzen du. Ondorengoa da bere formula:
\begin{equation}
\nonumber
ZT=S^{2}\sigma T/\kappa.
\end{equation}
$ZT$ dimentsiorik gabeko magnitude bat da. $S$ Seebeck koefizientea da, $\sigma$ eroankortasun elektrikoa, T 
tenperatura eta $\kappa$ eroankortasun termikoa. Materialen $ZT$a 3 ingurukoa izango balitz tenperatura eremu handi 
batean, material termoelektrikoak konpetentzia egingo liekete errekuntza motorrei\cite{zhang2015thermoelectric}. 
Baina, orokorrean materialen $ZT$a 2 baino baxuagoa da eta oso tenperatura eremu mugatuetan. Propietate hauek asko 
mugatzen dituzte material hauen aplikazioak. \\

Potentzia faktorea $PF=S^{2}\sigma$ Seebeck koefizientearen karratuaren eta eroankortasun elektrikoaren arteko
biderkadura da. Material termoelektriko onak bilatzeko ideia oso sinplea da, potentzia faktore handia behar dugu 
eroankortasun termiko baxuarekin batera. Ideia sinple honen arazorik handiena propietateen arteko korrelazioa da. 
Esate baterako, materialak dopatu egiten baditugu, eroankortasun elektrikoa handitu egiten da baina, Seebeck 
koefizientea txikitu. \\

Modu desberdinak daude materialen eraginkortasun termoelektrikoa handitzeko. Materialak 
dopatu\cite{kim2013engineered,pei2011stabilizing,heremans2008enhancement} edo 
nanoegituratu\cite{vineis2010nanostructured,minnich2009bulk} 
egin daitezke, potentzia faktore handi bat lortzeko eroankortasun termiko baxu batekin batera. Azken bi metodo 
hauen adibiderik garbiena PbTe materiala da. Esperimentalki ikusi da ere fase trantsizioetatik gertu $ZT$a asko handitu daitekeela\cite{liu2013ultrahigh} 2.5 inguruko balioak lortuz, Cu$_{2}$Se-n gertatzen den moduan. \\

Azken urteetan ikusi da eroankortasun termiko baxua duten erdieroale intrintsekoak oso interesgarriak direla 
propietate termoelektrikoei dagokienez\cite{zhao2014ultralow,he2018remarkable}. Eroankortasun termikoa eroankortasun 
termiko elektroniko eta sare atomikoari dagokion eronakortasun termikoaren arteko batura da. Erdieroale 
intrintsekoetan, eroankortasun termikoaren zatirik handiena bibrazio atomikoetatik dator, tenperatura arruntetan ez 
daude karga elektriko aske asko eskuragarri eta. Beraz, erronka nagusia sareari dagokion eroankortasun termiko baxua 
duten erdieroaleak aurkitzea da. Propietate hau duen materialik eraginkorrena SnSe\cite{zhao2014ultralow} da eta 
bere $ZT$a 2.6 ingurukoa da 900 K-tan. Ez da soilik propietate hauek dituen materialik termoelektrikorik 
eraginkorrena, baizik eta gaur egun ezagutzen den materialik termoelektrikorik eraginkorrena. \\

Nahiz eta SnSe material termoelektriko eraginkorrena den, hainbat gauza ez dira ondo ulertzen oraindik. SnSe-k, 
trantsizio fase bat pairatzen 800 K inguruan simetria eta tenperatura baxuko fasetik $Pnma$, simetria eta tenperatura 
altuko fasera $Cmcm$. Ez dago argi trantsizio fase hau lehen edo bigarren ordenakoa den\cite{zhao2014ultralow,
adouby1998structure,chattopadhyay1986neutron,chatterji2018soft}. Material honen propietate 
termoelektriko interesagarrienak tenperatura altuko fasean agertzen dira, non gap elektronikoa txikitu egiten den, 
eroankortasun elektrikoa handituz. Aldi berean, oso eroankortasun termiko baxua mantentzen du. Esperimentu ezberdinak
ez dira ados jartzen SnSe-ren eroankortasun termikoaren balioari dagokionez. Zhao eta bere 
lankideek\cite{zhao2014ultralow} erakutsi zuten SnSe-k 0.3 W/mK inguruko eroankortasun termiko isotropoa duela 
tenperatura altuetan (800-900 K). Beste lan 
batzuek\cite{ibrahim2017reinvestigation,sassi2014assessment,chen2014thermoelectric}, esate baterako Ibrahim eta bere 
lankideenak, erakutsi dute SnSe-ren eroankostasun termikoa ez dela isotropoa eta balioa 1 W/mk ingurukoa dela 
800 K-tan. Eroankortasun termikoaren kalkulu teorikoei dagokienez, simetria altuko faseak fonoi ezegonkorrak 
ditu hurbilketa harmonikoan\cite{dewandre2016two,skelton2016anharmonicity,yu2016enhanced}, beraz, perturbazio 
teoria erabilita ezinekoa da eroankortasun termikoa kalkulatzea. \\

Tesi honetako bosgarren kapituluan ikusi dugu SnSe-k pairatzen duen fase trantsizioa bigarren ordenakoa dela. Hain 
zuzen ere trantsizioa gertatzen den tenperaturan trantsizioa gauzatzen duen fonoiaren maiztasuna 0 bilakatzen delako. 
Kalkulatutako fase trantsizio tenperatura asko aldatzen da gelaxkaren bolumenarekin eta erabilitako 
pseudopotentzialaren arabera. Ikusi dugunez, bolumenarekiko dependentzia hau oso ohikoa da mota honetako 
materialetan. Ikusi dugu ere, fonoi batzuei dagozkien funtzio espektralak ez dutela zerikusirik funtzio Lorentziar 
batekin. Honek argi uzten du efektu anarmonikoak ikaragarriak direla material honetan. Emaitza hauek oso 
erabilgarriak izango dira etorkizunen esperimentalentzat, forma ez Lorentziarrak asko zailtzen baitu fonoien 
deskribapen esperimentala. Eronkortasun termikoari dagokionez, lehenik eta behin, ikusi dugu efektu ez 
perturbatiboak ezinbestekoak direla esperimentalekin bat datozen balioak lortzeko. Anisotropiari dagokionez, ikusi 
dugu SnSe-ren eroankortasun termikoa oso anisotropikoa dela. Gure emaitzen arabera, Zhao eta bere lankideek egindako 
esperimentuetan efektu ez intrintsekoren bat egon behar da, horren balio isotropo eta baxuak lortzeko. \\

SnS eta SnSe oso material antzekoak dira propietate elektroniko eta bibrazionalei\cite{guo2015first} dagokienez. Bi 
erdieroaleek pairatzen dute fase trantsizio bat $Pnma$ fasetik $Cmcm$ 
fasera\cite{chattopadhyay1986neutron,von1981high} eta bi materialek dute eroankortasun termiko baxu bat tenperatura 
baxuko fasean. SnS-ren propietate termoelektrikoak ez dira neurtu tenperatura altuko fasean, baina SnSe-rekin duen 
antzekotasuna kontuan hartuta, uste dugu SnS oso material termoelektriko eraginkorra izan daitekeela tenperatura 
altuko fasean. Esaldi hau baieztatzeko SnS-ren propietate elektroniko eta bibrazionalak kalkulatu ditugu SnSe-ren 
balioekin konparatzeko. \\

SnS-ren propietate elektronikoak eta bibrazionalak tesi honetako seigarren kapituluan aztertu ditugu.
Hain zuzen ere, espero bezala, tenperatura altuko fasean SnS eta SnSe-k propietatea termoelektriko oso antzekoak 
dituzte. SnS-k bigarren ordenako trantsizio fase bat pairatzen du SnSe-k bezalaxe. Fonoi batzuk funtzio espektral 
ez Lorentziarrak erakusten dituzte eta SnS-k eroankortasun termiko oso baxua du simetri altuko fasean. Ikusi dugu 
eroankortasun termiko honen balioa SnSe-ren eroankortasun termikoaren balioaren oso antzekoa dela. Propietate 
elektronikoak ere kalkulatu ditugu. Ikusi dugu bi materialen potentzia faktoreak oso antzekoak direla, behintzat
aintzat hartuz bi materialetan elektroien bizitza denbora antzekoa dela. Beraz, bi materialek potentzia faktore 
eta eroankortasun termiko antzekoak badituzte, ondorioztatzen dugu SnS oso material termoelektriko interesgarria 
izan daitekeela tenperatura altuko fasean. \\

SnSe eta SnS-k pairatzen dituzten moduko fase trantsizioetan, baliteke kasu batzuetan sistemaren inbertsio simetria 
apurtzea\cite{ribeiro2018strong,chang2016discovery}. Honek propietate fisiko askori irekitzen die atea, esate 
baterako, ferroelektrizitatea. Esperimentalki ikusi dute posible dela material ferroelektrikoak eraginkortasun 
termoelektriko handia izatea\cite{ribeiro2018strong,zhang2013high}, SnTe-n gertatzen den moduan. Aldi 
berean, esperimentalki ikusi da, material baten dimentsionalitateak guztiz alda lezakeela trantsizio 
ferroelektrikoaren tenperatura. \\

Xafla meheko material ferroelektrikoak garrantzitsuak dira gailu modernoen 
aplikazioetan\cite{lallart2011ferroelectrics}. Aplikazioen aldetik, arazo bat da materialak gero eta meheagoak 
egitea, lodiera batetik behera, depolarizazio eremuak egoera ferroelektrikoa 
suntsitu\cite{batra1973new,zhong1994giant,dawber2005physics} egiten du eta. Efektu honek, egoera ferroelektrikorako 
trantsizio tenperatura txikitzen du lodierarekin batera eta oinarrizko limite bat jartzen die aplikazio 
teknologikoei. SnSe-ren geruza bakarrak\cite{chang2020controlled} inbertsio simetria apurtzen duen fase trantsizio 
bat jasa lezake. Beraz, oso material interesgarria da aplikazio teknologikoen ikuspuntutik. Izatez, badaude orain 
dela gutxi burututako esperimentuak\cite{chang2020controlled} non SnSe-ren polarizazio elektrikoa norabide batetik 
bestera aldatzen duten giro tenperaturan. \\

SnSe-ren monogeruzak, solteko SnSe-ren egitura berdina dauka, baina gelaxka unitaten bi geruza atomiko ditu lau 
beharrean, material soltean gertatzen den moduan. Arrazoi honengatik 4 atomo ditu gelaxka unitatean eta ez 8. Ez 
daude neurketa esperimentalak trantsizio ferroelektrikoarentzat, baina, badirudi giro tenperaturan $Pnm2_{1}$ 
egituran kristalizatzen dela. Beraz, trantsizio tenperaturak 300 K baina handiagoa behar du izan. Hainbat lan 
teorikoren araberan\cite{mehboudi2016structural,barraza2018tuning} trantsizio tenperatura 200 eta 320 K artean dago. 
Propietate termoelektrikoei dagokienez, ez daude neurketa esperimentalak simetria altuko fasearentzat, baina kalkulu 
teorikoen arabera\cite{wang2015thermoelectric,hu2017high} simetria baxuko monogeruzak eranginkortasun handiagoa izan 
lezake material solteak baino. Simetria altuko fasearentzat ez daude kalkulu teorikoak fonoi ezegonkorrak ditu eta 
hurbilketa harmonikoan. \\

Tesi honetako zazpigarren kapituluan SnSe monogeruzaren propietate bibrazionalak aztertzen ditugu. Simetria altuko 
monogeruza aztertzen dugu, $Pnmm$ simetria duena. Gure kalkuluen arabera, SnSe monogeruzak fase trantsizio 
ferroelektriko bat pairatzen du 107 K-tan. Tenperatura horretan, inbertsio simetria apurtzen duen fonoi baten 
maiztasuna 0 bilakatzen da eta. Aipatu dugun moduan, esperimentalki trantsizio tenperaturak 300 K baino handiagoa 
izan behar du. Litekeena da ezadostasuna supergelaxka edota bolumen efektu bat izatea, material solteetan gertatzen 
den moduan. Eroankortasun termikoa ere saiatu gara kalkulatzen SnSe-ren monogeruzan, baina materialaren bi 
dimentsiotako egiturarekin lotutako arazo bat topatu dugu. Arazo hau tesiko azken kapituluan aztertu dugu. \\

Grafenoaren aurkikuntzak\cite{novoselov2004electric,novoselov2005two,novoselov2005twoo,zhang2005experimental} 2 
dimentsiotako materialen existentzia frogatu zuen eta euren zientzia eta teknologia bultzatu. Bi dimentsiotako 
materialen propietate mekanikoak eta termikoak ez dira tribialak. Izatez, historikoki luzaroan pentsatu izan da 
orden kristalinoa ez dela posible bi dimentsiotan\cite{landau_statistical_physics,mermin1968crystalline}. 
Desplazamendu atomikoak dibergitu egiten dute uhin luzera handien limitean hurbilketa harmonikoan. Honek esan nahi 
du, bi dimentsiotako materiala gero eta handiagoa izan, handiagoak izango direla desplazamendu atomikoak, edozein 
orden kristalino ezinezko eginez. Aldiz, esperimentalki badirudi orden kristalinoa posible dela bi dimentsioetan. \\

Aipatutako arazoak planotik kanpoko modu akustikoen (ZA) dispertsio kuadratikoagatik agertzen dira. Dispertsio 
kuadratikoa hurbilketa harmonikoan lortzen da. Dispertsio hau simetria errotazionalak sortzen du indar konstanteen 
2 dimentsiotako izaerarekin batera. Arrazoi honengatik, hain zuzen ere, ZA moduaren maiztasunak baxuagoak dira eta 
maiztasun irudikariak lortzen ditutu SnSe monogeruzaren kasuan Fourier interpolazioa erabiltzean eroankortasun 
termikoa kalkulatzeko. Dispertsio kuadratikoak ez ditu soilik arazoak sortzen desplazamendu atomikoetan. Plano 
barruko fonoien desintegrazioa dela eta planoz kanpoko fonoietan, planoko fonoi akustikoen (LA, TA) bizitza denbora 
ez da handitzen fonoi hauen uhin luzera handitzen den einean\cite{bonini2012acoustic}. Honek esan nahi du, behar 
bezain uhin luzera handi batentzako, fonoi hauek fonoi izaera galtzen dutela ez direlako bibrazio periodo bat 
burutzeko behar bezainbeste denbora bizitzen. Honek esan nahi du ere soinua ezin dela hedatu grafenoan. \\

Hainbat lan teorikotan\cite{wang2016anharmonic,los2009scaling,katsnelson2013graphene,zakharchenko2009finite,mariani2008flexural,amorim2014thermodynamics,de2012bending} argudiatu da, efektu anarmonikoek ZA moduen dispertsioa linearizatu 
dezaketela. Honek aipaturiko arazo guztiak konponduko lituzke. Ikusiko dugun moduan hau ez dator bat teorikoki espero 
den emaitzarekin. Esperimentalki neurtutako fonoiak funtzio espektrala erabiliz kalkulatu behar dira. Energia baxuko 
fonoiek kasuan, funtzio espektralaren maximoak bat datoz energia asketik definitutako fonoien energiekin. Izatez, 
energia askeak, sistemaren simetria errotazional guztiak ditu, eta beraz, fonoi kuadratikoak espero ditugu. Hau 
horrela da, energia asketik datozen fonoiak energia askearen bigarren deribatutik kalkulatzen 
direlako, kasu harmonikoan gertatzen den bezalaxe Born Oppenheimer potentzialarekin. \\

Tesi honen azken kapituluan efektu anarmonikoak aztertu ditugu grafenoaren ZA moduan. Nahiz eta grafenoan soilik 
egin kalkuluak, emaitzak guztiz orokorrak dira edozein material bidimentsionalentzat. Ikusi dugu efektu anharmonikoak 
ezinbestekoak direla grafenoaren egonkortasun mekanikoa eta soinuaren hedapena lortzeko. Efektu anarmonikoak 
aztertzeko SSCHA aplikatu dugu grafenoan. SSCHA-n, energia askea minimizatzen duten fonoiak erabiltzen dira 
desplazamendu atomikoak eta fonoien desintegrazio energia mailak kalkulatzeko. Modu honetan definitutako 
SSCHA ZA fonoiak linealak dira, beraz, desplazamendu atomikoen dibergentziak asko txikitzen dira eta fonoien bizitza 
denborak handitu egiten dira euren uhin luzera handitzen den einean. Nahiz eta SSCHA fonoiak linealak izan, 
lehentxeago aipatu dugu esperimentalki esperotako fonoiak funtzio espektraletik kalkulatu behar direla. Kalkulu 
hauxe egin dugu SSCHA-a erabilita eta ikusi dugu modu honetan definitutako ZA fonoiak kuadratikoak direla. Beraz, 
ez dago desadostasunik simetria eta efektu anarmonikoen bateratzean, hau da, posible da simetriak ematen duen 
dispertsio kuadratikoa edukitzea eta desplazamendu atomikoak txikitzea aldi berean. Hauekin batera posible da ere 
grafenoan soinua hedatzea eta ZA fonoiek euren fonoi izaera berreskuratzea uhin luzera handientzako. Ikusi dugunez, 
oso garrantzitsua da grafenoak inolako estresik ez izatea dispertsio kuadratikoa izateko ZA fonoiek. Ondorio hauetara 
heltzeko, kalkulu atomistikoak egin ditugu grafeno xafla batean eta baita membrana modelo batean. Bi sistemetan 
emaitza guztiz bateragarriak lortu ditutgu. \\

Ideia guztiak paragrafo batean laburbilduz, tesi honetan efektu anarmonikoak aztertu ditugu teknologikoki 
garrantzitsuak eta interesgarriak diren material termoelektriko eta bidimentsionaletan. Aztertutako material 
guztietan, ikusi dugu efektu anarmonikoak ikaragarri garrantzitsuak direla esperimentuekin bat datozen emaitzak 
lortzeko eta azpian dagoen fisika ondo ulertzeko.
