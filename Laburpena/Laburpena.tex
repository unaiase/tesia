% Appendix A

\chapter*{Laburpena} % Main appendix title

Efektu termoelektrikoak aplikazio ugari ditu. Tenperatura gradiente bat boltai elektriko bihur dezake Seebeck 
efektuaren bitartez. Alderantzizko norantzan, boltai elektriko bat tenperatura gradiente batean bilakatzea ere 
posible da. Material termoelektriko onenek ere oso eraginkortasun baxua dute aipaturako prozesuetan. Arrazoi 
honengatik oso kasu espezifikoetan erabiltzen den efektua da efektu termoelektrikoa. Esate baterako, historikoki 
asko erabili den aplikazio bat NASA-ren espazio ontzietako radioisotopo sorgailu elektrikoak izan dira. Arrazoi 
hauek direla eta, material termoelektrikoetan gehien jarraitzen diren ikerkuntza lerroak ondorengoak dira:
\begin{enumerate}
\item Material termoelektriko eraginkorragoak bilatu eta sintetizatu.
\item Eraginkortasun termoelektrikoa handitzen duten mekanismo fisiko eta kimiko berriak bilatu eta aztertu. 
\end{enumerate}
Material termoelektriko baten eraginkortasuna $figure$ $of$ $merit$ delakoak neurtzen du. Ondorengoa da bere formula:
\begin{equation}
\nonumber
ZT=S^{2}\sigma T/\kappa.
\end{equation}
$ZT$ dimentsiorik gabeko magnitude bat da. $S$ Seebeck koefizientea da, $\sigma$ eroankortasun elektrikoa, T 
tenperatura eta $\kappa$ eroankortasun termikoa. Materialen $ZT$a 3 ingurukoa izango balitz tenperatura eremu handi 
batean, material termoelektrikoak konpetentzia egingo liekete errekuntza motorrei. Baina, orokorrean materialen $ZT$a 
2 baino baxuagoa da eta oso tenperature eremu mugatuetan. Propietate hauek asko mugatzen dituzte material hauen 
aplikazioak. \\

Potentzia faktorea $PF=S^{2}\sigma$ Seebeck koefizientearen karratuaren eta eroankortasun elektrikoaren arteko
biderkadura da. Material termoelektriko onak bilatzeko ideia oso sinplea da, potentzia faktore handia behar dugu 
eroankortasun termiko baxuarekin batera. Ideia sinple honen arazorik handiena propietateen arteko korrelazioa da. 
Esate baterako, materialak dopatu egiten baditugu, eroankortasun elektrikoa handitu egiten da baina, Seebeck 
koefizientea txikitu. \\

Modu desberdinak daude materialen eraginkortasun termoelektrikoa handitzeko. Materialak dopatu edo nanoegituratu 
egin daitezke, potentzia faktore handi bat lortzeko eroankortasun termiko baxu batekin batera. Esperimentalki ikusi 
da ere fase trantsizioetatik gertu $ZT$a asko handitu daitekeela 2.5 inguruko balioak lortuz. \\

Azken urteetan ikusi da eroankortasun termiko baxua duten erdieroale intrintsekoak oso interesgarriak direla 
propietate termoelektrikoei dagokienez. Eroankortasun termikoa eroankortasun termiko elektroniko eta sare atomikoari 
dagokion eronakortasun termikoaren arteko batura da. Erdieroale intrintsekoetan, eroankortasun termikoaren zatirik 
handiena bibrazio atomikoetatik dator, tenperatura arruntetan ez daude karga aske asko eskuragarri eta. Beraz, 
erronka nagusia sareari dagokion eroankortasun termiko baxua duten erdieroaleak aurkitzea da. Propietate hau duen 
materialik eraginkorrena SnSe da eta bere $ZT$a 2.6 ingurukoa da 900 K-tan. Ez da soilik propietate hauek dituen 
materialik termoelektrikorik eraginkorrena, baizik eta gaur egun ezagutzen den materialik eraginkorrena. \\

Nahiz eta SnSe material termoelektriko eraginkorrena den, hainbat gauza ez dira ondo ulertzen oraindik. SnSe-k, 
trantsizio fase bat pairatzen 800 K inguruan simetria eta tenperatura baxuko fasetik $Pnma$, simetria eta tenperatura 
altuko fasera $Cmcm$. Ez dago argi trantsizio fase hau lehen edo bigarren ordenakoa den. Material honen propietate 
termoelektriko interesagarrienak tenperatura altuko fasean agertzen dira, non gap elektronikoa txikitu egiten den, 
eroankortasun elektrikoa handituz. Aldi berean, oso eroankortasun termiko baxua mantentzen du. Esperimentu ezberdinak
ez dira ados jartzen SnSe-ren eroankortasun termikoaren balioari dagokionez. Zhao eta bere lankideek erakutsi zuten 
SnSe-k 0.3 W/mK inguruko eroankortasun terimko isotropoa duela tenperatura altuetan (800-900 K). Beste lan batzuek, 
esate baterako Ibrahim eta bere lankideenak, erakutsi dute SnSe-ren eroankostasun termikoa ez dela isotropoa eta 
balioa 1 W/mk ingurukoa dela 800 K-tan. Eroankortasun termikoaren kalkulu teorikoei dagokienez, simetria altuko 
gaseak fonoi ezegonkorrak dituene hurbilketa harmonikoan, beraz, perturbazio teoria erabilita ezinekoa da 
eroankortasun termikoa kalkulatzea. \\

Tesi honetako bosgarren kapituluan ikusi dugu SnSe-k pairatzen duen fase trantsizioa bigarren ordenakoa dela. Hain 
zuzen ere trantsizioa gertatzen den tenperaturan trantsizioa gauzatzen duen fonoiaren maiztasuna 0 bilakatzen delako. 
Kalkulatutako fase trantsizio tenperatura asko aldatzen da gelaxkaren bolumenarekin eta erabilitako 
pseudopotentzialaren arabera. Ikusi dugunez, bolumenarekiko dependentzia hau oso ohikoa da mota honetako 
materialetan. Ikusi dugu ere, fonoi batzuei dagozkien funtzio espektralak ez dutela zerikusirik funtzio Lorentziar 
batekin. Honek argi uzten du efektu anarmonikoak ikaragarriak direla material honetan. Emaitza hauek oso 
erabilgarriak izango dira etorkizunen esperimentalentzat, forma ez Lorentziarrak asko zailtzen baitu fonoien 
deskribapen esperimentala. Eronkortasun termikoari dagokionez, lehenik eta behin, ikusi dugu efektu ez 
perturbatiboak ezinbestekoak direla esperimentalekin bat datozen balioak lortzeko. Anisotropiari dagokionez, ikusi 
dugu SnSe-ren eroankortasun termikoa oso anisotropikoa dela. Gure emaitzen arabera, Zhao eta bere lankideek egindako 
esperimentuetan efektu ez intrintsekoren bat egon behar da, horren balio isotropo eta baxuak lortzeko. \\

SnS eta SnSe oso material antzekoak dira propietate elektroniko eta bibrazionalei dagokienez. Bi erdieroaleek 
pairatzen dute fase trantsizio bat $Pnma$ fasetik $Cmcm$ fasera eta bi materialak dute eroankortasun termiko baxu 
bat tenperatura baxuko fasean. SnS-ren propietate termoelektrikoak ez dira neurtu tenperatura altuko fasean, baina 
SnSe-rekin duen antzekotasuna kontuan hartuta, uste dugu SnS oso material termoelektriko eraginkorra izan daitekeela 
tenperatura altuko fasean. Esaldi hau baieztatzeko SnS-ren propietate elektroniko eta bibrazionalak kalkulatu ditugu 
SnSe-ren balioekin konparatzeko. \\

SnS-ren propietate elektronikoak eta bibrazionalak tesi honetako seigarren kapituluan aztertu ditugu.
Hain zuzen ere, espero bezala, tenpertaura altuko fasean SnS eta SnSe-k propietatea termoelektriko oso antzekoak 
dituzte. SnS-k bigarren ordenako trantsizio fase bat pairatzen du SnSe-k bezalaxe. Fonoi batzuek funtzio espektral 
ez Lorentziarrak erakusten dituzte eta eroankortasun termiko oso baxua du simetri altuko fasean. Ikusi dugu 
eroankortasun termiko honen balioa SnSe-ren eroankortasun termikoaren balioaren oso antzekoa dela. Propietate 
elektronikoak ere kalkulatu ditugu. Ikusi dugu bi materialen potentzia faktoreak oso antzekoak direla, behintzat
haintzat hartuz bi materialetan elektroien bizitza denbora antzekoa dela. Beraz, bi materialek potentzia faktore 
eta eronakortasun termikoa antzekoak badituzte, ondorioztatzen dugu SnS oso material termoelektriko interesgarria 
izan daitekeela tenperatura altuko fasean. \\

SnSe eta SnS-k pairatzen dituzten moduko fase trantsizioetan, baliteke kasu batzuetan sistemaren inbertsio simetria 
apurtzea. Honek propietate fisiko askori irekitzen die atea, esate baterako, ferroelektrizitatea. Esperimentalki 
ikusi dute posible dela material ferroelektrikoak eraginkortasun termoelektriko handia izatea, SnTe-n gertatzen 
den moduan. Aldi berean, esperimentalki ikusi da, material baten dimentsionalitateak guztiz alda lezakeela 
trantsizio ferroelektrikoaren tenperatura. \\

Xafla meheko material ferroelektrikoak garrantzitsuak dira gailu modernoen aplikazioetan. Aplikazioen aldetik, 
arazo bat da materialak gero eta meheagoak egitea, lodiera batetik behera, depolarizazio eremuak egoera 
ferroelektrikoa suntsitu egiten du eta. Efektu honek, egoera ferroelektrikorako trantsizio tenperatura txikitzen du 
lodierarekin batera eta oinarrizko limite bat jartzen die aplikazio teknologikoei. SnSe-ren geruza bakarrak inbertsio 
simetria apurtzen duen fase trantsizio bat jasa lezake. Beraz, oso material interesgarria da aplikazio teknologikoen 
ikuspuntutik. Izatez, badaude orain dela gutxi burututako esperimentuak non SnSe-ren polarizazio elektrikoa 
norabide batetik bestera aldatzen duten giro tenperaturan. \\

SnSe-ren monogeruzak, solteko SnSe-ren egitua berdina dauka, baina gelaxka unitaten bi geruza atomiko ditu lau 
beharren, material soltean gertatzen den moduan. Arrazoi honengatik 4 atomo ditu gelaxka unitatean eta ez 8. Ez 
daude neurketa esperimentalak trantsizio ferroelektrikoarentzat, baina, badirudi giro tenperaturan $Pnm2_{1}$ 
egituran kristalizatzen dela. Beraz, trantsizio tenperaturak 300 K baina handiagoa behar du izan. Hainbat lan 
teorikoren araberan trantsizio tenperatura 200 eta 320 K artean dago. Propietate termoelektrikoei dagokienez, ez 
daude neurketa esperimentalak simetria altuko fasearentzat, baina kalkulu teorikoen arabera simetria baxuko 
monogeruzak eranginkortasun handiagoa izan lezake material solteak baino. Simetria altuko fasearentzat ez daude 
kalkulu teorikoak fonoi ezegonkorrak ditu eta hurbilketa harmonikoan. \\

Tesi honetako zazpigarren kapituluan SnSe monogeruzaren propietate bibrazionalak aztertzen ditugu. Simmetria altuko 
monogeruza aztertzen dugu, $Pnmm$ simetria duena. Gure kalkuluen arabera, SnSe monogeruzak fase trantsizio 
ferroelektriko bat pairatzen du 107 K-tan. Tenperatura horretan, inbertsio simetria apurtze duen fonoi baten 
maiztasuna 0 bilakatzen da eta. Aipatu dugun moduan, esperimentalki trantsizio tenperaturak 300 K baino handiagoa 
izan behar du. Litekeena da ezadostasuna supergelaxka edota bolumen efektu bat izatea, material solteetan gertatzen 
de moduan. Eroankortasun termikoa ere saiatu gara kalkulatzen SnSe-ren monogeruzan baina materialaren bi 
dimentsiotako egiturarekin lotutako arazo bat topatu dugu. Arazo hau tesiko azken kapituluan aztertu dugu. \\

Grafenoaren aurkikuntzak 2 dimentsiotako materiales existentzia frogatu zuen eta euren zientzia eta teknologia 
bultzatu. Bi dimentsioatko materialen propietate mekanikoak eta termikoak ez dira tribialak. Izatez, historikoki 
luzaroan pentsatu izan da orden kristalinoa ez dela posible bi dimentsiotan. Desplazamendu atomikoak dibergitu 
egiten dute uhin luzera handien limitean hurbilketa harmonikoan. Honek esan nahi du, bi dimentsiotako materiala gero 
eta handiagoa izan, handiagoak izango direla desplazamendu atomikoak, edozein orden kristalino ezinezko eginez. 
Aldiz, esperimentalki badirudi orden kristalinoa posible dela bi dimentsioetan. \\

Aipatutako arazoak planotik kanpoko modu akustikoen (ZA) dispertsio kuadratikoagatik agertzen dira. Dispertsio 
kuadratikoa hurbilketa harmonikoan lortzen da. Dispertsio hau simetria errotazionalak sortzen du indar konstanteen 
2 dimentsiotako izaerarekin batera. Arrazoi honengatik, hain zuzen ere, ZA modularen maiztasunak baxuagoak dira eta 
maitasun irudikariak lortzen ditutu SnSe monogeruzaren kasuan Fourier interpolazioa erabiltzen eroankortasun termikoa 
kalkulatzeko. Dispertsio kuadratikoak ez ditu soilik arazoak sortzen desplazamendu atomikoetan. 
Plano barruko fonoiek desintegrazioa dela eta planoz kanpoko fonoietan, planoko fonoien (LA, TA) bizitza denbora ez 
da handitzen fonoi hauen uhin luzera handitzen de einean. Honek esan nahi du, behar bezain uhin luzera handi 
batentzako, fonoi hauek fonoi izaera galtzen dutela ez direlako bibrazio periodo bat burutzeko behar bezainbeste 
denbora bizitzen. Honek esan nahi du ere soinua ezin dela hedatu grafenoan. \\

Hainbat lan teorikotan argudiatu da, efektu anarmonikoek ZA moduen dispertsioa linearizatu dezaketela eta honek 
aipaturiko arazo guztiak konponduko lituzke. Ikusiko dugun moduan hau ez dator bat teorikoki espero den 
emaitzarekin. Esperimentalki neurtutako fonoiak funtzio espektrala erabiliz kalkuluatu behar dira. Energia baxuko 
fonoiek kasuan, funtzio espektralaren maximoak bat datoz energia asketik definitutako fonoien energiarekin. Izatez, 
energia askeak, sistemaren simetria errotazional guztiak ditu, eta beraz, fonoi kuadratikoak espero ditugu. \\

Tesis honen azken kapituluan efektu anarmonikoak aztertu ditugu grafenoaren ZA moduan. Nahiz eta grafenoan soilik 
egin kalkuluak, emaitzak guztiz orokorrak dira edozein material bidimentsionalentzat. Ikusi dugu efektu anharmonikoak 
ezinbestekoak direla grafenoaren egonkortasun mekanikoa soinuaren hedapena lortzeko. 
